\section{Methodology}\label{sec:method-mpchecker}


\begin{figure*}[h]
	\vspace{2mm}
    \centering
    \includegraphics[width=\linewidth]{Figures/Chapter4/arch.pdf}
    \caption{The architectural overview of \toolchecker.}
    \label{fig:arch}
    \vspace{-5pt}
\end{figure*}

In this section, we define the issue of code-documentation inconsistency caused by multi-parameter
constraints and provide a detailed description of our approach. An API documentation error is an inconsistency between the library source code and its API documentation. Multi-parameter constraints refer to conditional dependency relationships that exist among multiple parameters within functions or classes. If a constraint is never violated across all execution paths in the code, it is considered as a benign constraint, or it indicates a potential documentation error. According to literature~\cite{uddin2015api,zhou2017analyzing, zhu2022identifying}, API documentation inconsistency can be categorized into two types: incorrectness and incompleteness. Incorrectness refers to cases where the documentation describes behavior that is not implemented in the code, while incompleteness arises when certain code behaviors are not reflected in the documentation. Typically, incorrectness issues are considered more critical than incompleteness.

In addition, when it comes to constraint extraction, compared to single-parameter constraints, we need to classify the multi-parameter constraint extraction problem into two types, as discussed in Section~\ref{sec:eg-mpchecker}, 1) explicit constraint and 2) implicit constraint.

\toolchecker aims to accurately extract multi-parameter constraints from API documentation and detect both types of inconsistency. As the architecture depicted in Figure~\ref{fig:arch}, we have designed a three-phase workflow comprising the \textbf{1) Data Preprocessing; 2) Constraint Extraction; 3) Inconsistency Detection}. During the preprocessing phase, we separate the code and documentation within the project.
\toolchecker will then automatically rewrite each function to be compatible with the symbolic
execution tool. This includes replacing advanced Python syntax that the tool cannot handle with
simpler constructs and symbolizing external function calls, such as replacing ternary conditional
expressions with if-else statements. These modifications do not alter the path constraints of the
original program.
In the constraint extraction and expression generation phase, on the one hand, we leverage large language models to extract constraints in a specific format from the documentation. On the other hand, the symbolic execution tool dynamically analyzes the code and solves the constraint paths. Those constraints are then converted into expressions that can be processed by the SMT solver. In the fuzzy constraint checking phase (Phase III), a constraint checker with SMT solver and fuzzy constraint reasoner performs comprehensive reasoning to detect inconsistencies. It is worth noting that we propose and implement an extended fuzzy constraint satisfaction to mitigate the hallucination issues often introduced by large language models, and reduce the risk of false positives and missed detections.




\subsection{Preprocessing}\label{sec:preprocess-mpchecker}
In this step, we will discuss the details of separating the documentation and corresponding code from the project and the specifics of preprocessing the documentation content.

In modern data science libraries, documentation is typically auto-generated using Sphinx, a tool that can automatically create HTML documentation from Python code. Sphinx supports various docstring styles, with Google style and NumPy style being commonly used. \fref{fig:docstring} respectively display docstring examples of two different styles from Sphinx official website~\cite{sphnix:google,sphnix:numpy}. Google-style docstrings use a clear and concise format with a minimalistic structure. It divides the docstring into sections like \textbf{Args}, \textbf{Attributes}, etc., with each section using plain indentation. Similarly, Numpy-style docstrings organize sections more rigidly. Sections are divided by using \textbf{Parameters}, \textbf{Attributes}, etc. with horizontal dash lines ``\textbf{- - -}'' under the section header. The number of dashes is the same as the number of letters in the section header. Regardless of the style used, the docstring is normally placed at the beginning inside its corresponding class or function.

After downloading the project, our tool first converts every Python file from the project into an Abstract Syntax Tree (AST) and isolates the classes and independent functions. This paper focuses on the CDI issue, so in this step, we filter out code without documentation and separately extract the code and documentation from the remaining code. Since Python supports object-oriented programming but current symbolic execution tools have limited support for classes, we have to limit our experimental units to functions. For independent functions, the scope of constraints in the documentation usually applies within the function itself. For classes, however, the constraints cover the entire class, including each member function. Therefore, we create a new directory for every class and independent function, with member function directories placed within their corresponding class directories to maintain structural consistency. If the member function has its own documentation, it will also be retained. 

To help the LLM better focus on the constraints between parameters and reduce the occurrence of erroneous constraints, we retained parameters or attributes and their corresponding descriptions in the form of key-value pairs, based on the two aforementioned docstring styles. We subsequently applied a rule-based heuristic approach to retain documentation that potentially contain constraints and to discard the rest. For instance, if none of the parameters or attributes appear in other descriptions, this indicates the absence of multi-parameter constraints in that documentation.


\begin{figure*}[t]
	\vspace{2mm}
    \begin{subfigure}{0.49\linewidth}
        \begin{tcolorbox}[colback=Emerald!10,colframe=cyan!40!black,title=\textbf{Numpy Style Docstrings}]
\begin{lstlisting}[escapechar=@]
class ExampleNumpyStyle():
  """Exceptions are documented
  @\colorbox{blue!20}{Parameters}@
  ----------
  msg : str
    Human readable string describing the exception.
  code : obj:`int`, optional 
    Numeric error code.
  @\colorbox{blue!20}{Attributes}@
  ----------
  msg : str
    Human readable string describing the exception.
  code : int
    Numeric error code.
  """
  def __init__(self, msg, code):
    self.msg = msg
    self.code = code
\end{lstlisting}
        \end{tcolorbox}
        \label{fig:numpy}
    \end{subfigure}
    \begin{subfigure}{0.49\linewidth}
\begin{tcolorbox}[colback=Salmon!20, colframe=Salmon!90!Black,title=\textbf{Google Style Docstrings}]
\begin{lstlisting}[escapechar=@]
class ExampleGoogleStyle():
  """Exceptions are documented
  Note:
    Do not include the `self` parameter in the ``Args`` section.
        
  @\colorbox{blue!20}{Args:}@
    msg (str): Human readable string describing the exception.
    code (:obj:`int`, optional): Error code.

  @\colorbox{blue!20}{Attributes:}@    
    msg (str): Human readable string describing the exception.
    code (int): Exception error code
  """
  def __init__(self, msg, code):
    self.msg = msg
    self.code = code
\end{lstlisting}
\end{tcolorbox}
    \label{fig:google}
    \end{subfigure}
    \caption{Example of two docstring styles}
    \label{fig:docstring}
    \vspace{-10pt}
\end{figure*}



\subsection{Constraint Extraction}\label{sec:extraction}

We now specially explain how to extract path constraints from code and convert them into expressions which are solvable by SMT solver, as well as how to use LLM to extract constraints from documentation and transform them into expressions containing fuzzy words.

\subsubsection{Code Constraint Expression Extraction}

The goal of \tool is to verify whether the constraints between multiple parameters in documentation align with the logic during actual code execution. This requires our tool to understand and analyze deeper constraint relationships. Therefore, we employ symbolic execution to capture as many path conditions as possible and precisely handle complex paths and constraints.

We modified current advanced dynamic symbolic execution tools~\cite{github:pyexsmt, github:pyexz3, github:pysmt, ball2015deconstructing, bruni2011peer} for path exploration. Unfortunately, supporting dynamic languages like Python is more challenging compared to symbolic execution tools designed for static languages such as Java and C. Despite Python's rapid evolution, symbolic execution tools specifically designed for Python have developed slowly, struggling to keep pace with the growing new syntax and features. This forces us to make reasonable modifications to the source code extracted directly from repositories. However, these modifications must not alter the path constraints of the original code; they should be equivalent code transformations that do not affect path exploration. We mainly made the following modifications:


\begin{enumerate}
    \item Current Python symbol execution tool can not solve class directly. Therefore, it is necessary to split the class into functions (i.e. member functions). The corresponding member variables also need to be changed and used as symbolic inputs.
    \item Replace complex structures and operations, such as lists and dictionaries, that are difficult to handle and do not affect the path, as well as external function calls that may cause path explosion, with symbolic inputs.
    \item Replace the handling of exceptions and warnings that do not affect the path with \textit{return}.
    \item Add a fixed format of \textit{return} statement to capture concrete values of potential symbols.
    \item Equivalent code implementation replacement to avoid being unable to find useful path constraints due to poor support for some advanced syntax. For example, replace ternary operator to conventional if-else statement.
\end{enumerate}


In the limit, \tool strives to explore all feasible paths in a Python function by following these processes: 1) Running the function with specific input to trace a path through the control flow of the function; 2) Symbolic executing the path to determine how its conditions depend on the function's input parameters; 3) Utilizing Z3 to generate new parameter values that guide the function toward paths that haven't been covered yet.

Although \tool supports a certain level of external function call analysis, in complex real-world code, an external function call often corresponds to extra more function calls, leading to path explosion. Furthermore, documentation constraints are usually handled within the target function, so we still prefer not to introduce external function calls and to focus the analysis within the target function. Additionally, similar to the current concolic symbolic execution tools for Python, \tool does not yet provide strong support for theorem of strings. Thus, during the actual execution process, we replace the string with a unique large number, which does not affect the exploration of condition constraints.


\begin{figure*}[t]
	\vspace{2mm}
    \begin{subfigure}{0.49\linewidth}
        \begin{tcolorbox}[colback=Salmon!20, colframe=Salmon!90!Black,title=\textbf{Original source code}]
\begin{lstlisting}[escapechar=@]
def fit(self, sample_weight):
  if sample_weight is not None and self.strategy == "uniform":
    raise ValueError("Warning Info")
  if sample_weight is not None:
    sample_weight = _check_sample_weight(sample_weight, X)
\end{lstlisting}
        \end{tcolorbox}
        \label{fig:sourcecode}
    \end{subfigure}
    \begin{subfigure}{0.49\linewidth}
        \includegraphics[width=\linewidth]{Figures/Chapter4/pathgraphpic.pdf}
        \vspace{-8pt}
        \label{fig:pathgraph}
    \end{subfigure}
    \begin{subfigure}{0.98\linewidth}
\begin{tcolorbox}[colback=OliveGreen!10,colframe=Green!70,title=\textbf{Modified source code}]
\begin{lstlisting}[escapechar=@,basicstyle={\footnotesize\ttfamily}]
def fit(sample_weight, strategy, call__check_sample_weight):
  if sample_weight != 'None' and strategy == 'uniform':
    return '(sample_weight)_(strategy)_ERROR_END'
  if sample_weight != 'None':
    sample_weight = call__check_sample_weight
  return (f'(sample_weight = {sample_weight}) ^ (call__check_sample_weight =                         {call__check_sample_weight}) ^ (strategy = {strategy})')
\end{lstlisting}
\end{tcolorbox}
        \label{fig:modifiedcode}
    \end{subfigure}
    \caption{Extracting constraint from code}
    \label{fig:extractpath}
    \vspace{-5pt}
\end{figure*}

We will use an example depicted in Figure~\ref{fig:extractpath} to illustrate the entire extraction phase, containing a simplified original source code and modified code from a popular data science project \texttt{scikit-learn}, and its corresponding path constraints. The \textit{fit} function is a member function within a class, and thus member variables such as ``self.strategy'' also exist within the code. We also modified ``None'' as a string to make it easier to be captured, since it is represented as a number 0 during symbolic execution. Our tool first modifies the code and replaces exception handling and external function calls with symbolic inputs, marked as ``ERROR\_END'' and ``call\_'', respectively. For those paths whose final states are ``ERROR\_END'', the final results of the conjunction of the documentation constraint and these paths will be negated during reasoning phase.



\subsubsection{Documentation Constraint Expression Extraction}\label{sec:llm}

In this step, we extract constraints from Python documentation by applying LLMs.
Since Python documentation can vary in quality and may contain informal writing~\cite{rani2021comments}, the important task is to understand the parameter information within the documentation. To achieve this, we resort to SOTA LLMs. Given Python documentation as input, the LLM is asked to first extract constraint-related sentences and then output them in a standard logical expression format. This includes two steps, model selection and prompt design.


\paragraph{Model Selection} We adopt GPT-4, which is pretrained on a diverse corpus and shows
excellent performance in natural language understanding. Based on our preliminary study, GPT-4's
performance stands out compared to Gemini-1.5~\cite{gemini} and LLaMA-3~\cite{llama3} due to its
ability to capture details, and it is also well-acquainted with the context of code
documentation~\cite{dvivedi2024comparative}.


\paragraph{Prompt Design} Because the constraint extraction task is relatively complex and can be
broken down into clear steps, we apply the chain-of-thought approach~\cite{wei2024cot}, which has
been widely proven effective in improving GPT-based model performance. We first divide the prompt
task into two steps, document input and constraint extraction. Figure~\ref{fig:prompt} shows the
structure and some details of the used prompt.
Below, we detail our prompt mechanism for each step.

\paragraph{Document Input Prompt}
We observe that some documentation may be too lengthy to provide to GPT-4 in a single input, considering that GPT-4 has a maximum token length limit of 8,192 tokens~\cite{modeltoken}. We also find that LLMs exhibit lower performance when dealing with long and complex text inputs as noted in previous research~\cite{han2024lm, jin2024llm}. Thus, we decide to segment the lengthy documents into smaller sections. To determine a heuristic chunk size, we randomly select ten lengthy Python documentation files, split them into chunks of varying word lengths, and use these as inputs for GPT-4. We then evaluate the constraint extraction task performance of GPT-4 based on these inputs to determine which chunk size yields better results. Based on our findings, we decide to standardize the chunk size to 1,500 words (around 2,048 tokens~\cite{tokencount}).
We also input the parameter list obtained in Section~\ref{sec:preprocess} into GPT-4 to help model better recognize the information related to parameters. The details of the document input prompt are shown in Prompt 1 in Figure~\ref{fig:prompt}.

\paragraph{Constraint Extraction Prompt}
For the constraint extraction task, our prompt is divided into three parts to guide GPT-4 in recognizing text related to constraints in the original documentation, and then, based on that text, to generate a formatted logical expression of the constraint.

The first part involves defining the logical symbols that can be used in the logical format, including implication, negation NOT, logical AND, logical OR, and also defining parentheses to indicate the precedence of logical expressions.

The second part arises from our preliminary study, in which we observed that some Python documentation uses vague terms such as ``override'', ``specify'', ``have an effect'', ``no effect'', ``significant'', and ``ignore'' when mentioning constraints related to parameters. To preserve as much detail as possible from the documentation, we design prompts to guide GPT-4 so that if text related to parameter constraints contains vague keywords, these keywords should be retained in the final logical expression.

In the third part, to ensure that the format of the logical expression in GPT-4's output is consistent each time and convenient to process, we apply in-context learning techniques that widely used in previous works~\cite{min2022rethinking, rubin2021learning} to enable GPT-based models to handle tasks specific to a domain.
We include four examples that contain pairs of original constraint-related sentences selected from Python documentation and their corresponding logical expression constraints.


\begin{figure*}[h]
    \centering
    \includegraphics[width=\linewidth]{Figures/Chapter4/prompt_template.pdf}
    \caption{Prompt structure for constraints extraction}
    \label{fig:prompt}
    \vspace{-5pt}
\end{figure*}
\subsection{Inconsistency Detection}\label{sec:detect}

In the second phase, we extracted constraints from both documentation and code. While \emph{code-constraints} are deterministic in nature, \emph{doc-constraints} inherently contain uncertainties stemming from two main sources. First, there are \emph{implicit constraints} arising from vague or incomplete descriptions, which we address by defining fuzzy words to extend them into soft constraints. Second, we encounter uncertainties introduced by generative models' limited reasoning capabilities and unavoidable hallucination issues, for which no validator exists to definitively determine the correctness of generated constraints. To address this challenging scenario, we proposed a customized fuzzy constraint logic to mitigate such vagueness. With the help of fuzzy words and fuzzy constraint logic, our converter can effectively handle both explicit and implicit constraints. The converter ultimately produces fuzzy expressions, which are then processed by a reasoner based on the z3 SMT (Satisfiability Modulo Theories) solver to detect inconsistencies.


The reasoner offers two strategies, from relaxed to strict, to identify inconsistencies from the perspectives of satisfiability and equivalence. Given a \emph{doc-constraint} $c$ and a \emph{code-constraint} set $P$ containing a group of path constraints $p$, the detection strategies are defined as follows:
\begin{itemize}
	\item \textbf{Unsatisfiability} checking determines whether the \emph{doc-constraint} $c$ is unsatisfiable under all path constraints. If the conjunction of $c$ and every path constraint $p$ is unsatisfiable, it indicates a contradictory inconsistency, meaning that under all possible execution paths, the code violates the constraint stipulated in the documentation.
	\begin{equation}\label{eq: uq1}
		\forall p \in P, \neg (c \wedge p)
	\end{equation}
	\item \textbf{Nonequivalence} checking determines whether the \emph{doc-constraint} $c$ and \emph{code-constraints} are logically equivalent. If equivalence holds only under specific conditions, a behavioral inconsistency may arise, implying that the constraints implemented in the code are not fully equivalent to the documentation.
	\begin{equation}\label{eq: uq3}
		\exists p \in P, \neg (c \Leftrightarrow p)
	\end{equation}
\end{itemize}

For constraints containing sub-constraints that do not exist in the code logic, we employ a heuristic approach to provide suggestions. Specifically, when a constraint mentioned in the documentation is not present in the corresponding code logic, we issue a warning and label this constraint as a potential weak constraint to prompt further investigation by the user.



\subsubsection{Fuzzy Words}\label{sec:fuzzword}

The example from~\cref{sec:eg2} illustrates a very typical implicit constraint with fuzzy words, where part of constraint is clearly defined while others remain uncertain. We introduce a series of fuzzy words to help LLM extract constraints better. These fuzzy words frequently appear in documentation but don't represent specific values, making it challenging for the LLM to extract them directly. We generally categorize these fuzzy words into two types: \textit{existence} and \textit{non-existence}. In fuzzy words, \textit{non-existence} includes ``ignore'', ``no effect'', ``unused'', ``override'', indicating that a parameter either is unused or does not exist within code segments where other conditions are met. Similarly, \textit{existence} includes ``specify'', ``have an effect'', ``exist'', ``significant'', indicating that the parameter is used or exists when other conditions are met.

We implement several specialized predicates to evaluate such implicit constraints. \tool will first trace the target variable's define-use chain (DU-chain)~\cite{kennedy1978use, harrold1994efficient} and then check if the definition and usage of it are existed or not under a specific program path. For instance, ``\texttt{exist(x)}'' will check if the definition and usage of a variable \texttt{x} are existed in the program path with other explicit conditions. If found, it will return \texttt{True}; otherwise, it will return \texttt{False}. Similarly, ``\texttt{ignore(x)}'' will check whether the definition and usage of \texttt{x} are absent under the program path with given explicit conditions. It can be further extended to check weak \emph{doc-constraints}. Like the example in~\cref{sec:eg2}, \textit{gamma} will be ignored not only when \texttt{``affinity$=$nearest\_neighbors''} but also ignored when \texttt{``affinity$=$precomputed\_nearest\_neighbors''} as well as \texttt{``affinity$=$precomputed''}.



\subsubsection{Fuzzy Constraint Satisfaction}\label{sec:fuzzlogic}

While we have employed several strategies in Phase II to maximize LLM's understanding of
constraints and restrict randomness in outputs, inaccurate extraction is still unavoidable.
The main reasons are threefold: (1) typos inevitably occur when developers write documentation; (2)
the hallucination issues inherent to black-box generative models; and (3) the intrinsic ambiguity in
natural languages. This implies that correct documentation descriptions can generate incorrect \emph{doc-constraints}, and incorrect documentation descriptions can also have the chance to generate correct \emph{doc-constraints}.

In the absence of LLM unpredictability, detecting CDI issues is a crisp constraint satisfaction
problem (CSP), deciding whether a \emph{doc-constraint} is consistent with the actual code implementation.
Nevertheless, due to minor errors introduced by LLMs, such as a single letter being wrongly spelled
in a parameter name, or a comparison operator being reversed, e.g., writing ``<'' instead of ``>'',
a \emph{doc-constraint} can be mistakenly identified as inconsistent when it is actually correct.

To address this, we proposed a customized fuzzy constraint logic that reconciles such
unpredictability.
In a traditional fuzzy constraint~\cite{kosko1993fuzzy, ruttkay1994fuzzy}, a membership function
assigns a degree of satisfaction (ranging from 0 to 1) to each possible variable value.
It enables partial fulfillment of a condition, with satisfaction measured on a continuous scale.
In our case, a constraint needs to be measured on a new scale, assessing ``how likely'' the
extracted \emph{doc-constraint} conforms to the \emph{code-constraints}.
Therefore, we introduced a unique similarity computation which serves as the membership function.

\cref{fig:ebnf} shows an EBNF grammar for our multi-parameter constraints.
A multi-parameter constraint is a combination and nesting of binary expressions and Boolean
operators, which can be viewed as a complete binary tree where leaf nodes are binary expressions
over single parameters and non-leaf nodes are logical operators connecting them.
Without loss of generality, we only keep negation, conjunction, and disjunction in the constraints;
logical relations such as implications can be simplified accordingly.
The fuzziness of a constraint is defined with respect to a set of \emph{environment expressions},
facts that are known to hold (with a truth value of 1).
In other words, the instantiation of a specific tree structure and nodes is a constraint $c$
evaluated against a set of expressions $\{e_1, e_2, \ldots, e_n\}$.
Next, we define the membership function of our fuzzy constraint logic through a few similarity
functions.


\begin{figure}[t]
\small\centering
\begin{align*}
   c \in Constraint &::= e \; |\; \neg c \;|\; c \vee c \;|\; c \wedge c \nonumber\\
   e \in Expression &::= \; p \bowtie v\\
   p \in Parameter &::= \; char,\{char \;|\; digit\} \nonumber\\
   \bowtie \; \in Operator &::= \; \textbf{<} \;|\; \textbf{>} \;|\; \textbf{<=} \;|\;
   \textbf{>=} \;|\; \textbf{=} \;|\; \textbf{!=} \; \nonumber\\
   v \in Value &::= \; string \;|\; number \;|\; bool \nonumber
\end{align*}%
\caption{Extended Backus-Naur form for multi-parameter constraint.}\label{fig:ebnf}
\vspace{-5pt}
\end{figure}

\begin{definition}[Expression Similarity]\label{def:es}
The similarity between two expressions $e_1$ and $e_2$ is defined as,
\begin{equation}\label{eq:simexpr}
  \sigma(e_1, e_2) = \alpha * (1 - \frac{LD(p_1, p_2)}{\max(|p_1|, |p_2|)}) + \beta *
  (\frac{\delta_{\bowtie_1} \cdot \delta_{\bowtie_2}}{\|\delta_{\bowtie_1}\|
  \|\delta_{\bowtie_2}\|}) + \alpha * (1 - \frac{LD(v_1, v_2)}{\max(|v_1|, |v_2|)}),
\end{equation}
where $\alpha$ and $\beta$ denote the relative weights, $p$, $\bowtie$, and $v$
are parameter, operator, and value, respectively, $|p|$ and $|v|$ denotes the length of $p$ and $v$, $\|\delta_{\bowtie}\|$ denotes the magnitudes (or Euclidean norms) of the vector $\delta_{\bowtie}$.
\end{definition}

The similarity between two expressions are considered separately for the parameters, operators, and
values appeared in the expressions.
Both $p$ and $v$ can be treated as texts, therefore, Levenshtien Distance (a.k.a.
edit distance) is used to represent their similarity.
The normalized Levenshtien Distance (NLD) is given in \cref{eq:ld}, where $s$ denotes strings ($p$
or $v$) and $|s|$ denotes the length of it.

\begin{equation}\label{eq:ld}
    \eta(s_1, s_2) = NLD =  1 - \frac{LD(s_1, s_2)}{\max(|s_1|, |s_2|)}
\end{equation}

\begin{equation}\label{eq:embedding}
    \delta_{\bowtie} = (C, E, G, L, N), \text{where } C, E, G, L, N \in \{0, 1\}
\end{equation}

\begin{equation}\label{eq:cossim}
    cos\theta(\bowtie_1, \bowtie_2) = \frac{\delta_{\bowtie_1} \cdot \delta_{\bowtie_2}}{\|\delta_{\bowtie_1}\| \|\delta_{\bowtie_2}\|}
\end{equation}

As illustrated in \cref{eq:embedding}, we design an operator embedding across five key dimensions: \underline{\textbf{C}}omparison, \underline{\textbf{E}}quality, \underline{\textbf{G}}reater than, \underline{\textbf{L}}ess than, and \underline{\textbf{N}}egativity.
This way, we may calculate the similarity between two operators by simply calculating the cosine
similarity between two vectors.
The result is highly intuitive. For example, with $\delta_{<} = (1, 0, 0, 1, 0)$, $\delta_{>} =
(1, 0, 1, 0, 0)$, and $\delta_{<=} = (1, 1, 0, 1, 0)$, the similarity between ``$<$'' and ``$>$''
is 0.5, while the similarity between ``$<$'' and ``$<=$'' is 0.82.

The weight of operator similarity is set as $\beta$ such that the weights of operators, values, and
parameters within a given experssion should sum to one.
Thus, we have $\alpha = \frac{1 - \beta}{2}$ and the similarity $\sigma$ of two single parameter
expressions (i.e., atomic constraint) can be calculated according to \cref{eq:simexpr}.


\begin{definition}[Constraint Similarity]
Let $c$ be a constraint and $\Phi = \{e_i | i=1,\ldots,n\}$ be a set of
environment expressions assumed to hold true.
The similarity of $c$ against $\Phi$ is given by the following set of calculations.
\begin{equation}\label{eq:js}
    \rho(c,\Phi) =
    \begin{cases}
        \mathop{\arg\max}\limits_{e_i \in \Phi} \sigma(e, e_i),
        & \text{if } c \text{ is an expression } e \\
        1 - \sigma(c', \Phi), & \text{if $c = \neg c'$} \\
        \min\{\sigma(c_1, \Phi), \sigma(c_2, \Phi)\}, & \text{if } c = c_1 \wedge c_2 \\
        \max\{\sigma(c_1, \Phi), \sigma(c_2, \Phi)\}, & \text{if } c = c_1 \vee c_2 \\
    \end{cases}
\end{equation}
\end{definition}

Consider an atomic constraint with a single expression; its similarity to $\Phi$ associated with
a set of \emph{environment expressions} can be represented by the maximum expression similarity among
all expressions within $\Phi$.
Based on the \textit{conjunctive combination principle}~\cite{zadeh1965fuzzy}, when combining two
constraints using a conjunction, their degree of joint similarity $\rho$ should be represented by
the minimum similarity between them. Similarly, based on the \textit{disjunctive combination
principle}~\cite{zadeh1965fuzzy}, when they are combined with a disjunction, the maximum similarity
should be used. For negation, the complementary similarity is used.



\begin{definition}[Membership Function for Fuzzy Constraint Satisfaction]
Constraint similarity serves as the membership function $\mu_{\Omega}$, quantifying the degree to
which a given constraint $\epsilon$ is consistent with the code, which is represented as a set
$\Omega$ of path constraints $\omega$:
%\vspace{-8pt}
\begin{equation}
\mu_{\Omega}(\epsilon) = \rho(\epsilon, \Phi_\Omega) \cdot \epsilon[e \mapsto e_{\Phi_\Omega}]
\end{equation}
where $\Phi_\Omega$ denotes the set of expressions aggregated from all the path constraints in
$\Omega$, and $\epsilon[e \mapsto e_{\Phi_\Omega}]$ is a rewrite of $\epsilon$, where each
expression $e$ has been replaced by its cloest counterpart from $\Phi_\Omega$.
\end{definition}

The inconsistency between the modified constraint $\epsilon[e \mapsto e_{\Phi_\Omega}]$ and
$\Omega$ is then evaluated (according to \cref{eq: uq1} or
\cref{eq: uq3}), yielding a binary result (\texttt{True} or \texttt{False}).
To enable the probabilistic interpretation, a linear transformation
ensures complementary probabilities. For instance, ``0.7$\cdot$False = 0.3$\cdot$True'', indicating
a 70\% probability of inconsistency or a 30\% probability of consistency.


\subsubsection{Constraint Similarity Threshold.} LLMs demonstrate a great potential in constraint extraction, yet they still encounter errors such as using incorrect parameter names or values and introducing non-existent constraints. To reduce false positives from these inevitable issues, we set a constraint similarity threshold of 0.85. This is based on the observation that a high constraint similarity (>0.85) indicates a high likelihood of misinterpretation or conflation by the LLM.

For instance, in \texttt{scikit-learn}, the documentation of \texttt{LinearSVC} states: ``\emph{If n\_samples < n\_features and optimizer supports chosen loss, multi\_class and penalty, then dual will be set to True}''. However, the extracted constraint is ``\texttt{$($samples$<$features$)$$\wedge$$($dual$=$True$)$}'', where two parameters are mistakenly mapped to similar names. Their constraint similarity is 0.86, exceeding the threshold, leading \tool to discard the result. Moreover, in most cases where the expression contains only a single parameter, the threshold exhibits stronger filtering capability.

However, setting a threshold cannot entirely exclude all false positives, as there is no definitive rule to ascertain whether an error originates from the documentation or the LLM. Another example from \texttt{scikit-learn} illustrates this limitation: the documentation of \texttt{estimator\_} states: \emph{``The child estimator template used to create the collection of fitted sub-estimators''}. Yet, the extracted constraint ``\texttt{$($estimator\_$=$child\_estimator\_template$)\wedge($collection$=$fitted\_sub\_estimators$)$}'' has a constraint similarity of 0.67, which is below the threshold, leading \tool to accept the result.








