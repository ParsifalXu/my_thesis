%!TEX root=../mythesis.tex
% Chapter Template

\chapter{Conclusion and Future Work} % Main chapter title
\chaptermark{Conclusion}  % replace the chapter name with its abbreviated form
\label{ch:chapter6}
\section{Conclusion}
This thesis addresses the fundamental tension in modern software engineering: the need for rapid, AI-driven software evolution versus the requirement for deterministic reliability. While Large Language Models (LLMs) have ushered in a paradigm shift toward autonomous intelligence, their probabilistic nature often fails to meet the rigorous demands of interconnected dependency networks. To bridge this gap, this research introduces a neurosymbolic approach that anchors generative intelligence within formal logical frameworks, systematically addressing the three core pillars of evolutionary reliability.

\begin{itemize}
    \item In \hyperref[ch:chapter3]{\underline{\textbf{\cref{ch:chapter3}}}}, we introduce \toolcompsuite,  a dataset that includes 123 real-world Java client-library pairs where upgrading the library causes an incompatibility issue in the corresponding client. Each incompatibility issue in \toolcompsuite is associated with a test case authored by the developers, which can be used to reproduce the issue. The dataset also provides a command-line interface that simplifies the execution and validation of each issue. By providing a structured dataset for library upgrade incompatibilities, \toolcompsuite allows AI agents to move beyond anecdotal fixes toward a systematic understanding of how third-party dependency shifts propagate through client systems.
    
    \item In \hyperref[ch:chapter4]{\underline{\textbf{\cref{ch:chapter4}}}}, we developed \toolchecker for detecting inconsistencies between code and documentation, specifically focusing on multi-parameter constraints. \toolchecker identifies these constraints at the code level by exploring execution paths through symbolic execution and further extracts corresponding constraints from documentation using large language models (LLMs). We propose a customized fuzzy constraint logic to reconcile the unpredictability of LLM outputs and detect logical inconsistencies between the code and documentation constraints. This not only ensures internal semantic alignment but also provides a reliable interface contract for external dependencies, mitigating the risks of external usage.
    
    \item In \hyperref[ch:chapter5]{\underline{\textbf{\cref{ch:chapter5}}}}, we first identify a critical yet overlooked bias: existing real-world issue benchmarks are saturated with keyword references (e.g. file paths, function names), encouraging models to rely on superficial lexical matching rather than genuine structural reasoning. We term this phenomenon the Keyword Shortcut. To address this, we formalize the challenge of Keyword-Agnostic Logical Code Localization (KA-LCL) and introduce \dataset, a diagnostic benchmark requiring structural reasoning without any naming hints. Our evaluation reveals a catastrophic performance drop of state-of-the-art approaches on \dataset, exposing their lack of deterministic reasoning capabilities. We propose \tooldataloc, a novel agentic framework that combines large language models with the rigorous logical reasoning of Datalog for precise localization. \tooldataloc extracts program facts from the codebase and leverages an LLM to synthesize Datalog programs, with parser-gated validation and mutation-based intermediate-rule diagnostic feedback to ensure correctness and efficiency. The validated programs are executed by a high-performance inference engine, enabling accurate and verifiable localization in a fully automated, closed-loop workflow. 
\end{itemize}
In conclusion, this dissertation demonstrates that the path to reliable software evolution lies not in choosing between human-centered logic and AI-driven automation, but in their synthesis. By embedding neurosymbolic reasoning into the software evolution lifecycle, we provide a theoretical and practical foundation for building autonomous systems that are both as flexible as neural networks and as trustworthy as formal logic.

In conclusion, all of chapters in this dissertation offer a comprehensive exploration of how neurosymbolic approaches can enhance the reliability of software evolution in the age of AI. By systematically addressing the challenges of dependency management, code-documentation consistency, and precise code localization, this research lays the groundwork for a new paradigm of autonomous software maintenance that is both intelligent and dependable.


\section{Future Research}