\section{Dataset Usage}\label{sec:usage-compsuite}

In this section, we provide instructions on the usage of our dataset.

\subsection{Exploring an Incompatibility Issue}\label{sec:isolate-compsuite}
To ensure the reproducibility of incompatibility issues and to
facilitate the demonstration of such issues, we have annotated
checkpoints in the version histories of the client projects and
provided tags that guide users to explore any incompatibility issues
present in the \toolcompsuite dataset.

As illustrated on the right-hand side of \cref{fig:architecture}, our
approach to handling incompatible client-library pairs involved
creating a fork~\cite{gitfork} of the original client project for each
identified pair, while preserving all code and version history
information. To mark the base version of the project, we utilized the
\Code{git tag}~\cite{gittag} command, designating it as $V_{base}$. Subsequently, we developed a
patch to upgrade the library from its old version to its new version,
a simple process that can be accomplished with a single line change in
the \Code{pom.xml} file for Maven projects. This patch was then
applied to the $V_{base}$ version, resulting in a new version that we
identified as $V_{incomp}$. Notably, the only difference between
$V_{base}$ and $V_{incomp}$ lies in the library version used: the old
(compatible) version is utilized on $V_{base}$ while the new
(incompatible) version is utilized on $V_{incomp}$. For instance, in
\cref{fig:architecture}, the client project employs version 2.2.1 of
the \Code{org.restlet.jse-org.restlet} library on its $V_{base}$ and
version 3.0-M1 on its $V_{incomp}$. In cases where multiple libraries
exhibit incompatibility issues in the client project, we not only create different branches for each library with its name, but also generate a
$V_{incomp}$ version tag for each, with accompanying annotations that
denote the corresponding library name and version, as depicted in
\cref{fig:architecture}.


The $V_{incomp}$ tag for each client-library pair also specifies the
specific test that can reveal the incompatibility issue during its
run. Following Maven's convention, the test name is formatted as
\Code{TestClassName\#testMethodName}. By simply copying the text from
the tag, users can easily run the incompatibility-revealing test on
the $V_{incomp}$ version and observe the incompatibility issue. On the
$V_{base}$ version, all tests should pass. This design aims to
simplify the usage of \toolcompsuite and make it more accessible and
user-friendly.


Using the forked client repositories and version tags provided in the
\toolcompsuite dataset, users can easily reproduce any incompatibility issue by
checking out to $V_{incomp}$ and running the corresponding
incompatibility-revealing test. To compare the behaviors of the client
with compatible and incompatible library versions, users can run the
incompatibility-revealing test on both $V_{base}$ and $V_{incomp}$ and
compare the test outcomes. This allows for a clear understanding of
the impact of the library upgrade on the client behaviors.


\subsection{\runner: An Automated Tool for Reproducing Incompatibility Issues}

We further developed an automated tool, named \runner, which is a part of
\toolcompsuite. With \runner, users can easily reproduce and investigate any
incompatibility issue in a one-click manner by providing the issue ID
as input.

We offer an option which enables users to reproduce an
incompatibility issue end-to-end with a single command as is shown below.
The command outputs and saves all intermediate results and logs for future
reference.

\begin{Verbatim}[fontsize=\small,breaklines,linenos,xleftmargin=8pt,numbersep=5pt,commandchars=\\\{\}]
python\PYG{+w}{ }main.py\PYG{+w}{ }\PYGZhy{}\PYGZhy{}incompat\PYG{+w}{ }i\PYGZhy{}56
\end{Verbatim}

When \runner runs, it clones the client project from our forked
code repository and saves it in the output directory (which is configurable).
Then, it checks out to the base version, compiles the code, and runs the
incompatibility-revealing test. Next, it upgrades the library to the
new version, reruns the incompatibility-revealing test, and reports
any failure information to the user.

We also provide a set of commands that break down the entire cycle of incompatibility
exploration into separate steps:

\begin{Verbatim}[fontsize=\small,breaklines,linenos,xleftmargin=8pt,numbersep=5pt,commandchars=\\\{\}]
python main.py \PYGZhy{}\PYGZhy{}download i\PYGZhy{}56
python main.py \PYGZhy{}\PYGZhy{}compile i\PYGZhy{}56
python main.py \PYGZhy{}\PYGZhy{}testold i\PYGZhy{}56
python main.py \PYGZhy{}\PYGZhy{}testnew i\PYGZhy{}56
\end{Verbatim}


We provide several other \runner commands for users to inspect
different aspects of the incompatibility issues from the \toolcompsuite dataset. A
complete list of these commands can be found on \toolcompsuite's website at
\url{https://github.com/compsuite-team/compsuite}.


