\section{Evaluation}\label{sec:eval}

To evaluate the effectiveness and practicality of our approach at repository-level, we design the following research questions:
\begin{enumerate}
	\item \textbf{RQ1:} How effective is \tooldataloc in keyword-agnostic logical code localization?
	\item \textbf{RQ2:} How effective is \tooldataloc for issue-based code localization?
	\item \textbf{RQ3:} How efficient is \tooldataloc compared to baselines?
    \item \textbf{RQ4:} How does each component of \tooldataloc contribute to its performance?
\end{enumerate}


\subsection{Experiment Setup}

\subsubsection{Benchmarks.} 
We evaluate code localization performance on three Python-based benchmarks, covering both complex logical reasoning challenges and industrial issue-resolution tasks.

\textbf{SWE-bench Lite}~\cite{swebench2024}. A carefully curated and widely recognized subset from the full SWE-bench for more efficient and cost-effective evaluation of autonomous issue-solving capabilities. It consists of real-world GitHub issues with repository metadata and ground-truth patch locations. Following Suresh et al.~\cite{suresh2024cornstack}, we retained 274 of 300 original instances where patches modify existing functions or classes. We intentionally excluded instances introducing code corresponding to new functions or import statements to focus the evaluation on code localization within existing structures.

\textbf{\dataset} (Ours). To evaluate the capability of localization approaches in keyword-agnostic logical code localization, we constructed \dataset, a diagnostic benchmark comprising 25 high-quality logic-intensive queries. As illustrated in Table~\ref{tab:code_dimensions}, each query is formulated as a composite logical proposition by integrating code features across multiple dimensions. By combining structural granularity (e.g., classes, methods) with behavioral attributes (e.g., control flow, exception handling) and code metrics (e.g., inheritance depth and branch count), \dataset captures complex patterns that demand deep repository understanding.

For each query, we utilize the environment (repository and base commit version) from the first case of SWE-bench Lite as the foundation. For each query, we executed searches using \texttt{Cursor} and \texttt{GitHub Copilot} in agent mode with multiple latest advanced models like Claude-4.5-Opus and GPT-5.2, and manually validated all returned results to establish the ground-truth locations.

\dataset serves as a critical complement to issue-based benchmarks for localization task. In practice, issues are one of the most important channels for error feedback between users and maintainers. To facilitate debugging, those issue descriptions often provide sufficient information and clear keywords as cues to help maintainers better locate faults, such as accurate file paths, function identifiers, or even specific code snippets. Our analysis of SWE-bench Lite instances reveals that over 50\% of ground-truth locations are mentioned in the issue descriptions. Such \textit{keyword shortcut} enables models to succeed via simple lexical matching (e.g. grep) or embedding-based retrieval, without requiring genuine understanding and reasoning over the codebase. This undermines the validity of localization performance evaluations. Moreover, LLM-assisted development shifts the codebase interaction toward intent-based question answering, allowing developers to query repositories using natural language. However, for developers unfamiliar with a given repository, they typically cannot use precise identifiers and instead tend to express their search intent through high-level behavioral pattern descriptions or abstract logical structures. 


\textbf{\negset} (Ours). \negset is a variant of \dataset where queries are intentionally modified to ensure their ground-truth sets are empty. Current methods often adopt top-$n$ ranking to maximize recall, but ideal robust localization requires the ability to provide ascertained answers and avoid false positives. \negset evaluates the abstention capability when no valid location meets the query. Such ``refusal'' mechanism is a critical metric for ensuring the reliability of autonomous agents in production environments.


\begin{table}[htbp]
\centering
\caption{Taxonomy of Python Code Dimensions and Representative Elements}
\label{tab:code_dimensions}
\small
\renewcommand{\arraystretch}{1.2}
\begin{tabularx}{\textwidth}{lX}
\toprule
\textbf{Query Dimensions} & \textbf{Examples / Typical Elements} \\ \midrule
Code Structure & Functions, Methods, Classes, Modules, Decorators \\
Control Flow & Conditional (\texttt{if-elif-else}), Iteration (\texttt{for}, \texttt{while}), Context Management (\texttt{with}) \\
Condition Logic & Comparison (\texttt{==}, \texttt{>}), Identity (\texttt{is}), Membership (\texttt{in}), Type Checks (\texttt{isinstance}), Logical Operators (\texttt{and}, \texttt{or}, \texttt{not}), Early Exit (\texttt{return}, \texttt{break}, \texttt{continue}) \\
Data Structure & Built-in Collections (\texttt{list}, \texttt{dict}, \texttt{set}), Primitive Types (\texttt{int}, \texttt{str}) \\
Function Signatures & Default Values, Variadic Parameters (\texttt{*args}, \texttt{**kwargs}), Type Annotations \\
Exception Handling & Exception Propagation (\texttt{try-except-finally}), Exceptions (\texttt{TypeError}) \\
Code Metrics & Nesting Depth, Inheritance Depth, Assertion Count, Branch count \\ \bottomrule
\end{tabularx}
\end{table}

% To assess the effectiveness of our framework, we require datasets containing ground-truth annotations for code localization that can serve as reliable and standardized benchmarks. SWE-bench~\cite{jimenez2024swebench} constitutes a widely adopted benchmark for evaluating the capabilities of AI systems in performing end-to-end bug fixing across repository-level codebases. Each instance consists of a GitHub issue paired with its corresponding code patches. SWE-bench Lite~\cite{swebench2024} provides a carefully curated subset of 300 tasks from the full benchmark, designed to reduce evaluation costs while preserving the benchmark’s representativeness and overall quality. Following the approach of Suresh et al.~\cite{suresh2024cornstack}, we retained 274 of the 300 instances where patches modify existing functions or classes, excluding instances that introduce new functions or import statements. As code localization constitutes a critical yet implicit intermediate step in bug fixing process, we use the modified code locations from the patches as ground truth to evaluate localization performance.

% Similar to Swe-Bench, LocBench~\cite{chen-etal-2025-locagent}, proposed by Chen et al., is a dataset specifically designed for code localization, comprising 560 issues from Python repositories. Collected after October 2024 to mitigate data leakage and pre-training bias in recent LLMs, it encompasses a broad range of issue categories beyond bug fixing. After removing inaccessible repositories, we retain xxx of the original 560 examples. In summary, for Python, we adopt both SWE-bench Lite and LocBench.

% Our framework supports not only Python but also Java. For Java evaluation, we extract relevant instances from two multilingual datasets: SWE-bench Multilingual~\cite{yang2025swesmith}, which provides 300 tasks across 42 repositories and 9 programming languages including Java, and Multi-SWE-bench~\cite{zan2025multiswebench} from ByteDance, spanning 7 languages with 1,632 high-quality instances. After extracting Java-related entries and deduplicating across both sources, the combined dataset contains XXX instances. We manually annotate code locations as ground truth based on the \texttt{git diff} information, forming a dataset we refer to as SWE-bench Java.

% \todo{average file change, class change, function change, etc.}

\subsubsection{Baselines.} To assess \tooldataloc, we select four state-of-the-art baselines representing three distinct technical paradigms: embedding-based, pipeline-based, and agent-based approaches:
\begin{enumerate}[leftmargin=*]
    \item \textbf{SweRank}~\cite{reddy2025swerank} (\textit{Embedding-based}): It utilizes a retrieve-and-rerank architecture to identify issue locations. It employs SWERankEmbed (137M/7B parameters) to perform initial retrieval and SWERankLLM (7B/32B parameters) to rerank the results.
    \item \textbf{Agentless}~\cite{xia2024agentless} (\textit{Pipeline-based}): This approaches employs a hierarchical filtering strategy within a procedural workflow. It progressively prunes the search space from the file level down to specific classes or functions, utilizing an LLM to rank and select candidates at each stage.
    \item \textbf{LocAgent}~\cite{chen-etal-2025-locagent} (\textit{Agent-based}): It constructs a graph-based representation and sparse indexes of the project and enable an autonomous agent to perform iterative, tool-assisted retrieval.
    \item \textbf{CoSIL}~\cite{jiang2025cosil} (\textit{Agent-based}): This framework focuses on structural dependency traversal through call graphs to identity implicit locations via iterative exploration. It incorporates pruning to maintain context efficiency and restrict the search to high-relevance execution paths.
    % \item \textbf{Orca Loca~\cite{yu2025orcalocallmagentframework}:} Integrates priority-based scheduling, action decomposition with relevance scoring, and distance-aware context pruning. By optimizing the synergy between agentic reasoning and precise retrieval, it effectively navigates complex repositories to resolve the suboptimality of current search mechanisms.
\end{enumerate}

\subsection{Metrics}
We evaluate localization performance at three granularity: \textit{file, module, and function}. Let $Q$ denote the set of query instances, $G_q$ the set of ground-truth locations for query $q \in Q$, and $\mathcal{P}_q$ the set of predicted locations inferred by our agent workflow. $\mathbf{1}(\cdot)$ denotes the \textbf{indicator function}, which equals 1 if the logical condition holds and 0 otherwise. We adopt the following six metrics:

\begin{enumerate}[leftmargin=*, label=\textbf{M\arabic*.}]
    \item \textbf{Accuracy@k (ACC@k):} It measures the ability to achieve full coverage, where a success requires all ground-truth locations to be present within the top-$k$ predicted locations. When $k$ equals the length of the prediction set, this metric becomes the \textbf{Success Rate (SR)}:
    \begin{equation}
        Acc@k = \frac{1}{|Q|} \sum_{q \in Q} \mathbf{1}(G_q \subseteq \mathcal{P}_{q,k})
    \end{equation}

    \item \textbf{Recall (REC):} It represents the proportion of ground-truth locations successfully captured by the predicted set $\mathcal{P}_q$:
    \begin{equation}
        Rec@k = \frac{1}{|Q|} \sum_{q \in Q} \frac{|G_q \cap \mathcal{P}_{q,k}|}{|G_q|}
    \end{equation}

    \item \textbf{Precision (PRE):} This metric penalizes overprediction by calculating the fraction of predicted locations that are correct:
    \begin{equation}
        Pre = \frac{1}{|Q|} \sum_{q \in Q} \frac{|G_q \cap \mathcal{P}_q|}{|\mathcal{P}_q|}
    \end{equation}

    \item \textbf{Average Jaccard Similarity (AJS):} It quantifies the overlap between the predicted and ground-truth sets, which penalizes both missing targets and redundant predictions:
    \begin{equation}
        AJS = \frac{1}{|Q|} \sum_{q \in Q} \frac{|G_q \cap \mathcal{P}_q|}{|G_q \cup \mathcal{P}_q|}
    \end{equation}

    \item \textbf{Perfect Location Rate (PLR):} The most Stringent metric, measuring the ratio of instances where the predicted set $\mathcal{P}_q$ exactly matches the ground-truth set $G_q$. A PLR of 1.0 indicates perfect localization without any extraneous noise (i.e., $AJS = 1.0$):
    \begin{equation}
        PLR = \frac{1}{|Q|} \sum_{q \in Q} \mathbf{1}(\mathcal{P}_q = G_q)
    \end{equation}

    \item \textbf{Hit Rate (HR):} The most lenient metric, measuring the ratio of instances where the predicted set $\mathcal{P}_q$ provides at least one correct location:
    \begin{equation}
        HR = \frac{1}{|Q|} \sum_{q \in Q} \mathbf{1}(\mathcal{P}_{q} \cap G_q \neq \emptyset)
    \end{equation}
\end{enumerate}




\subsubsection{Implementation and environment.} All experiments were conducted on a server equipped with an Intel Xeon Silver 4216 CPU (2.10 GHz) and 62 GB RAM, running Ubuntu 22.04.5 LTS. Our framework was implemented using Python 3.12.11 and the Soufflé 2.4 Datalog engine. To evaluate \tooldataloc, we accessed \texttt{gpt-4o-20240513} via OpenAI’s API, \texttt{claude-3-5-sonnet-20241022} through AWS Bedrock services, \texttt{Deepseek-reasoner} via DeepSeek' API, and \texttt{Qwen3-Max} via Alibaba Cloud Service. For baseline comparisons, we instantiated runtime environments according to their respective official specifications and dependency requirements to ensure a fair evaluation.


\subsection{Results}

\subsubsection{Effectiveness for logic query}

As summarized in Table~\ref{tab:comparison_lq}, \tooldataloc achieves a decisive lead over all baselines across all metrics and granularities. At the file level, \tooldataloc reaches a Precision of 65\% and a Success Rate of 64\%, which is significantly higher than the baseline methods. Additionally, while all baselines fail completely to achieve perfect location (0\% PLR), \tooldataloc attains a PLR of 44\%, indicating the unique advantage of our framework's in capturing code structure and reasoning capabilities in assisting precise localization. To evaluate the generality of our framework, we applied it to Qwen3-Max. The framework achieved strong performance, even surpassing Claude-3.5 on some metrics, suggesting that it generalizes well across different models.

Even with the most lenient metric, Hit Rate (HR), which only requires at least one correct location, baseline performance drops sharply as the granularity shifts from file-level to module-level and function-level. Other metrics even approach zero. It indicates that, when deprived of explicit keywords and forced into deep tracing, they tend to resort to near-random guessing rather than structural reasoning. In contrast, \tooldataloc demonstrates strong stability, achieving a high HR of around 80\%. This robustness proves that \tooldataloc's success is not a byproduct of a coarse search space but is driven by rigorous, logic-based reasoning.

Baselines typically rely on top-$n$ recommendations to increase the probability of covering relevant locations, but this strategy is inherently a compromise rather than an optimal solution. An effective code localization tool should return results that precisely satisfy the query constraints, since the true number of relevant locations varies across tasks and is not predetermined. To evaluate this capability, we introduce two additional metrics: Average Jaccard Similarity (AJS) and Perfect Localization Rate (PLR). AJS penalizes both false positives and false negatives, while PLR represents the most stringent criterion, requiring the predicted set to exactly match the ground truth (i.e., achieving 100\% AJS). For instance, while LocAgent (Claude-3.5) achieves a 44\% hit rate at the file level, its AJS is only 1.57\%, indicating that true positives are diluted within an inflated candidate set containing substantial noise. By comparison, our approach consistently maintains high AJS scores, reflecting greater precision in returning constraint-satisfying results without extraneous recommendations. This precision is important for industrial deployment, as it reduces the validation overhead for developers or downstream automated agents, improving the efficiency of maintenance workflows.

Our investigation of SWE-bench Lite shows that most issues are highly localized, involving an average of only 1.15 code changes. This sparsity raises a key question: do existing tools truly pinpoint root causes, or do they merely rely on high-probability guessing within a narrow search space? To examine this, we use \negset to evaluate whether tools can recognize when no valid location exists. By modifying constraints, we deliberately created a mismatch between the issue description and the codebase, such that the original ground-truth locations are no longer valid. In this setting, the only correct output is a clear ``no match found''. Unfortunately, all SOTA baselines suffer from a compulsion to guess. They persistently return top-$n$ recommendations even when query prerequisites are not met. This over-eager behavior proves harmful in practice, as confident yet wrong targets mislead downstream agents, wasting computational resources, and risk introducing regression bugs. These findings suggest that the strong performance reported by existing baselines is partially inflated by their recommendation-centric design, which lacks true localization rationale. Notably, \tooldataloc demonstrates the necessary discernment to abstain when no valid location exists, returning a clear “no match found” response for over 70\% of the queries.
\begin{landscape}
\begin{table*}[t]
    \centering
    \scriptsize 
    \setlength{\tabcolsep}{1.2pt} 
    \caption{Evaluation results on LogicQuery}
    \label{tab:comparison_lq}
    
    \begin{tabularx}{\linewidth}{@{} l l *{18}{C} @{}} 
        \toprule
        \multirow{2}{*}{\textbf{Methods}} & \multirow{2}{*}{\textbf{LLM}} & \multicolumn{6}{c}{\textbf{File Level (\%)}} & \multicolumn{6}{c}{\textbf{Module Level (\%)}} & \multicolumn{6}{c}{\textbf{Function Level (\%)}} \\
        \cmidrule(lr){3-8} \cmidrule(lr){9-14} \cmidrule(lr){15-20}
        & & SR & REC & PRE & AJS & PLR & HR & SR & REC & PRE & AJS & PLR & HR & SR & REC & PRE & AJS & PLR & HR \\
        
        \midrule
        \multirow{2}{*}{SweRank} 
        & Small & 4.00 & 8.00 & 3.30 & 2.66 & 0 & 20.00 & 0 & 0 & 0 & 0 & 0 & 0 & 0 & 0 & 0 & 0 & 0 & 0 \\
        & Large & 0 & 8.13 & 4.67 & 3.47 & 0 & 20.00 & 0 & 6.04 & 2.92 & 2.18 & 0 & 16.67 & 0 & 4.76 & 2.38 & 1.87 & 0 & 14.29 \\

        \midrule
        \multirow{2}{*}{Agentless} 
        & GPT-4o & 0 & 0 & 0.80 & 0.50 & 0 & 4.00 & 0 & 0 & 0.35 & 0.28 & 0 & 4.17 & 0 & 0 & 0 & 0 & 0 & 0 \\
        & Claude-3.5 & 0 & 2.00 & 1.00 & 0.80 & 0 & 4.00 & 0 & 2.08 & 0.60 & 0.52 & 0 & 4.17 & 0 & 0 & 0 & 0 & 0 & 0 \\

        \midrule
        \multirow{2}{*}{LocAgent} 
        & GPT-4o & 12.00 & 2.00 & 1.37 & 1.12 & 0 & 20.00 & 4.17 & 0 & 0.85 & 0.62 & 0 & 12.50 & 0 & 0 & 0 & 0 & 0 & 0 \\
        & Claude-3.5 & 12.00 & 3.33 & 1.85 & 1.78 & 0 & 44.00 & 8.33 & 2.08 & 1.20 & 1.16 & 0 & 25.00 & 4.76 & 2.38 & 0.65 & 0.62 & 0 & 19.05 \\

        \midrule
        \multirow{2}{*}{CoSIL} 
        & GPT-4o & 0 & 1.00 & 1.00 & 0.57 & 0 & 4.00 & 0 & 0 & 0 & 0 & 0 & 0 & 0 & 0 & 0 & 0 & 0 & 0 \\
        & Claude-3.5 & 4.00 & 9.00 & 4.13 & 3.24 & 0 & 16.00 & 4.17 & 6.25 & 2.22 & 1.88 & 0 & 8.33 & 4.76 & 7.14 & 1.90 & 1.75 & 0 & 9.52 \\

        \midrule
        \multirow{2}{*}{\makecell{\tooldataloc}} 
        & Claude-3.5 & \textbf{64.00} & \textbf{61.00} & 62.45 & \textbf{57.05} & \textbf{44.00} & \textbf{80.00} & \textbf{62.50} & \textbf{60.42} & \textbf{60.85} & 55.12 & \textbf{41.67} & \textbf{79.17} & \textbf{61.90} & \textbf{62.70} & \textbf{63.63} & 57.09 & \textbf{42.86} & \textbf{80.95} \\
        & Qwen3-Max & \textbf{64.00} & 55.00 & \textbf{65.00} & 56.40 & \textbf{44.00} & 68.00 & 60.00 & 50.33 & 60.00 & \textbf{56.07} & 40.00 & 64.00 & 56.00 & 45.53 & 55.20 & \textbf{58.73} & 40.00 & 60.00 \\
        \bottomrule
    \end{tabularx}
\end{table*}
\end{landscape}

\smallskip
\noindent\shadowbox{%
  \begin{minipage}{0.98\columnwidth}
    \textbf{Answer to RQ1:}
		The results clearly show that \tooldataloc effectively handles the keyword-agnostic logical code localization challenge, whereas baselines perform poorly when keyword shorts are unavailable. This means these approaches still rely on shallow lexical matching rather than genuine logical reasoning. Furthermore, their performance on the \negset exposes fundamental weakness in their refusal ability.
  \end{minipage}}
%%% Local Variables:
%%% mode: latex
%%% TeX-master: "../main"
%%% End:

\subsubsection{Effectiveness for general issues}

To evaluate the practical utility of \tooldataloc in real-world software maintenance, we conducted a comparative analysis on the SWE-bench Lite. Existing approaches usually employ top-$n$ strategy, whereas \tooldataloc operates without a predefined $n$. Table~\ref{tab:comparison_isq} illustrates that \tooldataloc remains highly competitive.

A critical distinction lies in the recommendation density and target precision. While \tooldataloc provides instance-specific localization with an average of only 2 candidates per issue, SOTA baselines rely on much broader and often fixed-size candidate sets to improve their accuracy. Specifically, CoSIL and LocAgent typically default to a top-$5$ recommendation at the function level, while Agentless routinely recommends 5 locations as candidates regardless of the issue's actual complexity. SweRank adopts an even more aggressive strategy, utilizing top-100 rankings.

Consequently, even when Acc@$n$ metrics appear comparable, \tooldataloc achieves substantially higher precision. By providing a concise and accurate set of entry points, \tooldataloc minimizes the noise that developers or downstream agents must filter, reducing validation overhead. This precision-centric design also yields substantial gains in resource efficiency (details in Section~\ref{sec:cost}). 

% Compared to agent-based baselines that require extensive multi-turn interactions, \tooldataloc operates with remarkably low latency and token consumption by focusing only on logically necessary code segments. We also observed that this efficiency contrasts with the operational instability of tools like LocAgent, which frequently falls into infinite execution loops on complex or mutated instances. While we will provide a comprehensive breakdown of these overhead metrics in RQ3, these preliminary results highlight that \tooldataloc delivers a superior balance of competitive recall, surgical precision, and minimal resource expenditure.


\begin{table}[t]
\centering
\caption{Comparison of different methods and models across various localization granularities.}
\label{tab:comparison_isq}
\small 
\begin{tabularx}{\textwidth}{@{} ll CCCCC CC @{}}
\toprule
\multirow{2}{*}{\textbf{Method}} & \multirow{2}{*}{\textbf{Model}} & \multicolumn{3}{c}{\textbf{File (\%)}} & \multicolumn{2}{c}{\textbf{Module (\%)}} & \multicolumn{2}{c}{\textbf{Function (\%)}} \\ 
\cmidrule(lr){3-5} \cmidrule(lr){6-7} \cmidrule(lr){8-9}
& & Acc@1 & Acc@3 & Acc@5 & Acc@5 & Acc@10 & Acc@5 & Acc@10 \\ 
\midrule
\multirow{2}{*}{SweRank}   & Small           & 78.10 & 92.34 & 94.53 & 89.05 & 92.70 & 79.56 & 86.13 \\
                           & Large           & 83.21 & 94.89 & 95.99 & 90.88 & 93.43 & 81.39 & 88.69 \\ 
\midrule
% \addlinespace[0.5em] 
\multirow{2}{*}{Agentless} & gpt-4o          & 67.50 & 74.45 & 74.45 & 67.15 & 67.15 & 55.47 & 55.47 \\
                           & claude-3.5      & 72.63 & 79.20 & 79.56 & 68.98 & 68.98 & 58.76 & 58.76 \\ 
\midrule
% \addlinespace[0.5em]
\multirow{2}{*}{LocAgent}  & Qwen2.5-32B(ft) & 75.91 & 90.51 & 92.70 & 85.77 & 87.23 & 71.90 & 77.01 \\
                           & claude-3.5      & 77.74 & 91.97 & 94.16 & 86.50 & 87.59 & 73.36 & 77.37 \\ 
\midrule
% \addlinespace[0.5em]
\multirow{2}{*}{DataLoc}   & gpt-5.1         & 71.53 & 77.38 & 78.47 & 70.80 & 72.26 & 63.14 & 64.96 \\
                           & claude-3.5      & 72.26 & 80.66 & 81.02 & 75.55 & 75.55 & 68.98 & 68.98 \\ 
\bottomrule
\end{tabularx}
\end{table}

\smallskip
\noindent\shadowbox{%
  \begin{minipage}{0.98\columnwidth}
    \textbf{Answer to RQ2:}
		\tooldataloc also demonstrates competitive performance on issue-solving benchmarks. Notably, given that \tooldataloc produces only 2 candidate locations on average, it offers distinct advantages in recommendation efficiency and in excluding incorrect locations. This precise localization helps reduce the potential overhead of downstream tasks.
  \end{minipage}}
\subsubsection{Efficiency}\label{sec:cost-dataloc}
Figure~\ref{fig:cost_analysis} presents a comprehensive comparison using a lollipop chart, where the vertical axis represents average execution time (seconds) and bubble size reflects total token consumption (labeled in thousands). As illustrated, \tooldataloc (Claude3.5) establishes a new efficiency frontier for KA-LCL tasks, achieving the optimal balance between speed and token consumption with 37 seconds execution time and a 13.5k token consumption. Compared to other approaches, \tooldataloc exhibits a clear dual advantage in both temporal efficiency and token economy:

\textbf{Temporal efficiency.} \tooldataloc (Claude-3.5) completes localization in around half minute per task, outperforming all agentic baselines. Even the fastest CoSIL (GPT-4o) remain over 3$\times$ slower. The poor performance in Table~\ref{tab:comparison_lq} further indicate that, without keyword shortcuts, baseline methods fail to perform meaningful repo-level inference, despite exhibiting imtermediate reasoning steps.

\textbf{Token economy.} Approaches that rely more on agents incur substantially higher token consumption, with LocAgent and Agentless consuming over an order of magnitude more tokens per task than \tooldataloc. This token explosion reflects a trade-off where tokens are exchanged for intelligence, but this intelligence is currently tie to text. As a result, without keyword shortcuts to guide retrieval, these agents are trapped in multi-round conversation and iterative codebase exploration. In our evaluation, three queries causes LocAgent to enter infinite loops without producing results.

The efficiency of \tooldataloc stems from its hybrid architecture. In our design, the LLM-based agent is strategically confined to high-level tasks: query analysis, Datalog program synthesis, and final candidate verification. By shifting deep inference to a specialized engine, \tooldataloc greatly reduces the overhead of redundant multi-round exploration. In addition, errors carry small cost, requiring only the regeneration of a Datalog program. Beyond this low overhead, our parser-gated validation and intermediate-rule feedback actively assist LLM to synthesize high-quality programs.



\begin{figure}[t]
	\centering
	\includegraphics[width=\textwidth]{Figures/Chapter5/cost.png}
	\caption{Average execution time and token consumption of \dataset}
	\label{fig:cost_analysis}
\end{figure}

\smallskip
\noindent\shadowbox{%
  \begin{minipage}{0.98\columnwidth}
    \textbf{Answer to RQ3:}
		\tooldataloc defines the efficiency frontier in the KA-LCL challenge, maintaining an average execution time of 37s and a token consumption of 13.5k. By replacing expensive LLM-based exploration with Datalog-driven inference, \tooldataloc bypasses the prohibitive costs and execution loops of existing agentic baselines, demonstrating its potential industrial deployment.
  \end{minipage}}
\subsubsection{Ablation Study}\label{sec:ablation}
We design an ablation study to quantify how two mechanisms detailed in~\cref{sec:method:val} and \cref{sec:method:int} improve the quality of LLM-generated Datalog program.

%where ``quality'' encompasses (i) executability (passing parsing and evaluation), (ii) result usefulness (avoiding silent empty outputs caused by mistakes), and (iii) convergence efficiency (reaching a working query with fewer LLM iterations).

%For brevity, we refer to these two types of mechanism as \emph{validation and repair} (VAL) and \emph{intermediate-rule feedback} (INT), respectively.

\textbf{Configurations}
We evaluate three configurations that progressively enable these mechanisms:
\textbf{Base} directly executes the LLM-generated Datalog program without any validation or additional feedback, except the output or error messages from \souffle{} itself. \textbf{VAL} (validation and repair) enables parser-gated validation with deterministic syntactic repairs and high-confidence semantic checks. \textbf{Full} further enables intermediate-rule mutation and feedback.
%
All configurations share the same LLM, initial prompt, iteration budget, and underlying fact bases.
%Each natural-language query is evaluated independently.

\textbf{Metrics}
We compare  (i) execution success rate and non-empty result rate, (ii) mean LLM iterations to first successful execution and (iii) final answer correctness on \dataset.

% \usepackage{siunitx}
\begin{table}[t]
  \centering
  \small
  \setlength{\tabcolsep}{4.0pt}
  \renewcommand{\arraystretch}{1.15}
  \caption{Ablation results under two LLMs. \textbf{Base}: no mechanism;
  \textbf{VAL}: validation \& repair; \textbf{Full}: VAL+intermediate-rule feedback (INT).
  Rates are reported in \%. Iteration and time metrics report mean values (lower is better). Time is reported in seconds.}
  \label{tab:ablation}

  \resizebox{\columnwidth}{!}{%
  \begin{tabular}{ll
                  cc
                  cc c
                  cccccc}
    \toprule
    \multirow{2}{*}{\textbf{Model}} &
    \multirow{2}{*}{\textbf{Config}} &
    \multicolumn{2}{c}{\textbf{Rates} $\uparrow$} &
    \multicolumn{3}{c}{\textbf{Cost (Iter, Time)} $\downarrow$} &
    \multicolumn{6}{c}{\textbf{Correctness (function-level)} $\uparrow$} \\
    \cmidrule(lr){3-4}
    \cmidrule(lr){5-7}
    \cmidrule(lr){8-13}
    & & \textbf{ExecSucc} & \textbf{$\ne\varnothing$} &
    \textbf{First} & \textbf{Iter} & \textbf{Time} &
    \textbf{SR} & \textbf{PRE} & \textbf{REC} & \textbf{AJS} & \textbf{PLR} & \textbf{HR} \\
    \midrule

    \multirow{3}{*}{Qwen3-Max} &
    Base   & 73.59 & 49.41 & 1.48 & 12.92 & 427 & 20.00 & 18.20 & 15.87 & 26.73 & 16.00 & 20.00 \\
    & VAL  & 84.12 & 75.68 & 1.28 & 10.16 & 104 & 40.00 & 38.93 & 31.87 & 43.67 & 32.00 & 44.00 \\
    & Full & 84.12 & 78.52 & 1.24 & 10.92 & 136 & 56.00 & 55.20 & 45.53 & 58.73 & 40.00 & 60.00 \\
    \midrule

    \multirow{3}{*}{Claude-3.5} &
    Base   & 89.42 & 67.95 & 1.16 & 4.20 & 38 & 44.00 & 38.10 & 33.87 & 43.70 & 24.00 & 52.00 \\
    & VAL  & 94.67 & 75.86 & 1.04 & 4.28 & 40 & 44.00 & 38.90 & 31.87 & 48.36 & 32.00 & 52.00 \\
    & Full & 94.67 & 80.69 & 1.08 & 4.08 & 37 & 61.90 & 62.70 & 63.63 & 57.09 & 42.86 & 80.95 \\
    \bottomrule
  \end{tabular}}

  \vspace{1mm}
  \footnotesize\raggedright
  \textbf{Cost metrics.}
  \textbf{$\ne\varnothing$} denotes the fraction of queries that produce non-empty answers.
  \textbf{First} is the mean number of LLM iterations to the first successfully executing program.
  \textbf{Iter} is the average total number of LLM iterations consumed per query.
  \textbf{Time} reports the mean end-to-end wall-clock time per query, including failed attempts and tool feedback.
\end{table}

\ref{tab:ablation} summarizes the ablation results under two different LLMs, comparing the baseline system with progressively enabled mechanisms.
The results show that validation and repair (VAL) substantially improves the quality of LLM-generated Datalog for both models, with a markedly stronger effect on Qwen3 Max than on Claude 3.5 Sonnet. This difference is expected given the models' baseline capabilities: without VAL, Qwen3 Max exhibits a significantly lower execution success rate due to its weaker ability to consistently produce syntactically correct \souffle{} Datalog, whereas Claude already achieves a relatively high baseline level of syntactic validity.

For Qwen3 Max, enabling VAL leads to a dramatic improvement across nearly all metrics. The non-empty result rate increases from 49\% to 75\%, indicating that a large fraction of previously failing or unproductive queries were recoverable once parser-gated validation and conservative repairs were applied. At the same time, the average end-to-end execution time drops sharply from 427 seconds to 104 seconds, reflecting a reduction in wasted iterations caused by unrecoverable parser errors and repeated failed executions. Improvements are also reflected in downstream task quality: function-level correctness metrics show substantial gains, with precision increasing from 18\% to 50\%, demonstrating that VAL does not merely enable execution but also materially improves the semantic adequacy of the resulting queries.
In contrast, the effect of VAL on Claude is more moderate: execution success rate increases by +5 percentage points, and the non-empty result rate increases by +8 percentage points. 

Enabling intermediate-rule feedback (INT) in the Full configuration produces a different pattern. The primary role of INT is to guide the LLM toward identifying which specific rule is semantically invalid in the sense of producing no results, and how that rule can be locally revised. As a result, INT may slightly increase iteration count or execution time in some cases due to additional diagnostic executions; however, this overhead consistently translates into higher non-empty rates and improved final answer correctness.

\smallskip
\noindent\shadowbox{%
  \begin{minipage}{0.98\columnwidth}
    \textbf{Answer to RQ4:} Validation and repair (VAL) substantially improves the robustness of
    LLM-generated Datalog, with particularly large gains for weaker models by increasing non-empty
    result rates and reducing execution cost. Intermediate-rule feedback (INT) complements VAL by
    guiding targeted revisions of logically unproductive rules, occasionally incurring additional
    diagnostic cost but improving final answer correctness across models.
  \end{minipage}}
%%% Local Variables:
%%% mode: latex
%%% TeX-master: "../main"
%%% End:


%%% Local Variables:
%%% mode: latex
%%% TeX-master: "../main"
%%% End:
