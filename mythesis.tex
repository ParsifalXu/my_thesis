%MIT License
%
%Copyright (c) 2018 Chen Wang [https://chenwang.site]
%
%Permission is hereby granted, free of charge, to any person obtaining a copy
%of this software and associated documentation files (the "Software"), to deal
%in the Software without restriction, including without limitation the rights
%to use, copy, modify, merge, publish, distribute, sublicense, and/or sell
%copies of the Software, and to permit persons to whom the Software is
%furnished to do so, subject to the following conditions:
%
%The above copyright notice and this permission notice shall be included in all
%copies or substantial portions of the Software.
%
%THE SOFTWARE IS PROVIDED "AS IS", WITHOUT WARRANTY OF ANY KIND, EXPRESS OR
%IMPLIED, INCLUDING BUT NOT LIMITED TO THE WARRANTIES OF MERCHANTABILITY,
%FITNESS FOR A PARTICULAR PURPOSE AND NONINFRINGEMENT. IN NO EVENT SHALL THE
%AUTHORS OR COPYRIGHT HOLDERS BE LIABLE FOR ANY CLAIM, DAMAGES OR OTHER
%LIABILITY, WHETHER IN AN ACTION OF CONTRACT, TORT OR OTHERWISE, ARISING FROM,
%OUT OF OR IN CONNECTION WITH THE SOFTWARE OR THE USE OR OTHER DEALINGS IN THE
%SOFTWARE.
\PassOptionsToPackage{dvipsnames}{xcolor}
\documentclass[12pt,a4paper]{Thesis} % Paper size, default font size and one-sided paper

\graphicspath{%
	{./Pictures/}%
	{./Figures/}%
}
\DeclareMathOperator{\Tr}{Tr}
\let\savedegree\degree
\let\degree\relax
\let\savedegree\ref
\let\rem\relax
\usepackage{adjustbox}
\usepackage{amsmath}
% \usepackage[amssymb]{SIunits}
\usepackage{siunitx}
\usepackage{amssymb}
\usepackage{multirow}
\usepackage{wrapfig}
\usepackage{enumitem}
\usepackage{subcaption}
\usepackage{mathtools}
\usepackage{lipsum}
\usepackage{pifont}
\usepackage{tabularx}
\usepackage{makecell}
\usepackage{pdflscape}
\usepackage{xcolor}
\usepackage[flushleft]{threeparttable}
% \usepackage{tcolorbox}
% \usepackage[ruled,vlined,noresetcount]{algorithm2e}
%\usepackage{algpseudocode}
\usepackage[square, numbers, comma, sort&compress]{natbib} % Use the natbib reference package - read up on this to edit the reference style; if you want text (e.g. Smith et al., 2012) for the in-text references (instead of numbers), remove 'numbers'


\makeatletter
\def\PYG@reset{\let\PYG@it=\relax \let\PYG@bf=\relax%
    \let\PYG@ul=\relax \let\PYG@tc=\relax%
    \let\PYG@bc=\relax \let\PYG@ff=\relax}
\def\PYG@tok#1{\csname PYG@tok@#1\endcsname}
\def\PYG@toks#1+{\ifx\relax#1\empty\else%
    \PYG@tok{#1}\expandafter\PYG@toks\fi}
\def\PYG@do#1{\PYG@bc{\PYG@tc{\PYG@ul{%
    \PYG@it{\PYG@bf{\PYG@ff{#1}}}}}}}
\def\PYG#1#2{\PYG@reset\PYG@toks#1+\relax+\PYG@do{#2}}

\@namedef{PYG@tok@w}{\def\PYG@tc##1{\textcolor[rgb]{0.73,0.73,0.73}{##1}}}
\@namedef{PYG@tok@c}{\let\PYG@it=\textit\def\PYG@tc##1{\textcolor[rgb]{0.24,0.48,0.48}{##1}}}
\@namedef{PYG@tok@cp}{\def\PYG@tc##1{\textcolor[rgb]{0.61,0.40,0.00}{##1}}}
\@namedef{PYG@tok@k}{\let\PYG@bf=\textbf\def\PYG@tc##1{\textcolor[rgb]{0.00,0.50,0.00}{##1}}}
\@namedef{PYG@tok@kp}{\def\PYG@tc##1{\textcolor[rgb]{0.00,0.50,0.00}{##1}}}
\@namedef{PYG@tok@kt}{\def\PYG@tc##1{\textcolor[rgb]{0.69,0.00,0.25}{##1}}}
\@namedef{PYG@tok@o}{\def\PYG@tc##1{\textcolor[rgb]{0.40,0.40,0.40}{##1}}}
\@namedef{PYG@tok@ow}{\let\PYG@bf=\textbf\def\PYG@tc##1{\textcolor[rgb]{0.67,0.13,1.00}{##1}}}
\@namedef{PYG@tok@nb}{\def\PYG@tc##1{\textcolor[rgb]{0.00,0.50,0.00}{##1}}}
\@namedef{PYG@tok@nf}{\def\PYG@tc##1{\textcolor[rgb]{0.00,0.00,1.00}{##1}}}
\@namedef{PYG@tok@nc}{\let\PYG@bf=\textbf\def\PYG@tc##1{\textcolor[rgb]{0.00,0.00,1.00}{##1}}}
\@namedef{PYG@tok@nn}{\let\PYG@bf=\textbf\def\PYG@tc##1{\textcolor[rgb]{0.00,0.00,1.00}{##1}}}
\@namedef{PYG@tok@ne}{\let\PYG@bf=\textbf\def\PYG@tc##1{\textcolor[rgb]{0.80,0.25,0.22}{##1}}}
\@namedef{PYG@tok@nv}{\def\PYG@tc##1{\textcolor[rgb]{0.10,0.09,0.49}{##1}}}
\@namedef{PYG@tok@no}{\def\PYG@tc##1{\textcolor[rgb]{0.53,0.00,0.00}{##1}}}
\@namedef{PYG@tok@nl}{\def\PYG@tc##1{\textcolor[rgb]{0.46,0.46,0.00}{##1}}}
\@namedef{PYG@tok@ni}{\let\PYG@bf=\textbf\def\PYG@tc##1{\textcolor[rgb]{0.44,0.44,0.44}{##1}}}
\@namedef{PYG@tok@na}{\def\PYG@tc##1{\textcolor[rgb]{0.41,0.47,0.13}{##1}}}
\@namedef{PYG@tok@nt}{\let\PYG@bf=\textbf\def\PYG@tc##1{\textcolor[rgb]{0.00,0.50,0.00}{##1}}}
\@namedef{PYG@tok@nd}{\def\PYG@tc##1{\textcolor[rgb]{0.67,0.13,1.00}{##1}}}
\@namedef{PYG@tok@s}{\def\PYG@tc##1{\textcolor[rgb]{0.73,0.13,0.13}{##1}}}
\@namedef{PYG@tok@sd}{\let\PYG@it=\textit\def\PYG@tc##1{\textcolor[rgb]{0.73,0.13,0.13}{##1}}}
\@namedef{PYG@tok@si}{\let\PYG@bf=\textbf\def\PYG@tc##1{\textcolor[rgb]{0.64,0.35,0.47}{##1}}}
\@namedef{PYG@tok@se}{\let\PYG@bf=\textbf\def\PYG@tc##1{\textcolor[rgb]{0.67,0.36,0.12}{##1}}}
\@namedef{PYG@tok@sr}{\def\PYG@tc##1{\textcolor[rgb]{0.64,0.35,0.47}{##1}}}
\@namedef{PYG@tok@ss}{\def\PYG@tc##1{\textcolor[rgb]{0.10,0.09,0.49}{##1}}}
\@namedef{PYG@tok@sx}{\def\PYG@tc##1{\textcolor[rgb]{0.00,0.50,0.00}{##1}}}
\@namedef{PYG@tok@m}{\def\PYG@tc##1{\textcolor[rgb]{0.40,0.40,0.40}{##1}}}
\@namedef{PYG@tok@gh}{\let\PYG@bf=\textbf\def\PYG@tc##1{\textcolor[rgb]{0.00,0.00,0.50}{##1}}}
\@namedef{PYG@tok@gu}{\let\PYG@bf=\textbf\def\PYG@tc##1{\textcolor[rgb]{0.50,0.00,0.50}{##1}}}
\@namedef{PYG@tok@gd}{\def\PYG@tc##1{\textcolor[rgb]{0.63,0.00,0.00}{##1}}}
\@namedef{PYG@tok@gi}{\def\PYG@tc##1{\textcolor[rgb]{0.00,0.52,0.00}{##1}}}
\@namedef{PYG@tok@gr}{\def\PYG@tc##1{\textcolor[rgb]{0.89,0.00,0.00}{##1}}}
\@namedef{PYG@tok@ge}{\let\PYG@it=\textit}
\@namedef{PYG@tok@gs}{\let\PYG@bf=\textbf}
\@namedef{PYG@tok@gp}{\let\PYG@bf=\textbf\def\PYG@tc##1{\textcolor[rgb]{0.00,0.00,0.50}{##1}}}
\@namedef{PYG@tok@go}{\def\PYG@tc##1{\textcolor[rgb]{0.44,0.44,0.44}{##1}}}
\@namedef{PYG@tok@gt}{\def\PYG@tc##1{\textcolor[rgb]{0.00,0.27,0.87}{##1}}}
\@namedef{PYG@tok@err}{\def\PYG@bc##1{{\setlength{\fboxsep}{\string -\fboxrule}\fcolorbox[rgb]{1.00,0.00,0.00}{1,1,1}{\strut ##1}}}}
\@namedef{PYG@tok@kc}{\let\PYG@bf=\textbf\def\PYG@tc##1{\textcolor[rgb]{0.00,0.50,0.00}{##1}}}
\@namedef{PYG@tok@kd}{\let\PYG@bf=\textbf\def\PYG@tc##1{\textcolor[rgb]{0.00,0.50,0.00}{##1}}}
\@namedef{PYG@tok@kn}{\let\PYG@bf=\textbf\def\PYG@tc##1{\textcolor[rgb]{0.00,0.50,0.00}{##1}}}
\@namedef{PYG@tok@kr}{\let\PYG@bf=\textbf\def\PYG@tc##1{\textcolor[rgb]{0.00,0.50,0.00}{##1}}}
\@namedef{PYG@tok@bp}{\def\PYG@tc##1{\textcolor[rgb]{0.00,0.50,0.00}{##1}}}
\@namedef{PYG@tok@fm}{\def\PYG@tc##1{\textcolor[rgb]{0.00,0.00,1.00}{##1}}}
\@namedef{PYG@tok@vc}{\def\PYG@tc##1{\textcolor[rgb]{0.10,0.09,0.49}{##1}}}
\@namedef{PYG@tok@vg}{\def\PYG@tc##1{\textcolor[rgb]{0.10,0.09,0.49}{##1}}}
\@namedef{PYG@tok@vi}{\def\PYG@tc##1{\textcolor[rgb]{0.10,0.09,0.49}{##1}}}
\@namedef{PYG@tok@vm}{\def\PYG@tc##1{\textcolor[rgb]{0.10,0.09,0.49}{##1}}}
\@namedef{PYG@tok@sa}{\def\PYG@tc##1{\textcolor[rgb]{0.73,0.13,0.13}{##1}}}
\@namedef{PYG@tok@sb}{\def\PYG@tc##1{\textcolor[rgb]{0.73,0.13,0.13}{##1}}}
\@namedef{PYG@tok@sc}{\def\PYG@tc##1{\textcolor[rgb]{0.73,0.13,0.13}{##1}}}
\@namedef{PYG@tok@dl}{\def\PYG@tc##1{\textcolor[rgb]{0.73,0.13,0.13}{##1}}}
\@namedef{PYG@tok@s2}{\def\PYG@tc##1{\textcolor[rgb]{0.73,0.13,0.13}{##1}}}
\@namedef{PYG@tok@sh}{\def\PYG@tc##1{\textcolor[rgb]{0.73,0.13,0.13}{##1}}}
\@namedef{PYG@tok@s1}{\def\PYG@tc##1{\textcolor[rgb]{0.73,0.13,0.13}{##1}}}
\@namedef{PYG@tok@mb}{\def\PYG@tc##1{\textcolor[rgb]{0.40,0.40,0.40}{##1}}}
\@namedef{PYG@tok@mf}{\def\PYG@tc##1{\textcolor[rgb]{0.40,0.40,0.40}{##1}}}
\@namedef{PYG@tok@mh}{\def\PYG@tc##1{\textcolor[rgb]{0.40,0.40,0.40}{##1}}}
\@namedef{PYG@tok@mi}{\def\PYG@tc##1{\textcolor[rgb]{0.40,0.40,0.40}{##1}}}
\@namedef{PYG@tok@il}{\def\PYG@tc##1{\textcolor[rgb]{0.40,0.40,0.40}{##1}}}
\@namedef{PYG@tok@mo}{\def\PYG@tc##1{\textcolor[rgb]{0.40,0.40,0.40}{##1}}}
\@namedef{PYG@tok@ch}{\let\PYG@it=\textit\def\PYG@tc##1{\textcolor[rgb]{0.24,0.48,0.48}{##1}}}
\@namedef{PYG@tok@cm}{\let\PYG@it=\textit\def\PYG@tc##1{\textcolor[rgb]{0.24,0.48,0.48}{##1}}}
\@namedef{PYG@tok@cpf}{\let\PYG@it=\textit\def\PYG@tc##1{\textcolor[rgb]{0.24,0.48,0.48}{##1}}}
\@namedef{PYG@tok@c1}{\let\PYG@it=\textit\def\PYG@tc##1{\textcolor[rgb]{0.24,0.48,0.48}{##1}}}
\@namedef{PYG@tok@cs}{\let\PYG@it=\textit\def\PYG@tc##1{\textcolor[rgb]{0.24,0.48,0.48}{##1}}}

\def\PYGZbs{\char`\\}
\def\PYGZus{\char`\_}
\def\PYGZob{\char`\{}
\def\PYGZcb{\char`\}}
\def\PYGZca{\char`\^}
\def\PYGZam{\char`\&}
\def\PYGZlt{\char`\<}
\def\PYGZgt{\char`\>}
\def\PYGZsh{\char`\#}
\def\PYGZpc{\char`\%}
\def\PYGZdl{\char`\$}
\def\PYGZhy{\char`\-}
\def\PYGZsq{\char`\'}
\def\PYGZdq{\char`\"}
\def\PYGZti{\char`\~}
% for compatibility with earlier versions
\def\PYGZat{@}
\def\PYGZlb{[}
\def\PYGZrb{]}
\makeatother

% CompSuite specific commands
\newcommand{\toolcompsuite}{\textsc{CompSuite}\xspace}
\newcommand{\runner}{\textsc{CompRunner}\xspace}
% MPChecker specific commands
\newcommand{\toolchecker}{\textsc{MPChecker}\xspace}
\newcommand{\vanillatool}{\textsc{Vanilla MPChecker}\xspace}
\newcommand{\fuzzytool}{\textsc{Fuzzy MPChecker}\xspace}
\newcommand{\Code}[1]{$\mathtt{#1}$}
% DataLoc specific commands
\newcommand{\tooldataloc}{\textsc{DataLoc}\xspace}
\newcommand{\dataset}{\textsc{KA-LogicQuery}\xspace}
\newcommand{\negset}{\textsc{KA-LogicQuery-Neg}\xspace}
\newcommand{\souffle}{Soufflé}

\definecolor{dkgreen}{rgb}{0,0.6,0}
\definecolor{gray}{rgb}{0.5,0.5,0.5}
\definecolor{mauve}{rgb}{0.58,0,0.82}

\lstset{frame=tb,
  language=Python,
  aboveskip=-1mm,
  belowskip=-1mm,
  showstringspaces=false,
  columns=flexible,
  basicstyle={\footnotesize\ttfamily},
  numbers=left,
  numbersep=5pt,
  frame=none,
  numberstyle=\tiny\color{gray},
  keywordstyle=\color{blue},
  commentstyle=\color{dkgreen},
  % stringstyle=\color{mauve},
  breaklines,
  breakatwhitespace=false,
  tabsize=4
}

\usepackage[T1]{fontenc}



\title{\ttitle} % Defines the thesis title - don't touch this

%extra packages
\usepackage{float}
\usepackage{color,soul}             
\usepackage{Styles/mydefs}
\usepackage{fancybox}
\usepackage[most]{tcolorbox}
\usepackage{tikz}
\usetikzlibrary{arrows.meta, positioning, calc}

\newtcolorbox{stepbox}[2][]{
    enhanced,
    colback=white,
    colframe=gray!30,
    title=#2,
    fonttitle=\bfseries\sffamily,
    coltitle=black,
    attach boxed title to top left={xshift=3mm, yshift=-2mm},
    boxed title style={colback=gray!10, colframe=gray!30},
    drop shadow={gray!20},
    #1
}

\usepackage{minted}
\setminted[prolog]{
    linenos=true,
    breaklines=true,
    tabsize=4,
    xleftmargin=1.5em,
    fontsize=\footnotesize
}
\newcolumntype{C}{>{\centering\arraybackslash}X}

\begin{document}

\frontmatter % Use roman page numbering style (i, ii, iii, iv...) for the pre-content pages

\setstretch{1.3} % Line spacing of 1.3
% Define the page headers using the Fr package and set up for one-sided printing

\pagestyle{fancy} % Finally, use the "fancy" page style to implement the FancyHdr headers
\fancyhead{}   %clear all fields
\fancyhead[LO]{\sl{\leftmark}}
\fancyhead[RE]{\sl{\rightmark}}
\fancyhead[LE,RO]{\thepage}


%----------------------------------------------------------------------------------------
%	TITLE PAGE
%----------------------------------------------------------------------------------------

%%% Use \maketitleforreview instead of \maketitle, if you want a PLAIN TITLE PAGE

\maketitle
% \maketitleforreview


% ----------------------------------------------------------------------------------------
% 	DECLARATION PAGE
% ----------------------------------------------------------------------------------------

\thesisdeclareOriginality{I hereby certify that the work embodied in this thesis is the result of original research, is free of plagiarised materials, and has not been submitted for a higher degree to any other University or Institution.}{Feb. 2026}{Styles/signature.png}

\thesisdeclareSupervisor{I have reviewed the content and presentation style of this thesis and declare it is free of plagiarism and of sufficient grammatical clarity to be examined.  To the best of my knowledge, the research and writing are those of the candidate except as acknowledged in the Author Attribution Statement. I confirm that the investigations were conducted in accord with the ethics policies and integrity standards of Nanyang Technological University and that the research data are presented honestly and without prejudice.}{Mar. 2026}{Styles/signature.png}

\thesisdeclareAuthorship{This thesis contains material from two published peer-reviewed conference papers and one completed manuscript under review, all of which I am listed as an author.}
{
	Chapter 4 is published as {\color{blue}Xiufeng Xu, Chenguang Zhu, and Yi Li. Compsuite: A dataset of java library upgrade incompatibility issues. In International Conference on Automated Software Engineering. 2023. DOI: 10.1109/ASE56229.2023.00127.}
	
	The contributions of the co-authors are as follows:
	\begin{itemize}[topsep=1pt,itemsep=1pt,partopsep=1pt, parsep=1pt]
		\item Prof. Yi Li and Dr. Chenguang Zhu proposed the topic and provided insightful comments on the methodology and experiment design.
		\item I constructed the dataset, designed and implemented the tool, performed the experiments, analyzed the data and drafted the manuscript.
	\end{itemize}

	Chapter 5 is published as {\color{blue}Xiufeng Xu, Fuman Xie, Chenguang Zhu, Guangdong Bai, Sarfraz Khurshid, and Yi Li. Identifying Multi-parameter Constraint Errors in Python Data Science Library API Documentation. In Proceedings of the ACM on Software Engineering. 2025. DOI: 10.1145/3728945.}
	
	The contributions of the co-authors are as follows:
	\begin{itemize}[topsep=1pt,itemsep=1pt,partopsep=1pt, parsep=1pt]
		\item Prof. Yi Li and Dr. Chenguang Zhu proposed the topic and revised the manuscript.
		\item I co-designed the methodology, implemented and evaluated the approach, and drafted the manuscript.
		\item Fuman Xie contributed to the implementation of the tool, assisted in conducting the evaluation, and refined the manuscript.
		\item Prof. Sarfraz Khurshid and Prof. Guangdong Bai provided insightful comments on the methodology design.
	\end{itemize}

	Chapter 6 is from a completed paper under review.
	
	The contributions of the co-authors are as follows:
	\begin{itemize}[topsep=1pt,itemsep=1pt,partopsep=1pt, parsep=1pt]
		\item Prof. Yi Li proposed the topic, supported the research, provided guidance to the methodology design, and revised the manuscript.
		\item I co-designed the methodology, implemented the tool, conducted the experiments, analyzed the data, and drafted the manuscript.
		\item Dr. Xiuheng Wu contributed to the implementation of the tool and assisted in conducting the experiments.
		\item Dr. Zejun Zhang provided insightful comments and helped to write the related works section.
	\end{itemize}

}{Feb. 2026}{Styles/signature.png}

% ----------------------------------------------------------------------------------------
% 	ACKNOWLEDGEMENTS
% ----------------------------------------------------------------------------------------

\setstretch{1.3} % Reset the line-spacing to 1.3 for body text (if it has changed)

\acknowledgements{\addtocontents{toc}{\vspace{0.8em}} % Add a gap in the Contents, for aesthetics

I wish to express my greatest gratitude to my advisor.

% \begin{flushright}
% \emph{Jinming Xu, December 2015}
% \end{flushright}
}

%----------------------------------------------------------------------------------------
%	QUOTATION PAGE
%----------------------------------------------------------------------------------------

\pagestyle{empty} % No headers or footers for the following pages
\emph{``If I had one hour to save the world, I would spend 55 minutes defining the problem and only five minutes finding the solution."}

\begin{flushright}
---Einstein, Albert
\end{flushright}
\null\vfill % Add some space to move the quote down the page a bit

\begin{center}
\large{To my dear family}
\end{center}
\vfill\vfill\null % Add some space at the bottom to position the quote just right
\cleardoublepage % Start a new page and make the next page a right-hand (odd-numbered) page, producing a blank page if necessary

\pagestyle{fancy} % Return the page headers back to the "fancy" style
\fancyhead{}   %clear all fields
\fancyhead[LO]{\sl{\leftmark}}
\fancyhead[RE]{\sl{\rightmark}}
\fancyhead[LE,RO]{\thepage}

%----------------------------------------------------------------------------------------
%	ABSTRACT PAGE
%----------------------------------------------------------------------------------------

\addtotoc{Abstract} % Add the "Abstract" page entry to the Contents

\abstract{\addtocontents{toc}{\vspace{0.8em}}}% Add a gap in the Contents, for aesthetics

% 1. The Problem 2. The Issues 3. The Approach 4. The Results 5. The Experiments 6. The Applications

My abstracts



%----------------------------------------------------------------------------------------
%	LIST OF CONTENTS/FIGURES/TABLES PAGES
%----------------------------------------------------------------------------------------

\pagestyle{fancy} % The page style headers have been "empty" all this time, now use the "fancy" headers as defined before to bring them back

\tableofcontents % Write out the Table of Contents

\listoffigures % Write out the List of Figures

\listoftables % Write out the List of Tables


%----------------------------------------------------------------------------------------
%	ABBREVIATIONS
%----------------------------------------------------------------------------------------

%\clearpage % Start a new page
%
%\setstretch{1.5} % Set the line spacing to 1.5, this makes the following tables easier to read
%
%\lhead{\emph{Abbreviations}} % Set the left side page header to "Abbreviations"
%\listofsymbols{ll} % Include a list of Abbreviations (a table of two columns)
%{
%\textbf{LAH} & \textbf{L}ist \textbf{A}bbreviations \textbf{H}ere \\
%%\textbf{Acronym} & \textbf{W}hat (it) \textbf{S}tands \textbf{F}or \\
%}

%----------------------------------------------------------------------------------------
%	PHYSICAL CONSTANTS/OTHER DEFINITIONS
%----------------------------------------------------------------------------------------

%\clearpage % Start a new page
%
%\lhead{\emph{Physical Constants}} % Set the left side page header to "Physical Constants"
%
%\listofconstants{lrcl} % Include a list of Physical Constants (a four column table)
%{
%Speed of Light & $c$ & $=$ & $2.997\ 924\ 58\times10^{8}\ \mbox{ms}^{-\mbox{s}}$ (exact)\\
%% Constant Name & Symbol & = & Constant Value (with units) \\
%}

%----------------------------------------------------------------------------------------
%	SYMBOLS
%----------------------------------------------------------------------------------------

% \cleardoublepage  % Start a new page (no need if the previous page is list of figures)

% \listofnomenclature{ll} % Include a list of Symbols (a three column table)
% {
% \multicolumn{2}{l}{\LARGE{\textbf{Symbols}}}\\[0.618cm]
% $\mathcal{R}^n$                    &the $n$-dimensional Euclidean space\\
% $\mathcal{H}$                      &the Euclidean  space\\
% $\norm{\cdot}$                     &the 2-norm of a vector or matrix in Euclidean space\\
% $\norm{\cdot}_G$                   &the induced norm of a vector in G-space\\
% $\norm{\cdot}_E$                   &the induced norm of a vector or matrix in probabilistic space\\


% $\odot$                            &the Hadamard (component-wise) product\\
% $\otimes$                          &the Kronecker product\\
% $\innprod{\cdot}{\cdot}$           &the inner product of two vectors\\
% $\circ$                            &the composition of functions\\ [0.618cm]


% $\nabla f$                    &the gradient vector\\
% $\mathcal{C}^k$               &the function with continuous partial derivatives up to $k$ orders\\
% % $T_x\mathcal{M}$              &the tangent space of the set $\mathcal{M}$\\
% % $x_i$                         &the $i$-th component of a vector $x$\\
% $x_{i,k}$                     &the $i$-th component of a vector $x$ at time $k$\\
% $\bar{x}$                     &the vector with the average of all components of $x$ as each element\\
% $\ones$                       &all-ones column vector with proper dimension\\
% $\mathcal{C}$                 &the average space, i.e., $span\{{\bf 1}\}$ \\
% $\mathcal{C}^\perp$           &the disagreement space, i.e., $span^\perp\{{\bf 1}\}$ \\
% % W                             &the weight matrix\\
% % L                             &the Laplacian matrix\\
% $\Pi_\parallel$               &the projection matrix to the average space $\mathcal{C}$\\
% $\Pi_\perp$                   &the projection matrix to the disagreement space $\mathcal{C}^\perp$ \\
% $O(\cdot)$                    &order of magnitude or ergodic convergence rate (running average)\\
% $o(\cdot)$                    &non-ergodic convergence rate\\
% % $O(\cdot)$                    &the ergodic convergence rate stated in terms of the running average\\
% % $o(\cdot)$                    &the non-ergodic convergence rate stated in terms of $x_k$\\



% $\mathcal{N}_i$               &the index set of the neighbors of agent $i$ \\ [1cm]

% \multicolumn{2}{l}{\LARGE{\textbf{Acronyms}}}\\[0.618cm]
% DOP                      & Distributed Optimization Problem\\
% EDOP                     & Equivalent Distributed Optimization Problem\\
% SDOP                     & Stochastic Distributed Optimization Problem\\
% OEP                      & Optimal Exchange Problem\\
% OCP                      & Optimal Consensus Problem\\
% DOCP                     & Dynamic Optimal Consensus Problem\\[0.618cm]
% AugDGM           & Augmented Distributed Gradient Methods\\
% AsynDGM          & Asynchronous Distributed Gradient Methods\\
% D-ESC            & Distributed Extremum Seeking Control\\
% D-SPA            & Distributed Simultaneous Perturbation Approach\\
% % D-PDSPA & Distributed Primal-Dual Simultaneous Perturbation Approach\\
% D-FBBS           & Distributed Forward-Backward Bregman Splitting\\
% ADMM             & Alternating Direction Method of Multipliers\\
% DSM              & Distributed (Sub)gradient Method\\[0.618cm]

% GAS                      & Globally Asymptotically Stable \\
% UGAS                     & Uniformly Globally Asymptotically Stable \\
% SPAS                     & Semi-globally Practically Asymptotically Stable\\
% USPAS                    & Uniformly Semi-globally Practically Asymptotically Stable\\[0.618cm]
% HoS                      & Heterogeneity of Stepsize\\
% FPR                      & Fixed Point Residual\\
% OBE                      & Objective Error\\[0.618cm]
% i.i.d.           & independent and identically distributed\\
% $a.s.$           & almost sure convergence of a random sequence\\
% }

%----------------------------------------------------------------------------------------
%	THESIS CONTENT - CHAPTERS
%----------------------------------------------------------------------------------------

\mainmatter       % Begin numeric (1,2,3...) page numbering
\pagenumbering{arabic}
\setstretch{1.3}  % Return the line spacing back to 1.3

% define the headings for the body of the thesis
\fancyhead{}   %clear all fields
\fancyhead[LO]{\sl{\leftmark}}
\fancyhead[RE]{\sl{\rightmark}}
\fancyhead[LE,RO]{\thepage}


% Include the chapters of the thesis as separate files from the Chapters folder
% Uncomment the lines as you write the chapters

%!TEX root=../mythesis.tex
% Chapter 1

\chapter{Introduction} % Main chapter title
\chaptermark{Introduction}
\label{ch:introduction}

\section{Motivation}\label{sec:motivation}
In the context of rapid digital transformation and the accelerated advancedment of Artificial Intelligence (AI), software systems have evoloved from relatively static artifacts into dynamic entities capable of continuously adapting to changing requirements. To remain effective in the face of evolving requirements, technological innovations, and emerging security threats, ongoing modification and extension, collectively known as software evolution, have become an inherent and unavoidable characteristic of modern software engineering. Broadly, software evolution is primarily driven by two forces: \textbf{external adaptation}, which responds to changes in the third-party ecosystem, and \textbf{internal improvement}, which systematically modify code and update corresponding documentation to satisfy new functional and quality requirements.

In recent years, the deep integration of AI, especially Large Language Models (LLMs), into the software development lifecycle has signaled a significant paradigm shift in software evolution. Empowered by advances in artificial intelligence, intelligent development tools and autonomous agents are increasingly capable of understanding and processing large-scale codebases alongside natural language requirements, while progressively demonstrating higher-level reasoning abilities such as cross-context analysis and complicated problem decomposition. Leveraging these capabilities, AI systems can automate a wide range of labor-intensive and high-complexity tasks, such as code generation, bug fixing, documentation synchronization, and system maintenance, thereby reshaping critical stages of software evolution and enabling complex software systems to achieve continuous iteration and scalable growth at unprecedented speed. Meanwhile, this AI-driven development paradigm introduce new technical pathways for managing long-term software evolution, propelling software engineering beyond human-centered development model toward a new stage characterized by the coexistence of human–AI collaboration and autonomous intelligent.

Despite these advances, the application of LLMs in software evolution remains fundamentally constrained by their probabilistic generative nature. Although large-scale parameterization facilitates the emergence of complex reasoning capabilities, LLMs generate outputs based on learned statistical patterns rather than deterministic logic, making it a probabilistic mirage that lacks the deterministic guarantees required for absolute reliability. At the same time, modern software systems have evolved into highly interconnected dependency networks: a project is not only an internal system but also a third-party library for other projects. Allowing AI agents to arbitrarily modify code and documentation is highly risky, may inadvertently introduce new vulnerabilities or disrupt existing functionalities. In this context, even seemingly minor modifications, such as a library upgrade or internal refactoring, can propagate cascading effects that compromise system reliability, security, and maintainability at scale. 

Consequently, ensuring the reliability and robustness throughout AI-assisted software evolution can be distilled into three interrelated core challenges:

\begin{itemize}
  \item \textbf{How to handle external ecosystem changes?} The lack of high-quality, real-world benchmark data that captures intricate patterns of breaking changes caused by third-party library upgrades limits AI’s ability to understand and solve external dependency risks.
  \item \textbf{How to preserve internal semantic consistency?} Code and documentation are often updated asynchronously. Since a project simultaneously functions as a third-party dependency for others, its documentation forms the outward semantic interface; internal inconsistencies therefore directly amplify external usage risks.
  \item \textbf{How to perform intelligent code modification?} Accurate code localization is the prerequisite for all automated modification tasks. However, existing LLM-driven approaches still rely heavily on superficial textual cues rather than deep semantic and logical reasoning over program structure, leading to imprecise identification of modification scopes and degraded evolution quality. 
\end{itemize}

These three challenges correspond respectively to external adaptation, internal consistency, and reliable execution of evolution, together forming the central bottleneck in AI-enabled software evolution and providing the systematic research motivation and theoretical foundation for this dissertation.

\section{Contributions}\label{sec:contribution}

\begin{figure}[htbp]

\begin{tikzpicture}[
    node distance = 1.5cm and 0.2em,
    chapter_node/.style = {
        align=center,
        font=\sffamily\small,
        % text width=0.28\textwidth,
        inner sep=2pt,
    },
    arrow/.style = {
        -{Stealth[scale=1.2]},
        thin
    }
]

    \node[chapter_node] (c1) {
        \textcolor{blue}{\underline{\textbf{Chapter 1}}} \\ 
        Introduction
    };

    \node[chapter_node, below=of c1] (c2) {
        \textcolor{blue}{\underline{\textbf{Chapter 2}}} \\
        Background and Literature Review
    };

    \node[chapter_node, below=1.5cm of c2] (c4) {
        \textcolor{blue}{\underline{\textbf{Chapter 4}}} \\
        \textbf{\textsc{MPChecker}:} Multi-parameter \\ 
        Code-Doc Inconsistency Detection\\ 
        \textcolor{blue}{$\bullet$ new approach}
    };

    \node[chapter_node, left= of c4] (c3) {
        \textcolor{blue}{\underline{\textbf{Chapter 3}}} \\
        \textbf{\textsc{CompSuite}:} A Dataset of Lib-\\ 
        rary Upgrade Incompatibilities\\
        \textcolor{blue}{$\bullet$ new dataset}
    };

    \node[chapter_node, right= of c4] (c5) {
        \textcolor{blue}{\underline{\textbf{Chapter 5}}} \\
        \textbf{\textsc{DataLoc}:} Neurosymbolic \\ 
        Repo-level Code Localization\\ 
        \textcolor{blue}{$\bullet$ findings \& new approach}
    };

    \node[chapter_node, below=1.5cm of c4] (c6) {
        \textcolor{blue}{\underline{\textbf{Chapter 6}}} \\
        Conclusion and Future Work
    };

    \draw[arrow] (c1.south) -- (c2.north);

    \coordinate (split) at ($(c2.south)!0.3!(c4.north)$);
    \draw (c2.south) -- (split); 

    \draw[arrow] (split) .. controls ++(0,-1.2) and ++(0,1.2) .. (c3.north);
    \draw[arrow] (split) -- (c4.north);
    \draw[arrow] (split) .. controls ++(0,-1.2) and ++(0,1.2) .. (c5.north);

    \coordinate (merge) at ($(c4.south)!0.7!(c6.north)$);

    \draw[arrow] (merge) -- (c6.north); 
    \draw (c3.south) .. controls ++(0,-1.2) and ++(0,1.2) .. (merge);
    \draw (c4.south) -- (merge);
    \draw (c5.south) .. controls ++(0,-1.2) and ++(0,1.2) .. (merge);

\end{tikzpicture}
\caption{The Structure of the Thesis}\label{fig:thesis_structure}
\end{figure}



\section{Organization}\label{sec:organization}



%!TEX root=../mythesis.tex
% Chapter Template

\chapter{background} % Main chapter title
\chaptermark{Background}  % replace the chapter name with its abbreviated form
\label{ch:chapter2} % Change X to a consecutive number; for referencing this chapter elsewhere, use \ref{ChapterX}

%\lhead{Chapter X. \emph{Chapter Title Here}} % Change X to a consecutive number; this is for the header on each page - perhaps a shortened title
%-----------------------------------
% SECTION 1
%-----------------------------------

\section{Software Evolution and Maintenance}
%!TEX root=../mythesis.tex
% Chapter Template

\chapter{CompSuite} % Main chapter title
\chaptermark{CompSuite}  % replace the chapter name with its abbreviated form
\label{ch:chapter3} % Change X to a consecutive number; for referencing this chapter elsewhere, use \ref{ChapterX}
\section{Introduction}\label{sec:introduction-compsuite}

Modern software systems are becoming increasingly complex due to the
need for integrating various components developed by different teams
or organizations. These components are often subject to continuous
evolution, and as a result, ensuring that new
upgrades to third-party libraries do not cause any compatibility
issues with the existing software system is a challenging task. The
complexity of these systems and the number of dependencies involved
make it difficult to anticipate and identify incompatibilities that
may arise from updates to external components. Incompatibility issues resulting from upgrades to external components can
compromise the reliability of software systems, potentially leading to
significant financial losses for the organizations that rely on these systems.

Many techniques have been proposed to address third-party library
compatibility issues, including regression
testing~\cite{mezzetti2018type,moller2019model}, static
analysis~\cite{foo2018efficient}, dependency conflict
detection~\cite{wang2021will}, and client-specific compatibility
checking~\cite{mora2018client,zhu2019framework}. These techniques address
library compatibility issues in different dimensions and have been
evaluated with their own isolated datasets.


An excellent dataset has the potential to serve as a valuable reference for future research in
this field. However, composing the dataset requires intricate
manual validation, e.g., confirming whether the cause of a test
failure is due to runtime exception, assertion violation, or other reasons.
Therefore, we propose \toolcompsuite, the first incompatibility issue dataset focusing on
library behavioral incompatibility with concrete reproducible test cases.
Each test case is isolated and validated, enabling the direct manifestation of the
incompatibilities.


\toolcompsuite comprises 123 real-world Java client-library pairs such that upgrading any library results in
incompatibility issues for the corresponding client.
Every incompatibility issue in \toolcompsuite contains a test case created by developers, allowing for the
reproduction of the issue.
On top of this dataset, we also developed an automated command-line interface, which streamlines
all processes of the reproduction, such as downloading and compiling a projects, running target
tests and re-runing the tests after a library upgrade.
With this infrastructure, users may reproduce an incompatibility issue programmatically with
minimal efforts.




% In this study, we present \toolcompsuite, a dataset comprising 123 real-world Java client-library pairs in which upgrading the library results in compatibility problems for the respective client. Each compatibility issue in CompSuite is linked to a test case created by developers, allowing for the reproduction of the issue. Additionally, the dataset features an automated tool that streamlines the execution and verification of each problem. This setup enables users to swiftly assess any compatibility issue with a single click or to reproduce the issue methodically for an in-depth examination.


\textbf{Contribution}
To summarize, we make the following contributions in this paper:%
\begin{enumerate}[leftmargin=*,topsep=2pt]
\item We construct a dataset, \toolcompsuite, including 123
  reproducible, real-world client-library pairs that manifest
  incompatibility issues when upgrading the library. These data points
  originate from 88 clients and 104 libraries.
  %% We forked each project and identified the base version that
  %% passed all tests before encountering incompatibility
  %% issues. Based on that version, we performed library updates and
  %% utilized a version control system to create isolated commits,
  %% with each version differing only in the library version.
\item We created an automated command-line interface for the dataset.
With this interface, users are able to programmatically replicate an incompatibility issue
from the dataset with a single command.
The interface also offers separate commands for each step involved in the reproduction of
incompatibility issues.
\end{enumerate}


% \todo{TODO: Add a paragraph here: We envision that \toolcompsuite can power various
%   program analysis techniques including xxx, xxx, and xxx. Briefly
%   describe how each of these techniques would use \toolcompsuite.}

We envision that \toolcompsuite to be used to evaluate various program analysis techniques, including
compatibility checking, module-level regression testing selection, and debugging techniques.
%For example, compatibility checkers and detecters can use \toolcompsuite as a performance evaluator, while
%module-level RTS tools utilize it as a safety evaluator to verify whether all relevant test cases
%have been selected. Additionally, \toolcompsuite is able to assist debugging techniques to do better root
%cause analysis.
More detailed information can be found in \cref{sec:app-compsuite}.

The dataset and tool are available at: \url{https://github.com/compsuite-team/compsuite}.


\section{Dataset Creation}\label{creation}

In this section, we outline the methodology and process employed to
create the \toolcompsuite dataset.

\subsection{Subjects Selection}

\begin{table*}[ht]
    \setlength{\abovecaptionskip}{5pt}
    \caption{Details of clients and libraries included in \toolcompsuite.}
  \small
    \begin{tabular}{lrrclr}
        \cline{1-3} \cline{5-6}
        \textbf{Client}         & \textbf{\#LoC} & \textbf{\#Star} &  & \textbf{Library}                            & \textbf{\#Maven Usage} \\ \cline{1-3} \cline{5-6}
        retrofit                & 29.7K          & 41.5K           &  & org.slf4j:slf4j-api                         & 62.5K                   \\
        apollo                  & 61.3K          & 28K             &  & com.google.guava:guava                      & 34.4K                   \\
        druid                   & 441.9K         & 26.8K           &  & org.scala-lang:scala-library                & 34K                     \\
        webmagic                & 17.4K          & 10.8K           &  & com.fasterxml.jackson.core:jackson-databind & 25.8K                   \\
        languagetool            & 171.2K         & 8.5K            &  & ch.qos.logback:logback-classic              & 25.5K                   \\ \cline{1-3} \cline{5-6}
        Other 83 clients (mean) & 371.6K         & 1.3K            &  & Other 99 libraries (mean)                   & 3.2K                    \\ \cline{1-3} \cline{5-6}
        All clients (mean)      & 358.7K         & 2.5K            &  & All libraries (mean)                        & 4.8K                    \\ \cline{1-3} \cline{5-6}
    \end{tabular}
    \label{client-library}
\end{table*}

To ensure the representativeness and reproducibility of the \toolcompsuite dataset, we
focus on including high-quality and popular client projects and
libraries. The selection of client projects was sourced from
GitHub~\cite{github}, a widely recognized online community for hosting
open-source codebases. To ensure the inclusion of the most
popular projects, we systematically sorted all the available projects
in descending order based on their number of stars on GitHub and
selected the target clients from the top of the list. The selection of
libraries was sourced from Maven Central~\cite{mvnrepo}, which hosts
33.5M of Java libraries and their associated binaries, making it
a widely used repository of libraries for Java API and library
research~\cite{mostafa2017study,wu2016exploratory,qiu2016understanding,kula2014visualizing}. We include a library in the dataset only
if it has more than 100 usages (i.e., clients) on Maven
Central. Our selection criteria aimed to ensure the inclusion of
popular and widely used client projects and libraries in the dataset,
thereby maximizing its relevance and usefulness to the research
community.

Among the highly-rated client projects, our selection criteria focused
on those that use Maven~\cite{mvn} as their build systems, given its
widespread adoption and maturity. Maven provides a standardized
approach to managing Java projects and their dependencies, where each
library dependency in a Maven client project is represented as an item
in a \Code{pom.xml} file, making it easy to identify and edit library
versions programmatically. Furthermore, Maven offers built-in
functionality for running unit tests and generating test reports,
which simplifies the identification and diagnosis of incompatibility
issues arising from test executions. Since Maven projects typically rely on Maven Central as their
centralized repository for hosting and downloading libraries, the process of obtaining and managing
libraries in our dataset is simplified.

\cref{client-library} presents the top 5 client projects and libraries in the \toolcompsuite dataset, ranked by popularity. For
each client project, we provide information on its lines of code (LoC)
and the number of stars it has received on GitHub, while for each
library, we include its number of usages by other projects from Maven Central.

In total, \toolcompsuite comprises 123 incompatible client-library pairs. These pairs encompass 88 distinct
clients and 104 libraries altogether. On average, the affected clients have 2.5K stars on GitHub
and 358.7K lines of code, while incompatible libraries have 4.8K usages on Maven Central.
Thus, we believe that the incompatibility issues present in
the \toolcompsuite dataset have a significant impact on a large number of
codebases and can affect many users of the libraries, either directly
or indirectly.



To ensure that all client projects in the dataset are executable and
the runs are reproducible, we performed a series of checks on each
project. First, we checked out the project to the version (SHA) at the time of
the dataset creation, which we refer to as the \emph{base version}.
Next, we ran the standard Maven project compilation command to verify if the
project compiles successfully. If the project fails to compile, we
excluded it from the dataset. Subsequently, we ran the standard Maven
test command to execute all the tests in the project, ensuring that
all tests pass on the base version. We excluded any project that fails
to pass tests at this stage. Finally, we only included the client
projects that successfully compile and pass all tests on the base version,
thereby ensuring that the dataset is only consist of projects which can
be executed and whose executions can be reproduced.


\subsection{Data Collection}\label{sec:data:collect-compsuite}

\begin{figure*}[ht]
  \setlength{\abovecaptionskip}{5pt}
  \setlength{\belowcaptionskip}{0pt}
  \centering
  \includegraphics[width=0.9\textwidth]{Figures/Chapter3/architecture.pdf}
  \caption{The architecture of \toolcompsuite.}
  \label{fig:architecture}
\end{figure*}



We collected the data following the below procedures.
\cref{fig:architecture} visualizes the overall architecture of \toolcompsuite.
In the upper left portion of \cref{fig:architecture}, we
illustrate the approach taken by \toolcompsuite to identify incompatibilities
between a client project and its dependent libraries. Specifically,
for each client project on its base version, we upgraded each of its
dependent libraries and tested if the upgrade caused any test
failures. Our intuition behind this approach is that since all the
tests in the client passed on the base version, if upgrading any
library causes a test failure, that library upgrade must have
introduced incompatibility issues. We refer to the test that flips
from passing to failing as an \emph{incompatibility-revealing test}.

To automatically upgrade the libraries and run the tests, we utilized
the Maven Versions Plugin~\cite{mvnplugin}. For a given client project, we
scanned its dependency list using this plugin to identify all the
libraries that had newer versions available on Maven Central. If a
library had a newer version, we marked it as upgradable. Next, for
each upgradable library, we used the plugin to upgrade it by updating
the \Code{pom.xml} file to the most recent version on Maven
Central. We then re-executed the test suite of the client. If any
tests failed during this run, we marked the client-library pair as
having an incompatibility issue and marked the test as an
incompatibility-revealing test of this issue. It is crucial to note
that we only upgraded one library at a time to isolate failures caused
by different libraries. To ensure the accuracy and dependability of
the dataset, we carried out a manual verification process for each
identified incompatibility issue. In particular, we carefully examined
the test failure messages and reports to confirm that they were indeed
caused by the upgraded library. For each incompatible client-library
pair, we selected a single incompatibility-revealing test to be
included in the final dataset. In cases where a client-library pair
had multiple incompatibility issues, we chose the one that we deemed
most representative and easy to comprehend.






% test fail due to upgrades
\begin{figure}
    \setlength{\abovecaptionskip}{5pt}
    \setlength{\belowcaptionskip}{0pt}
    \centering
\begin{Verbatim}[fontsize=\small,breaklines,linenos,xleftmargin=8pt,numbersep=5pt,commandchars=\\\{\}]
\PYG{p}{\PYGZob{}}
\PYG{+w}{  }\PYG{n+nt}{\PYGZdq{}id\PYGZdq{}}\PYG{p}{:}\PYG{+w}{ }\PYG{l+s+s2}{\PYGZdq{}i\PYGZhy{}49\PYGZdq{}}\PYG{p}{,}
\PYG{+w}{  }\PYG{n+nt}{\PYGZdq{}client\PYGZdq{}}\PYG{p}{:}\PYG{+w}{ }\PYG{l+s+s2}{\PYGZdq{}wasabi\PYGZdq{}}\PYG{p}{,}
\PYG{+w}{  }\PYG{n+nt}{\PYGZdq{}sha\PYGZdq{}}\PYG{p}{:}\PYG{+w}{ }\PYG{l+s+s2}{\PYGZdq{}9f2aa5f92e49c3844d787320e2d22e15317aa8e2\PYGZdq{}}\PYG{p}{,}
\PYG{+w}{  }\PYG{n+nt}{\PYGZdq{}url\PYGZdq{}}\PYG{p}{:}\PYG{+w}{ }\PYG{l+s+s2}{\PYGZdq{}https://github.com/intuit/wasabi\PYGZdq{}}\PYG{p}{,}
\PYG{+w}{  }\PYG{n+nt}{\PYGZdq{}lib\PYGZdq{}}\PYG{p}{:}\PYG{+w}{ }\PYG{l+s+s2}{\PYGZdq{}org.apache.httpcomponents:httpclient\PYGZdq{}}\PYG{p}{,}
\PYG{+w}{  }\PYG{n+nt}{\PYGZdq{}old\PYGZdq{}}\PYG{p}{:}\PYG{+w}{ }\PYG{l+s+s2}{\PYGZdq{}4.5.1\PYGZdq{}}\PYG{p}{,}
\PYG{+w}{  }\PYG{n+nt}{\PYGZdq{}new\PYGZdq{}}\PYG{p}{:}\PYG{+w}{ }\PYG{l+s+s2}{\PYGZdq{}4.5.10\PYGZdq{}}\PYG{p}{,}
\PYG{+w}{  }\PYG{n+nt}{\PYGZdq{}test\PYGZdq{}}\PYG{p}{:}\PYG{+w}{ }\PYG{l+s+s2}{\PYGZdq{}DefaultRestEndPointTest\PYGZsh{}testGetRestEndPointURI\PYGZdq{}}\PYG{p}{,}
\PYG{+w}{  }\PYG{n+nt}{\PYGZdq{}submodule\PYGZdq{}}\PYG{p}{:}\PYG{+w}{ }\PYG{l+s+s2}{\PYGZdq{}modules/export\PYGZdq{}}\PYG{p}{,}
\PYG{+w}{  }\PYG{n+nt}{\PYGZdq{}test\PYGZus{}cmd\PYGZdq{}}\PYG{p}{:}\PYG{+w}{ }\PYG{l+s+s2}{\PYGZdq{}mvn org.apache.maven.plugins:maven\PYGZhy{}surefire\PYGZhy{}plugin:2.20:test \PYGZhy{}fn \PYGZhy{}Drat.ignoreErrors=true \PYGZhy{}DtrimStackTrace=false \PYGZhy{}Dtest=DefaultRestEndPointTest\PYGZsh{}testGetRestEndPointURI\PYGZdq{}}
\PYG{p}{\PYGZcb{}}
\end{Verbatim}
\caption{The data schema of \toolcompsuite}\label{fig:json}
\end{figure}

Finally, we persisted the metadata of all the selected incompatibility
issues in a collection of \Code{json} files.
\cref{fig:json} presents the metadata of an incompatibility
issue in the \toolcompsuite dataset. The data schema includes the ID of the
issue, client project name, SHA of the client base version, URL of the
client project, library name, versions of the old and new libraries,
the name of the incompatibility-revealing test, the submodule
containing the incompatibility-revealing test, and the command to run
the test. The majority of the information is
self-explanatory. However, it is worth noting that the old version of
the library is the one utilized at the base version of the client,
while the new version is the most recent version found on Maven
Central that triggers the incompatibility when upgrading, as described in \cref{sec:data:collect-compsuite}.


\section{Dataset Usage}\label{sec:usage-compsuite}

In this section, we provide instructions on the usage of our dataset.

\subsection{Exploring an Incompatibility Issue}\label{sec:isolate-compsuite}
To ensure the reproducibility of incompatibility issues and to
facilitate the demonstration of such issues, we have annotated
checkpoints in the version histories of the client projects and
provided tags that guide users to explore any incompatibility issues
present in the \toolcompsuite dataset.

As illustrated on the right-hand side of \cref{fig:architecture}, our
approach to handling incompatible client-library pairs involved
creating a fork~\cite{gitfork} of the original client project for each
identified pair, while preserving all code and version history
information. To mark the base version of the project, we utilized the
\Code{git tag}~\cite{gittag} command, designating it as $V_{base}$. Subsequently, we developed a
patch to upgrade the library from its old version to its new version,
a simple process that can be accomplished with a single line change in
the \Code{pom.xml} file for Maven projects. This patch was then
applied to the $V_{base}$ version, resulting in a new version that we
identified as $V_{incomp}$. Notably, the only difference between
$V_{base}$ and $V_{incomp}$ lies in the library version used: the old
(compatible) version is utilized on $V_{base}$ while the new
(incompatible) version is utilized on $V_{incomp}$. For instance, in
\cref{fig:architecture}, the client project employs version 2.2.1 of
the \Code{org.restlet.jse-org.restlet} library on its $V_{base}$ and
version 3.0-M1 on its $V_{incomp}$. In cases where multiple libraries
exhibit incompatibility issues in the client project, we not only create different branches for each library with its name, but also generate a
$V_{incomp}$ version tag for each, with accompanying annotations that
denote the corresponding library name and version, as depicted in
\cref{fig:architecture}.


The $V_{incomp}$ tag for each client-library pair also specifies the
specific test that can reveal the incompatibility issue during its
run. Following Maven's convention, the test name is formatted as
\Code{TestClassName\#testMethodName}. By simply copying the text from
the tag, users can easily run the incompatibility-revealing test on
the $V_{incomp}$ version and observe the incompatibility issue. On the
$V_{base}$ version, all tests should pass. This design aims to
simplify the usage of \toolcompsuite and make it more accessible and
user-friendly.


Using the forked client repositories and version tags provided in the
\toolcompsuite dataset, users can easily reproduce any incompatibility issue by
checking out to $V_{incomp}$ and running the corresponding
incompatibility-revealing test. To compare the behaviors of the client
with compatible and incompatible library versions, users can run the
incompatibility-revealing test on both $V_{base}$ and $V_{incomp}$ and
compare the test outcomes. This allows for a clear understanding of
the impact of the library upgrade on the client behaviors.


\subsection{\runner: An Automated Tool for Reproducing Incompatibility Issues}

We further developed an automated tool, named \runner, which is a part of
\toolcompsuite. With \runner, users can easily reproduce and investigate any
incompatibility issue in a one-click manner by providing the issue ID
as input.

We offer an option which enables users to reproduce an
incompatibility issue end-to-end with a single command as is shown below.
The command outputs and saves all intermediate results and logs for future
reference.

\begin{Verbatim}[fontsize=\small,breaklines,linenos,xleftmargin=8pt,numbersep=5pt,commandchars=\\\{\}]
python\PYG{+w}{ }main.py\PYG{+w}{ }\PYGZhy{}\PYGZhy{}incompat\PYG{+w}{ }i\PYGZhy{}56
\end{Verbatim}

When \runner runs, it clones the client project from our forked
code repository and saves it in the output directory (which is configurable).
Then, it checks out to the base version, compiles the code, and runs the
incompatibility-revealing test. Next, it upgrades the library to the
new version, reruns the incompatibility-revealing test, and reports
any failure information to the user.

We also provide a set of commands that break down the entire cycle of incompatibility
exploration into separate steps:

\begin{Verbatim}[fontsize=\small,breaklines,linenos,xleftmargin=8pt,numbersep=5pt,commandchars=\\\{\}]
python main.py \PYGZhy{}\PYGZhy{}download i\PYGZhy{}56
python main.py \PYGZhy{}\PYGZhy{}compile i\PYGZhy{}56
python main.py \PYGZhy{}\PYGZhy{}testold i\PYGZhy{}56
python main.py \PYGZhy{}\PYGZhy{}testnew i\PYGZhy{}56
\end{Verbatim}


We provide several other \runner commands for users to inspect
different aspects of the incompatibility issues from the \toolcompsuite dataset. A
complete list of these commands can be found on \toolcompsuite's website at
\url{https://github.com/compsuite-team/compsuite}.



\section{Application Scenarios}\label{sec:app-compsuite}

We anticipate that both researchers and practitioners can benefit from
\toolcompsuite to facilitate their investigations and research on errors
and test failures induced by library upgrades. \toolcompsuite supports the evaluation of
various program analysis techniques, such as software upgrade compatibility checking, debugging,
and module-level regression test selection techniques.


As an overview, authors of compatibility checkers and detectors
may use \toolcompsuite as a benchmark to evaluate the performance of their
techniques against other baseline approaches. Furthermore, authors of
debugging techniques can utilize \toolcompsuite as a dataset of compatibility
bugs, where each bug corresponds to a test case that verifies the
existence or absence of the bug. Finally, authors of module-level
regression test selection techniques can use \toolcompsuite to assess the
safety of their approaches. A safe module-level RTS technique should
select all the corresponding incompatibility-revealing test cases when
the library changes.



We detail the three usage scenarios as follows.
\begin{itemize}[leftmargin=*,topsep=2pt]
    \item \textbf{Compatibility Checkers and Detectors.}
      The existing techniques for compatibility checking and detection
      in Java can be categorized into three groups: i) Techniques for
      detecting API incompatibility that focus on detecting
      API-breaking changes, such as renaming of code entities and
      changes in parameter types~\cite{li2018cid,he2018understanding,scalabrino2019data,wei2019pivot,huang2018understanding}. ii)
      Techniques for detecting behavioral incompatibility that focus
      on identifying behavioral differences that cause test failures
      when a library is upgraded in a client, such as changes in
      program states~\cite{zhuclient,mora2018client}. iii) Techniques for
      detecting dependency conflicts~\cite{wang2021will,wang2018dependency,wang2019could},
      which aim to identify library APIs that exhibit inconsistent
      semantics between libraries due to class path shading. We
      believe that developers of techniques in the first two
      categories can use \toolcompsuite as a benchmark to evaluate their tools'
      performance, such as precision and recall. They can run their
      tools on the \toolcompsuite dataset and compare the results with the
      incompatibility issues present in the dataset. On the other
      hand, developers of techniques for detecting dependency
      conflicts can slightly modify \toolcompsuite's dataset by placing both
      old and new libraries on the class path, running library
      conflict detection, and checking if the issues can be detected.

    \item \textbf{Module-Level Regression Test Selection.}  Regression
      test selection (RTS) is a technique that aims to reduce the cost
      of regression testing by selecting a subset of tests that may
      change the behavior due to code changes on each program
      version~\cite{gligoric2015practical,legunsen2016extensive,zhang2018hybrid,zhu2019framework}. Module-level
      RTS focuses on selecting the affected client tests when a
      dependent library is updated~\cite{gyori2018evaluating}. The developers of
      module-level RTS techniques can evaluate the safety of their
      tools using \toolcompsuite. For each client-library pair in \toolcompsuite, a
      module-level RTS tool should select all the corresponding
      incompatibility-revealing tests when upgrading the library from
      the old version to the new version.

    \item \textbf{Debugging.}  The existing debugging techniques for
      Java, include symbolic execution~\cite{baldoni2018survey}, delta
      debugging~\cite{zeller1999yesterday}, fault localization~\cite{wong2016survey}, etc. These techniques aim to identify the root
      cause of errors or failures in software. Developers of debugging
      techniques can use \toolcompsuite as a dataset of compatibility bugs,
      where each compatibility bug corresponds to a test case that
      checks the presence or absence of the bug. They can use \toolcompsuite to
      evaluate their techniques' ability to perform root cause
      analysis by trying to identify the corresponding library change
      that caused the compatibility issue.
\end{itemize}

\section{related work}\label{sec:related-compsuite}

To cater to the requirements of various research endeavors, numerous outstanding datasets have been
made available to date. Just et al.~\cite{just2014defects4j} introduced Defects4J, a database
supplies actual bugs, fixed program versions, and corresponding test suites. Bui et
al.~\cite{bui2022vul4j} introduced Vul4J focusing on Java vulnerabilities. Jezek et al.
\cite{jezek2017api} released their dataset of compatibility issues arising from program evolution.
There are also many datasets cater for other research domains and
ecosystems~\cite{zhu2017dataset,wei2016taming,nielebock2021androidcompass}. 


Distinct from the previously discussed datasets, \toolcompsuite is the first dataset emphasizes the
incompatibility issues caused by Java library behavior changes. This type of issues are prevalent
and difficult to detect. Additionally, Our developed automated tools also have the capability to assist researchers in swiftly reproducing issues. We believe that a dataset targeting the library upgrade incompatibility
issue will contribute to the advancement of the associated technologies.
%!TEX root=../mythesis.tex
% Chapter Template

\chapter{Identifying Multi-parameter Constraint Errors in Documentation} % Main chapter title
\chaptermark{MPChecker}  % replace the chapter name with its abbreviated form
\label{ch:chapter4}

\section{Introduction}\label{sec:intro}

Machine learning (ML) and Artificial Intelligence (AI) have consistently garnered widespread
attention, achieving remarkable breakthroughs in diverse domains including natural language
processing, recommendation systems, autonomous vehicles, and robotics.
Behind the rapid advancement of these transformative technologies, data science and machine
learning libraries play a crucial role in AI and ML development.
By providing extensive APIs for complex mathematical operations and algorithmic implementations,
these libraries enable researchers and practitioners to focus on solving domain-specific problems
rather than reimplementing fundamental algorithms.

A well-designed API documentation not only provides detailed descriptions of interfaces,
including the purpose and range of parameters or attributes, returns, and exceptions thrown, but may also specify logical constraints or dependencies among multiple parameters.
For data science and machine learning libraries, multi-parameter constraints are commonly mentioned
in their API documentation and users of these libraries are expected to follow them closely when
using the APIs.
However, frequent version updates may lead to the documentation out of sync with the corresponding
code, known as the \underline{\textbf{C}}ode-\underline{\textbf{D}}ocumentation
\underline{\textbf{I}}nconsistency (CDI) issue~\cite{aghajani2019software, rai2022review}.
Such CDI issues are particularly pronounced in data science libraries.
On one hand, the underlying mathematical models of DS/ML libraries inherently come with various
constraints, such as \emph{``a model X can only be chosen when a parameter Y is provided''}, and
incorrect parameter configurations not satisfying their constraints may lead to unexpected outcomes.
On the other hand, the number of parameters/attributes of these libraries can be significantly more
than a typical library API, sometimes over a few dozen.
Therefore, it is unrealistic to track all parameter constraints manually.

Detecting errors in multi-parameter constraints from Python API documentation is challenging for
several reasons.
(1) The quality of API documentation varies and lacks standardized writing guidelines.
Some API documentation uses ambiguous language, contains typos, and may not follow a consistent
styling guide.
This makes simple rule-based pattern-matching approaches ineffective.
(2) Existing approaches~\cite{ratol2017detecting, liu2020automating} for detecting documentation errors focus on a single parameter
only: e.g., checking whether the information provided on parameter ranges, nullness, and identifier
names is correct.
It is more challenging to extract multi-parameter constraints precisely from free-style descriptions
written in natural languages.
(3) For the same reason, a semantic-aware code analysis approach is essential, as logical relations among
multiple parameters cannot be easily identified through purely syntactic analysis. The challenge is further compounded by Python's dynamic nature, where variable types, attributes, and behaviors can change at runtime.



To detect multi-parameter constraint inconsistencies from data science library documentation, we
propose an automated tool \tool.
\tool identifies inconsistencies between API documentation and the corresponding library code by
combining symbolic execution-based program analysis techniques with constraint extraction methods
powered by large language models (LLMs).
We first extract multi-parameter constraints from documentation (a.k.a. \emph{doc-constraints}),
leveraging the powerful natural language understanding capability of LLMs.
We incorporate a few optimizations, such as Chain of Thought (CoT)~\cite{wei2022chain} and few-shot
learning to improve the accuracy of constraint extraction.
Then we use dynamic symbolic execution to collect all path constraints from the corresponding
Python source code.
The symbolic path constraints (a.k.a. \emph{code-constraints}) capture the real constraints that
the parameters have to follow according to the library code, which are then used to evaluate the
correctness of the \emph{doc-constraints}.


Then, in order to mitigate minor discrepancies that may arise from the \emph{doc-constraints} extracted by
LLMs, we design and implement a \emph{Fuzzy Constraint Logic} (FCL) framework to estimate how
logically consistent a \emph{doc-constraint} is with a set of given \emph{code-constraints}.
Intuitively, in the absence of LLM-induced unpredictability, a \emph{doc-constraint} must be evaluated as true
under the assumption of \emph{code-constraints}.
Through fuzzy constraint satisfaction, we can accommodate many \emph{nearly-correct} constraints
produced by LLMs and thus improve the accuracy of the overall approach.


\paragraph{Contributions}
Our work aims to integrate precise symbolic reasoning with the inherently fuzzy outputs of large language models.
To summarize, we make the following contributions.
\begin{enumerate}
    \item We proposed an automated multi-parameter code-documentation inconsistency detection technique and developed an end-to-end command-line tool called \tool. Existing techniques in the same area are only designed to handle single parameter inconsistencies, without considering inter-parameter constraints.
    \item We introduced a customized fuzzy constraint satisfaction framework to mitigate the
    uncertainties introduced by LLM outputs. We provide a theoretical derivation of the membership
    function based on constraint similarity.
    \item We constructed a documentation constraint dataset comprising 72 real-world constraints sourced from widely used data science libraries, and derived a mutation-based inconsistency dataset with 216 constraints.
    Our dataset and tool implementation are made available online: \url{https://github.com/ParsifalXu/MPChecker}.
    \item We evaluated our tool on four real-world popular data science libraries. We reported 14 inconsistency issues discovered by \tool to the developers, who have confirmed 11 inconsistencies at the time of writing.
\end{enumerate}
% \section{Motivating Example}\label{sec:bg-mpchecker}
% In this section, we review the essential terminology and background necessary for understanding the remainder of the paper.
\section{Multi-Parameter Constraints}\label{sec:eg-mpchecker}
We use two examples to illustrate inconsistencies between API documentation and the corresponding code
caused by multi-parameter interdependence.
Both of them come from open-source Python data science libraries and were successfully
detected by \toolchecker.
In general, there are two types of constraints found in the API documentation.
(1) An \emph{explicit constraint} clearly specifies the logical relationship among two or more
interrelated parameters.
(2) An \emph{implicit constraint} is an unstated or indirectly implied relationship among two or
more interrelated parameters, where the constraint is inferred through contexts or convention
rather than explicitly specified.


\subsubsection{Example 1: Explicit Constraint}\label{sec:eg1-mpchecker}

The first example comes from \texttt{statsmodels}~\cite{github:statsmodels}, which provides a
complement to \texttt{scipy} for statistical computations including descriptive statistics and
estimation and inference for statistical models.
\texttt{Statsmodels} has more than 10K stars on GitHub and is actively maintained.
\cref{fig:eg1} illustrates an inconsistency caused by an explicit constraint from the class
\textit{AutoReg}.
The relevant portions for the \emph{doc-} and \emph{code-constraints} are highlighted.
As mentioned in the documentation of \texttt{deterministic}, the trigger condition for the warning
is that ``trend is not n, \textbf{and} seasonal is not False''.
However, it is apparent that the \emph{code-constraint} for \texttt{trend} and \texttt{seasonal} (to be
used together correctly and avoid any warning) implemented is \textbf{or} instead of \textbf{and}.
One way to fix the documentation is to change ``and'' to ``or''.


\begin{figure*}[t]
	\vspace{2mm}
    \begin{subfigure}{\linewidth}
        \begin{tcolorbox}[colback=Emerald!10,colframe=cyan!40!black,title=\textbf{Constraint description of \texttt{trend} and \texttt{seasonal} in class \texttt{AutoReg}}]
            {\sffamily \textbf{> deterministic: } DeterministicProcess
            \\
            A deterministic process. If provided, trend and seasonal are ignored. \colorbox{blue!20}{A warning is raised if} \colorbox{blue!20}{trend is not "n" and seasonal is not False.}}
        \end{tcolorbox}
        \label{fig:eg1-doc}
    \end{subfigure}
    \begin{subfigure}{\linewidth}
\begin{tcolorbox}[colback=Salmon!20, colframe=Salmon!90!Black,title=\textbf{Corresponding code snippet in class \texttt{AutoReg}}]
\begin{lstlisting}[escapechar=@]
class AutoReg(tsa_model.TimeSeriesModel):
    def __init__(...):
        if deterministic is not None and @\colorbox{blue!20}{(self.trend != "n" or self.seasonal)}@:
            warnings.warn('When using deterministic, trend must be "n"
                and seasonal must be False.', SpecificationWarning, stacklevel=2)
\end{lstlisting}
\end{tcolorbox}
        \label{fig:eg1-code}
    \end{subfigure}
    \caption{Examples of an explicit constraint from \Code{Statsmodels}.}
    \label{fig:eg1}
    \vspace{-5pt}
\end{figure*}



\subsubsection{Example 2: Implicit Constraint}\label{sec:eg2-mpchecker}

The second example comes from \texttt{scikit-learn}~\cite{github:scikit}, which is a widely-used
(more than 60K stars on GitHub) open-source ML library in Python, designed to offer simple
and efficient tools for data mining and data analysis.
\ref{fig:eg2} displays an inconsistency caused by an implicit constraint from the class
\textit{SpectralClustering}.
It is evident that the highlighted part of the documentation only explicitly mentions one parameter
\textit{affinity}, omitting the subject ``gamma''.
More importantly, ``ignore'' is not a specific identifier or value but rather a description of
the program logic---if the parameter \textit{affinity} is set to \textit{nearest\_neighbors}, then
the parameter \textit{gamma} will not be used.
Whereas, above constraint does not faithfully reflect the behavior implemented in code.
According to the code snippet, ``gamma'' is not only ignored within the \textit{nearest\_neighbors}
branch, but also ignored within the \textit{precomputed\_nearest\_neighbors} and
\textit{precomputed} branches.
This indicates that the constraint is inaccurate and demonstrates a form of inconsistency.

For this type of implicit constraint, traditional pattern-based approaches are not able to extract
the \emph{doc-constraint} correctly, thus fail to detect the inconsistencies.
To solve this issue, we design a customized constraint that incorporates fuzzy words, and adopt
few-shot learning to teach LLMs how to generate such constraints (details in
\cref{sec:llm-mpchecker}).
In this case, the \emph{doc-constraint} should be \textit{``(affinity = "nearest\_neighbors") $\rightarrow$
(ignore(gamma))''}, where a special predicate \textit{``ignore(x)''} is used to indicate that a parameter
\textit{x} is ignored (see \cref{sec:fuzzword-mpchecker}).


\begin{figure*}[t]
	\vspace{2mm}
    \begin{subfigure}{\linewidth}
        \begin{tcolorbox}[colback=Emerald!10,colframe=cyan!40!black,title=\textbf{Constraint description of \texttt{gamma} and \texttt{affinity} in class \texttt{SpectralClustering}}]
            {\sffamily \textbf{> gamma} : float, default=10
            \\
            Kernel coefficient for rbf, poly, sigmoid, laplacian and chi2 kernels. \colorbox{blue!20}{\textbf{Ignored for}} \colorbox{blue!20}{\textbf{affinity="nearest\_neighbors".}}}
        \end{tcolorbox}
        \label{fig:eg2-doc}
    \end{subfigure}
    \begin{subfigure}{\linewidth}
\begin{tcolorbox}[colback=Salmon!20, colframe=Salmon!90!Black,title=\textbf{Corresponding code snippet in class \texttt{SpectralClustering}}]
\begin{lstlisting}[escapechar=@]
class SpectralClustering(ClusterMixin, BaseEstimator):
    def fit(self, X, y=None):
        if @\colorbox{blue!20}{self.affinity == "nearest\_neighbors"}@:
            ...
        elif @\colorbox{blue!20}{self.affinity == "precomputed\_nearest\_neighbors"}@:
            ...
        elif @\colorbox{blue!20}{self.affinity == "precomputed"}@:
            ...
        else:
            params = self.kernel_params
            if params is None:
                params = {}
            if not callable(self.affinity):
            @\colorbox{blue!20}{params["gamma"] = self.gamma}@
                params["degree"] = self.degree
                params["coef0"] = self.coef0
\end{lstlisting}
\end{tcolorbox}
        \label{fig:eg2-code}
    \end{subfigure}
    \caption{Examples of implicit constraint from \texttt{Scikit-learn}.}
    \label{fig:eg2}
    \vspace{-5pt}
\end{figure*}



% \subsection{Fuzzy Logic and Fuzzy Constraint Satisfaction}
% Unlike traditional Boolean logic, fuzzy logic~\cite{kosko1993fuzzy} is a multi-valued logic that
% allows for values between 0 and 1 to represent varying degrees of truth, where 0 represents
% absolute false, and 1 represents absolute true.
% The human brain can process vague statements or claims that involve uncertainties or subjective
% judgments, such as ``the weather is hot'', ``that man runs so fast'', or ``she is beautiful''.
% Unlike computers, humans possess common sense, allowing them to reason effectively in situations
% where things are only partially true.
% Fuzzy logic is primarily used to model uncertainty and vagueness, making it highly applicable in
% real-world scenarios where precision may be difficult or impossible to achieve.

% A traditional constraint satisfaction problem (CSP)~\cite{brailsford1999constraint} requires all constraints to be fully
% satisfied.
% Constraints are either completely satisfied or unsatisfied, which is why these strict,
% non-fuzzy constraints are referred to as ``\emph{crisp constraints}''.
% An extension of CSP, known as soft CSP~\cite{meseguer2006soft, schiex1992possibilistic}, introduces a distinction between hard
% constraints and soft constraints.
% Hard constraints must be absolutely satisfied, while soft constraints are typically assigned a
% weight or priority, allowing for lower-weighted constraints to be only partially satisfied or even
% unsatisfied under certain conditions during problem-solving.
% Another extension is Fuzzy CSP~\cite{ruttkay1994fuzzy}, which differs from soft constraints in that it
% incorporates fuzzy logic and allows each constraint to be ``partially satisfied'' to a degree,
% quantified by a ``satisfaction degree''.
% This satisfaction degree usually ranges from 0 to 1, indicating the extent to which a constraint is
% fulfilled.
% The goal in fuzzy constraint satisfaction is to find a solution that maximizes satisfaction, rather
% than strictly satisfying all constraints.

\section{Methodology}\label{sec:method-mpchecker}


\begin{figure*}[h]
	\vspace{2mm}
    \centering
    \includegraphics[width=\linewidth]{Figures/Chapter4/arch.pdf}
    \caption{The architectural overview of \toolchecker.}
    \label{fig:arch}
    \vspace{-5pt}
\end{figure*}

In this section, we define the issue of code-documentation inconsistency caused by multi-parameter
constraints and provide a detailed description of our approach. An API documentation error is an inconsistency between the library source code and its API documentation. Multi-parameter constraints refer to conditional dependency relationships that exist among multiple parameters within functions or classes. If a constraint is never violated across all execution paths in the code, it is considered as a benign constraint, or it indicates a potential documentation error. According to literature~\cite{uddin2015api,zhou2017analyzing, zhu2022identifying}, API documentation inconsistency can be categorized into two types: incorrectness and incompleteness. Incorrectness refers to cases where the documentation describes behavior that is not implemented in the code, while incompleteness arises when certain code behaviors are not reflected in the documentation. Typically, incorrectness issues are considered more critical than incompleteness.

In addition, when it comes to constraint extraction, compared to single-parameter constraints, we need to classify the multi-parameter constraint extraction problem into two types, as discussed in Section~\ref{sec:eg-mpchecker}, 1) explicit constraint and 2) implicit constraint.

\toolchecker aims to accurately extract multi-parameter constraints from API documentation and detect both types of inconsistency. As the architecture depicted in Figure~\ref{fig:arch}, we have designed a three-phase workflow comprising the \textbf{1) Data Preprocessing; 2) Constraint Extraction; 3) Inconsistency Detection}. During the preprocessing phase, we separate the code and documentation within the project.
\toolchecker will then automatically rewrite each function to be compatible with the symbolic
execution tool. This includes replacing advanced Python syntax that the tool cannot handle with
simpler constructs and symbolizing external function calls, such as replacing ternary conditional
expressions with if-else statements. These modifications do not alter the path constraints of the
original program.
In the constraint extraction and expression generation phase, on the one hand, we leverage large language models to extract constraints in a specific format from the documentation. On the other hand, the symbolic execution tool dynamically analyzes the code and solves the constraint paths. Those constraints are then converted into expressions that can be processed by the SMT solver. In the fuzzy constraint checking phase (Phase III), a constraint checker with SMT solver and fuzzy constraint reasoner performs comprehensive reasoning to detect inconsistencies. It is worth noting that we propose and implement an extended fuzzy constraint satisfaction to mitigate the hallucination issues often introduced by large language models, and reduce the risk of false positives and missed detections.




\subsection{Preprocessing}\label{sec:preprocess-mpchecker}
In this step, we will discuss the details of separating the documentation and corresponding code from the project and the specifics of preprocessing the documentation content.

In modern data science libraries, documentation is typically auto-generated using Sphinx, a tool that can automatically create HTML documentation from Python code. Sphinx supports various docstring styles, with Google style and NumPy style being commonly used. \fref{fig:docstring} respectively display docstring examples of two different styles from Sphinx official website~\cite{sphnix:google,sphnix:numpy}. Google-style docstrings use a clear and concise format with a minimalistic structure. It divides the docstring into sections like \textbf{Args}, \textbf{Attributes}, etc., with each section using plain indentation. Similarly, Numpy-style docstrings organize sections more rigidly. Sections are divided by using \textbf{Parameters}, \textbf{Attributes}, etc. with horizontal dash lines ``\textbf{- - -}'' under the section header. The number of dashes is the same as the number of letters in the section header. Regardless of the style used, the docstring is normally placed at the beginning inside its corresponding class or function.

After downloading the project, our tool first converts every Python file from the project into an Abstract Syntax Tree (AST) and isolates the classes and independent functions. This paper focuses on the CDI issue, so in this step, we filter out code without documentation and separately extract the code and documentation from the remaining code. Since Python supports object-oriented programming but current symbolic execution tools have limited support for classes, we have to limit our experimental units to functions. For independent functions, the scope of constraints in the documentation usually applies within the function itself. For classes, however, the constraints cover the entire class, including each member function. Therefore, we create a new directory for every class and independent function, with member function directories placed within their corresponding class directories to maintain structural consistency. If the member function has its own documentation, it will also be retained. 

To help the LLM better focus on the constraints between parameters and reduce the occurrence of erroneous constraints, we retained parameters or attributes and their corresponding descriptions in the form of key-value pairs, based on the two aforementioned docstring styles. We subsequently applied a rule-based heuristic approach to retain documentation that potentially contain constraints and to discard the rest. For instance, if none of the parameters or attributes appear in other descriptions, this indicates the absence of multi-parameter constraints in that documentation.


\begin{figure*}[t]
	\vspace{2mm}
    \begin{subfigure}{0.49\linewidth}
        \begin{tcolorbox}[colback=Emerald!10,colframe=cyan!40!black,title=\textbf{Numpy Style Docstrings}]
\begin{lstlisting}[escapechar=@]
class ExampleNumpyStyle():
  """Exceptions are documented
  @\colorbox{blue!20}{Parameters}@
  ----------
  msg : str
    Human readable string describing the exception.
  code : obj:`int`, optional 
    Numeric error code.
  @\colorbox{blue!20}{Attributes}@
  ----------
  msg : str
    Human readable string describing the exception.
  code : int
    Numeric error code.
  """
  def __init__(self, msg, code):
    self.msg = msg
    self.code = code
\end{lstlisting}
        \end{tcolorbox}
        \label{fig:numpy}
    \end{subfigure}
    \begin{subfigure}{0.49\linewidth}
\begin{tcolorbox}[colback=Salmon!20, colframe=Salmon!90!Black,title=\textbf{Google Style Docstrings}]
\begin{lstlisting}[escapechar=@]
class ExampleGoogleStyle():
  """Exceptions are documented
  Note:
    Do not include the `self` parameter in the ``Args`` section.
        
  @\colorbox{blue!20}{Args:}@
    msg (str): Human readable string describing the exception.
    code (:obj:`int`, optional): Error code.

  @\colorbox{blue!20}{Attributes:}@    
    msg (str): Human readable string describing the exception.
    code (int): Exception error code
  """
  def __init__(self, msg, code):
    self.msg = msg
    self.code = code
\end{lstlisting}
\end{tcolorbox}
    \label{fig:google}
    \end{subfigure}
    \caption{Example of two docstring styles}
    \label{fig:docstring}
    \vspace{-10pt}
\end{figure*}



\subsection{Constraint Extraction}\label{sec:extraction}

We now specially explain how to extract path constraints from code and convert them into expressions which are solvable by SMT solver, as well as how to use LLM to extract constraints from documentation and transform them into expressions containing fuzzy words.

\subsubsection{Code Constraint Expression Extraction}

The goal of \tool is to verify whether the constraints between multiple parameters in documentation align with the logic during actual code execution. This requires our tool to understand and analyze deeper constraint relationships. Therefore, we employ symbolic execution to capture as many path conditions as possible and precisely handle complex paths and constraints.

We modified current advanced dynamic symbolic execution tools~\cite{github:pyexsmt, github:pyexz3, github:pysmt, ball2015deconstructing, bruni2011peer} for path exploration. Unfortunately, supporting dynamic languages like Python is more challenging compared to symbolic execution tools designed for static languages such as Java and C. Despite Python's rapid evolution, symbolic execution tools specifically designed for Python have developed slowly, struggling to keep pace with the growing new syntax and features. This forces us to make reasonable modifications to the source code extracted directly from repositories. However, these modifications must not alter the path constraints of the original code; they should be equivalent code transformations that do not affect path exploration. We mainly made the following modifications:


\begin{enumerate}
    \item Current Python symbol execution tool can not solve class directly. Therefore, it is necessary to split the class into functions (i.e. member functions). The corresponding member variables also need to be changed and used as symbolic inputs.
    \item Replace complex structures and operations, such as lists and dictionaries, that are difficult to handle and do not affect the path, as well as external function calls that may cause path explosion, with symbolic inputs.
    \item Replace the handling of exceptions and warnings that do not affect the path with \textit{return}.
    \item Add a fixed format of \textit{return} statement to capture concrete values of potential symbols.
    \item Equivalent code implementation replacement to avoid being unable to find useful path constraints due to poor support for some advanced syntax. For example, replace ternary operator to conventional if-else statement.
\end{enumerate}


In the limit, \tool strives to explore all feasible paths in a Python function by following these processes: 1) Running the function with specific input to trace a path through the control flow of the function; 2) Symbolic executing the path to determine how its conditions depend on the function's input parameters; 3) Utilizing Z3 to generate new parameter values that guide the function toward paths that haven't been covered yet.

Although \tool supports a certain level of external function call analysis, in complex real-world code, an external function call often corresponds to extra more function calls, leading to path explosion. Furthermore, documentation constraints are usually handled within the target function, so we still prefer not to introduce external function calls and to focus the analysis within the target function. Additionally, similar to the current concolic symbolic execution tools for Python, \tool does not yet provide strong support for theorem of strings. Thus, during the actual execution process, we replace the string with a unique large number, which does not affect the exploration of condition constraints.


\begin{figure*}[t]
	\vspace{2mm}
    \begin{subfigure}{0.49\linewidth}
        \begin{tcolorbox}[colback=Salmon!20, colframe=Salmon!90!Black,title=\textbf{Original source code}]
\begin{lstlisting}[escapechar=@]
def fit(self, sample_weight):
  if sample_weight is not None and self.strategy == "uniform":
    raise ValueError("Warning Info")
  if sample_weight is not None:
    sample_weight = _check_sample_weight(sample_weight, X)
\end{lstlisting}
        \end{tcolorbox}
        \label{fig:sourcecode}
    \end{subfigure}
    \begin{subfigure}{0.49\linewidth}
        \includegraphics[width=\linewidth]{Figures/Chapter4/pathgraphpic.pdf}
        \vspace{-8pt}
        \label{fig:pathgraph}
    \end{subfigure}
    \begin{subfigure}{0.98\linewidth}
\begin{tcolorbox}[colback=OliveGreen!10,colframe=Green!70,title=\textbf{Modified source code}]
\begin{lstlisting}[escapechar=@,basicstyle={\footnotesize\ttfamily}]
def fit(sample_weight, strategy, call__check_sample_weight):
  if sample_weight != 'None' and strategy == 'uniform':
    return '(sample_weight)_(strategy)_ERROR_END'
  if sample_weight != 'None':
    sample_weight = call__check_sample_weight
  return (f'(sample_weight = {sample_weight}) ^ (call__check_sample_weight =                         {call__check_sample_weight}) ^ (strategy = {strategy})')
\end{lstlisting}
\end{tcolorbox}
        \label{fig:modifiedcode}
    \end{subfigure}
    \caption{Extracting constraint from code}
    \label{fig:extractpath}
    \vspace{-5pt}
\end{figure*}

We will use an example depicted in Figure~\ref{fig:extractpath} to illustrate the entire extraction phase, containing a simplified original source code and modified code from a popular data science project \texttt{scikit-learn}, and its corresponding path constraints. The \textit{fit} function is a member function within a class, and thus member variables such as ``self.strategy'' also exist within the code. We also modified ``None'' as a string to make it easier to be captured, since it is represented as a number 0 during symbolic execution. Our tool first modifies the code and replaces exception handling and external function calls with symbolic inputs, marked as ``ERROR\_END'' and ``call\_'', respectively. For those paths whose final states are ``ERROR\_END'', the final results of the conjunction of the documentation constraint and these paths will be negated during reasoning phase.



\subsubsection{Documentation Constraint Expression Extraction}\label{sec:llm}

In this step, we extract constraints from Python documentation by applying LLMs.
Since Python documentation can vary in quality and may contain informal writing~\cite{rani2021comments}, the important task is to understand the parameter information within the documentation. To achieve this, we resort to SOTA LLMs. Given Python documentation as input, the LLM is asked to first extract constraint-related sentences and then output them in a standard logical expression format. This includes two steps, model selection and prompt design.


\paragraph{Model Selection} We adopt GPT-4, which is pretrained on a diverse corpus and shows
excellent performance in natural language understanding. Based on our preliminary study, GPT-4's
performance stands out compared to Gemini-1.5~\cite{gemini} and LLaMA-3~\cite{llama3} due to its
ability to capture details, and it is also well-acquainted with the context of code
documentation~\cite{dvivedi2024comparative}.


\paragraph{Prompt Design} Because the constraint extraction task is relatively complex and can be
broken down into clear steps, we apply the chain-of-thought approach~\cite{wei2024cot}, which has
been widely proven effective in improving GPT-based model performance. We first divide the prompt
task into two steps, document input and constraint extraction. Figure~\ref{fig:prompt} shows the
structure and some details of the used prompt.
Below, we detail our prompt mechanism for each step.

\paragraph{Document Input Prompt}
We observe that some documentation may be too lengthy to provide to GPT-4 in a single input, considering that GPT-4 has a maximum token length limit of 8,192 tokens~\cite{modeltoken}. We also find that LLMs exhibit lower performance when dealing with long and complex text inputs as noted in previous research~\cite{han2024lm, jin2024llm}. Thus, we decide to segment the lengthy documents into smaller sections. To determine a heuristic chunk size, we randomly select ten lengthy Python documentation files, split them into chunks of varying word lengths, and use these as inputs for GPT-4. We then evaluate the constraint extraction task performance of GPT-4 based on these inputs to determine which chunk size yields better results. Based on our findings, we decide to standardize the chunk size to 1,500 words (around 2,048 tokens~\cite{tokencount}).
We also input the parameter list obtained in Section~\ref{sec:preprocess} into GPT-4 to help model better recognize the information related to parameters. The details of the document input prompt are shown in Prompt 1 in Figure~\ref{fig:prompt}.

\paragraph{Constraint Extraction Prompt}
For the constraint extraction task, our prompt is divided into three parts to guide GPT-4 in recognizing text related to constraints in the original documentation, and then, based on that text, to generate a formatted logical expression of the constraint.

The first part involves defining the logical symbols that can be used in the logical format, including implication, negation NOT, logical AND, logical OR, and also defining parentheses to indicate the precedence of logical expressions.

The second part arises from our preliminary study, in which we observed that some Python documentation uses vague terms such as ``override'', ``specify'', ``have an effect'', ``no effect'', ``significant'', and ``ignore'' when mentioning constraints related to parameters. To preserve as much detail as possible from the documentation, we design prompts to guide GPT-4 so that if text related to parameter constraints contains vague keywords, these keywords should be retained in the final logical expression.

In the third part, to ensure that the format of the logical expression in GPT-4's output is consistent each time and convenient to process, we apply in-context learning techniques that widely used in previous works~\cite{min2022rethinking, rubin2021learning} to enable GPT-based models to handle tasks specific to a domain.
We include four examples that contain pairs of original constraint-related sentences selected from Python documentation and their corresponding logical expression constraints.


\begin{figure*}[h]
    \centering
    \includegraphics[width=\linewidth]{Figures/Chapter4/prompt_template.pdf}
    \caption{Prompt structure for constraints extraction}
    \label{fig:prompt}
    \vspace{-5pt}
\end{figure*}
\subsection{Inconsistency Detection}\label{sec:detect}

In the second phase, we extracted constraints from both documentation and code. While \emph{code-constraints} are deterministic in nature, \emph{doc-constraints} inherently contain uncertainties stemming from two main sources. First, there are \emph{implicit constraints} arising from vague or incomplete descriptions, which we address by defining fuzzy words to extend them into soft constraints. Second, we encounter uncertainties introduced by generative models' limited reasoning capabilities and unavoidable hallucination issues, for which no validator exists to definitively determine the correctness of generated constraints. To address this challenging scenario, we proposed a customized fuzzy constraint logic to mitigate such vagueness. With the help of fuzzy words and fuzzy constraint logic, our converter can effectively handle both explicit and implicit constraints. The converter ultimately produces fuzzy expressions, which are then processed by a reasoner based on the z3 SMT (Satisfiability Modulo Theories) solver to detect inconsistencies.


The reasoner offers two strategies, from relaxed to strict, to identify inconsistencies from the perspectives of satisfiability and equivalence. Given a \emph{doc-constraint} $c$ and a \emph{code-constraint} set $P$ containing a group of path constraints $p$, the detection strategies are defined as follows:
\begin{itemize}
	\item \textbf{Unsatisfiability} checking determines whether the \emph{doc-constraint} $c$ is unsatisfiable under all path constraints. If the conjunction of $c$ and every path constraint $p$ is unsatisfiable, it indicates a contradictory inconsistency, meaning that under all possible execution paths, the code violates the constraint stipulated in the documentation.
	\begin{equation}\label{eq: uq1}
		\forall p \in P, \neg (c \wedge p)
	\end{equation}
	\item \textbf{Nonequivalence} checking determines whether the \emph{doc-constraint} $c$ and \emph{code-constraints} are logically equivalent. If equivalence holds only under specific conditions, a behavioral inconsistency may arise, implying that the constraints implemented in the code are not fully equivalent to the documentation.
	\begin{equation}\label{eq: uq3}
		\exists p \in P, \neg (c \Leftrightarrow p)
	\end{equation}
\end{itemize}

For constraints containing sub-constraints that do not exist in the code logic, we employ a heuristic approach to provide suggestions. Specifically, when a constraint mentioned in the documentation is not present in the corresponding code logic, we issue a warning and label this constraint as a potential weak constraint to prompt further investigation by the user.



\subsubsection{Fuzzy Words}\label{sec:fuzzword}

The example from~\cref{sec:eg2} illustrates a very typical implicit constraint with fuzzy words, where part of constraint is clearly defined while others remain uncertain. We introduce a series of fuzzy words to help LLM extract constraints better. These fuzzy words frequently appear in documentation but don't represent specific values, making it challenging for the LLM to extract them directly. We generally categorize these fuzzy words into two types: \textit{existence} and \textit{non-existence}. In fuzzy words, \textit{non-existence} includes ``ignore'', ``no effect'', ``unused'', ``override'', indicating that a parameter either is unused or does not exist within code segments where other conditions are met. Similarly, \textit{existence} includes ``specify'', ``have an effect'', ``exist'', ``significant'', indicating that the parameter is used or exists when other conditions are met.

We implement several specialized predicates to evaluate such implicit constraints. \tool will first trace the target variable's define-use chain (DU-chain)~\cite{kennedy1978use, harrold1994efficient} and then check if the definition and usage of it are existed or not under a specific program path. For instance, ``\texttt{exist(x)}'' will check if the definition and usage of a variable \texttt{x} are existed in the program path with other explicit conditions. If found, it will return \texttt{True}; otherwise, it will return \texttt{False}. Similarly, ``\texttt{ignore(x)}'' will check whether the definition and usage of \texttt{x} are absent under the program path with given explicit conditions. It can be further extended to check weak \emph{doc-constraints}. Like the example in~\cref{sec:eg2}, \textit{gamma} will be ignored not only when \texttt{``affinity$=$nearest\_neighbors''} but also ignored when \texttt{``affinity$=$precomputed\_nearest\_neighbors''} as well as \texttt{``affinity$=$precomputed''}.



\subsubsection{Fuzzy Constraint Satisfaction}\label{sec:fuzzlogic}

While we have employed several strategies in Phase II to maximize LLM's understanding of
constraints and restrict randomness in outputs, inaccurate extraction is still unavoidable.
The main reasons are threefold: (1) typos inevitably occur when developers write documentation; (2)
the hallucination issues inherent to black-box generative models; and (3) the intrinsic ambiguity in
natural languages. This implies that correct documentation descriptions can generate incorrect \emph{doc-constraints}, and incorrect documentation descriptions can also have the chance to generate correct \emph{doc-constraints}.

In the absence of LLM unpredictability, detecting CDI issues is a crisp constraint satisfaction
problem (CSP), deciding whether a \emph{doc-constraint} is consistent with the actual code implementation.
Nevertheless, due to minor errors introduced by LLMs, such as a single letter being wrongly spelled
in a parameter name, or a comparison operator being reversed, e.g., writing ``<'' instead of ``>'',
a \emph{doc-constraint} can be mistakenly identified as inconsistent when it is actually correct.

To address this, we proposed a customized fuzzy constraint logic that reconciles such
unpredictability.
In a traditional fuzzy constraint~\cite{kosko1993fuzzy, ruttkay1994fuzzy}, a membership function
assigns a degree of satisfaction (ranging from 0 to 1) to each possible variable value.
It enables partial fulfillment of a condition, with satisfaction measured on a continuous scale.
In our case, a constraint needs to be measured on a new scale, assessing ``how likely'' the
extracted \emph{doc-constraint} conforms to the \emph{code-constraints}.
Therefore, we introduced a unique similarity computation which serves as the membership function.

\cref{fig:ebnf} shows an EBNF grammar for our multi-parameter constraints.
A multi-parameter constraint is a combination and nesting of binary expressions and Boolean
operators, which can be viewed as a complete binary tree where leaf nodes are binary expressions
over single parameters and non-leaf nodes are logical operators connecting them.
Without loss of generality, we only keep negation, conjunction, and disjunction in the constraints;
logical relations such as implications can be simplified accordingly.
The fuzziness of a constraint is defined with respect to a set of \emph{environment expressions},
facts that are known to hold (with a truth value of 1).
In other words, the instantiation of a specific tree structure and nodes is a constraint $c$
evaluated against a set of expressions $\{e_1, e_2, \ldots, e_n\}$.
Next, we define the membership function of our fuzzy constraint logic through a few similarity
functions.


\begin{figure}[t]
\small\centering
\begin{align*}
   c \in Constraint &::= e \; |\; \neg c \;|\; c \vee c \;|\; c \wedge c \nonumber\\
   e \in Expression &::= \; p \bowtie v\\
   p \in Parameter &::= \; char,\{char \;|\; digit\} \nonumber\\
   \bowtie \; \in Operator &::= \; \textbf{<} \;|\; \textbf{>} \;|\; \textbf{<=} \;|\;
   \textbf{>=} \;|\; \textbf{=} \;|\; \textbf{!=} \; \nonumber\\
   v \in Value &::= \; string \;|\; number \;|\; bool \nonumber
\end{align*}%
\caption{Extended Backus-Naur form for multi-parameter constraint.}\label{fig:ebnf}
\vspace{-5pt}
\end{figure}

\begin{definition}[Expression Similarity]\label{def:es}
The similarity between two expressions $e_1$ and $e_2$ is defined as,
\begin{equation}\label{eq:simexpr}
  \sigma(e_1, e_2) = \alpha * (1 - \frac{LD(p_1, p_2)}{\max(|p_1|, |p_2|)}) + \beta *
  (\frac{\delta_{\bowtie_1} \cdot \delta_{\bowtie_2}}{\|\delta_{\bowtie_1}\|
  \|\delta_{\bowtie_2}\|}) + \alpha * (1 - \frac{LD(v_1, v_2)}{\max(|v_1|, |v_2|)}),
\end{equation}
where $\alpha$ and $\beta$ denote the relative weights, $p$, $\bowtie$, and $v$
are parameter, operator, and value, respectively, $|p|$ and $|v|$ denotes the length of $p$ and $v$, $\|\delta_{\bowtie}\|$ denotes the magnitudes (or Euclidean norms) of the vector $\delta_{\bowtie}$.
\end{definition}

The similarity between two expressions are considered separately for the parameters, operators, and
values appeared in the expressions.
Both $p$ and $v$ can be treated as texts, therefore, Levenshtien Distance (a.k.a.
edit distance) is used to represent their similarity.
The normalized Levenshtien Distance (NLD) is given in \cref{eq:ld}, where $s$ denotes strings ($p$
or $v$) and $|s|$ denotes the length of it.

\begin{equation}\label{eq:ld}
    \eta(s_1, s_2) = NLD =  1 - \frac{LD(s_1, s_2)}{\max(|s_1|, |s_2|)}
\end{equation}

\begin{equation}\label{eq:embedding}
    \delta_{\bowtie} = (C, E, G, L, N), \text{where } C, E, G, L, N \in \{0, 1\}
\end{equation}

\begin{equation}\label{eq:cossim}
    cos\theta(\bowtie_1, \bowtie_2) = \frac{\delta_{\bowtie_1} \cdot \delta_{\bowtie_2}}{\|\delta_{\bowtie_1}\| \|\delta_{\bowtie_2}\|}
\end{equation}

As illustrated in \cref{eq:embedding}, we design an operator embedding across five key dimensions: \underline{\textbf{C}}omparison, \underline{\textbf{E}}quality, \underline{\textbf{G}}reater than, \underline{\textbf{L}}ess than, and \underline{\textbf{N}}egativity.
This way, we may calculate the similarity between two operators by simply calculating the cosine
similarity between two vectors.
The result is highly intuitive. For example, with $\delta_{<} = (1, 0, 0, 1, 0)$, $\delta_{>} =
(1, 0, 1, 0, 0)$, and $\delta_{<=} = (1, 1, 0, 1, 0)$, the similarity between ``$<$'' and ``$>$''
is 0.5, while the similarity between ``$<$'' and ``$<=$'' is 0.82.

The weight of operator similarity is set as $\beta$ such that the weights of operators, values, and
parameters within a given experssion should sum to one.
Thus, we have $\alpha = \frac{1 - \beta}{2}$ and the similarity $\sigma$ of two single parameter
expressions (i.e., atomic constraint) can be calculated according to \cref{eq:simexpr}.


\begin{definition}[Constraint Similarity]
Let $c$ be a constraint and $\Phi = \{e_i | i=1,\ldots,n\}$ be a set of
environment expressions assumed to hold true.
The similarity of $c$ against $\Phi$ is given by the following set of calculations.
\begin{equation}\label{eq:js}
    \rho(c,\Phi) =
    \begin{cases}
        \mathop{\arg\max}\limits_{e_i \in \Phi} \sigma(e, e_i),
        & \text{if } c \text{ is an expression } e \\
        1 - \sigma(c', \Phi), & \text{if $c = \neg c'$} \\
        \min\{\sigma(c_1, \Phi), \sigma(c_2, \Phi)\}, & \text{if } c = c_1 \wedge c_2 \\
        \max\{\sigma(c_1, \Phi), \sigma(c_2, \Phi)\}, & \text{if } c = c_1 \vee c_2 \\
    \end{cases}
\end{equation}
\end{definition}

Consider an atomic constraint with a single expression; its similarity to $\Phi$ associated with
a set of \emph{environment expressions} can be represented by the maximum expression similarity among
all expressions within $\Phi$.
Based on the \textit{conjunctive combination principle}~\cite{zadeh1965fuzzy}, when combining two
constraints using a conjunction, their degree of joint similarity $\rho$ should be represented by
the minimum similarity between them. Similarly, based on the \textit{disjunctive combination
principle}~\cite{zadeh1965fuzzy}, when they are combined with a disjunction, the maximum similarity
should be used. For negation, the complementary similarity is used.



\begin{definition}[Membership Function for Fuzzy Constraint Satisfaction]
Constraint similarity serves as the membership function $\mu_{\Omega}$, quantifying the degree to
which a given constraint $\epsilon$ is consistent with the code, which is represented as a set
$\Omega$ of path constraints $\omega$:
%\vspace{-8pt}
\begin{equation}
\mu_{\Omega}(\epsilon) = \rho(\epsilon, \Phi_\Omega) \cdot \epsilon[e \mapsto e_{\Phi_\Omega}]
\end{equation}
where $\Phi_\Omega$ denotes the set of expressions aggregated from all the path constraints in
$\Omega$, and $\epsilon[e \mapsto e_{\Phi_\Omega}]$ is a rewrite of $\epsilon$, where each
expression $e$ has been replaced by its cloest counterpart from $\Phi_\Omega$.
\end{definition}

The inconsistency between the modified constraint $\epsilon[e \mapsto e_{\Phi_\Omega}]$ and
$\Omega$ is then evaluated (according to \cref{eq: uq1} or
\cref{eq: uq3}), yielding a binary result (\texttt{True} or \texttt{False}).
To enable the probabilistic interpretation, a linear transformation
ensures complementary probabilities. For instance, ``0.7$\cdot$False = 0.3$\cdot$True'', indicating
a 70\% probability of inconsistency or a 30\% probability of consistency.


\subsubsection{Constraint Similarity Threshold.} LLMs demonstrate a great potential in constraint extraction, yet they still encounter errors such as using incorrect parameter names or values and introducing non-existent constraints. To reduce false positives from these inevitable issues, we set a constraint similarity threshold of 0.85. This is based on the observation that a high constraint similarity (>0.85) indicates a high likelihood of misinterpretation or conflation by the LLM.

For instance, in \texttt{scikit-learn}, the documentation of \texttt{LinearSVC} states: ``\emph{If n\_samples < n\_features and optimizer supports chosen loss, multi\_class and penalty, then dual will be set to True}''. However, the extracted constraint is ``\texttt{$($samples$<$features$)$$\wedge$$($dual$=$True$)$}'', where two parameters are mistakenly mapped to similar names. Their constraint similarity is 0.86, exceeding the threshold, leading \tool to discard the result. Moreover, in most cases where the expression contains only a single parameter, the threshold exhibits stronger filtering capability.

However, setting a threshold cannot entirely exclude all false positives, as there is no definitive rule to ascertain whether an error originates from the documentation or the LLM. Another example from \texttt{scikit-learn} illustrates this limitation: the documentation of \texttt{estimator\_} states: \emph{``The child estimator template used to create the collection of fitted sub-estimators''}. Yet, the extracted constraint ``\texttt{$($estimator\_$=$child\_estimator\_template$)\wedge($collection$=$fitted\_sub\_estimators$)$}'' has a constraint similarity of 0.67, which is below the threshold, leading \tool to accept the result.









\section{Evaluation}\label{sec:eval}
This section describes our evaluation of \tool. We first present our research questions, then detail the experiment setup and evaluation subjects. Finally, we analyze our experimental results and answer each research question. Our evaluation was guided by the following research questions:

\begin{enumerate}
  \item \textbf{RQ1:} How accurate is \tool in extracting constraints from API documentation?
  \item \textbf{RQ2:} How effective is \tool in detecting errors related to multi-parameter constraints in API documentation?
  \item \textbf{RQ3:} How effective can \tool detect unknown inconsistency issues?
\end{enumerate}

\subsection{Experiment Setup}
\subsubsection{Dataset.}
There is currently no well-established dataset specifically focusing on multi-parameter API
documentation errors. To better evaluate the effectiveness of our tool, we constructed two datasets: a constraint dataset and an inconsistency dataset.

\paragraph{Constraint Dataset}
We constructed a dataset containing 72 constraints from 4 popular open-source data science libraries with well-maintained documentation, including \texttt{scikit-learn}, \texttt{scipy}, \texttt{numpy}, and \texttt{pandas}. The constraints were gathered by analyzing commits from each GitHub repository, focusing on developers' modifications to documentation related to multi-parameter constraints. To streamline this process, we developed an automated script to collect all documentation-related commits and identify changes within parameter (including attribute) descriptions, adhering to two distinct docstring styles (see details in Figure~\ref{fig:docstring}). By mapping parameter names to their descriptions, it then cross-checks if any parameter names appear within others' descriptions to keep potential constraint-related documentation. Approximately 90\% to 95\% of irrelevant commits are excluded, leaving a smaller subset of commits that may contain constraints. Additionally, we perform a random sampling of the excluded commits to assess and minimize the impact of this heuristic. Despite the approach's proven effectiveness, the remaining subset still contains a substantial number of submissions. Each of the four repositories has over 30,000 commits, requiring manual verification of approximately 1,500 commits per repository to further identify multi-parameter constraints. For each constraint, we log its source information (repository name, SHA, file path, etc.) and retain the code file, enabling swift extraction of documentation and code for reproducibility. 
In some cases, the condition enforcing the constraint is located not in the current function but in a function it calls, leading to a mismatch between documentation and code. To address this, we record such mismatches and relocate the constraint description to the function where the check actually occurs.




\paragraph{Inconsistency Dataset}
Based on constraint dataset, we constructed an inconsistency dataset comprising 126 multi-parameter constraints that lead to code-documentation inconsistencies. We analyzed around 20 resolved GitHub issues related to multi-parameter constraints and identified eight common patterns that may cause CDI: (1) Parameter name change; (2) Value Change; (3) Logic Change; (4) Remove Parameter; (5) Add Constraints; (6) Remove Constraints; (7) Missing Documentation; (8) Modify Description. To better evaluate the capabilities of our tool, we applied these eight patterns to mutate the dataset based on the \emph{Constraint Dataset}. For each constraint, we applied two types of modifications, resulting in an inconsistency dataset containing 216 constraints. We feed the mutated constraints and the original constraints into an SMT solver to verify if the mutations violate the original constraints. We also manually inspected each of them to ensure the constraint was inconsistent. It is important to note that modifying a correct constraint does not necessarily turn it into an incorrect one. As a result, we obtained an \emph{Inconsistency Dataset} containing 126 inconsistent constraints and 90 consistent constraints. In a sense, our dataset can be considered as potential inconsistencies that may realistically occur during development.

\begin{landscape}
\begin{table}[]
\caption{Data science libraries used in the experiments}
\begin{tabular}{lrrrrrrr}
\hline
\multicolumn{1}{c}{Project} & \multicolumn{1}{c}{class} & \multicolumn{1}{c}{class w/ doc} & \multicolumn{1}{c}{func} & \multicolumn{1}{c}{func w/ doc} & \multicolumn{1}{c}{KLOC} & \multicolumn{1}{c}{avg. params} & \multicolumn{1}{c}{\#Stars} \\ \hline
scikit-learn                & 878                       & 306                              & 9,332                    & 1,535                            & 400.1                    & 1.42                            & 61.8K                      \\
pandas                      & 2,211                     & 102                              & 28,880                   & 1,432                            & 620.6                    & 1.14                            & 45.2K                      \\
scipy                       & 2,570                     & 142                              & 22,059                   & 1,705                            & 517.9                    & 1.30                            & 13.6K                      \\
numpy                       & 2,000                     & 51                               & 12,618                   & 902                             & 276.3                    & 0.83                            & 29.4K                      \\ \hline
keras                       & 1,370                     & 254                              & 9,877                    & 724                             & 218.5                    & 1.42                            & 62.9K                     \\
dask                        & 284                       & 26                               & 6,914                    & 368                             & 157.5                    & 1.33                            & 13.1K                      \\
statsmodels                 & 2,184                     & 273                              & 11,590                   & 1,894                           & 424.6                    & 1.32                            & 10.6K                      \\ \hline
\end{tabular}
\label{tab:subjects}
\end{table}
\end{landscape}

\subsubsection{Subjects}
Table~\ref{tab:subjects} lists 7 popular libraries that \tool evaluated on, first four for dataset construction and last three for assessing \tool's ability to detect unknown issues. The selected libraries are of high quality and widely used, with tens of thousands of stars on GitHub. These libraries are substantial third-party libraries, averaging 1,642 classes, 14,467 functions, with an average of 373.6 thousand lines of code, and each function containing an average of 1.3 parameters. Our tool extracts documentation constraints and path constraints from these libraries and uses a fuzzy constraint reasoner to detect inconsistencies.

\tool was implemented in Python. All the experiments were performed on an Intel(R) Xeon(R) Silver 4214 CPU @ 2.20GHz machine with 252GB of RAM, running Ubuntu 18.04, with Python 3.8.19 and Z3 4.13.0.
When evaluating the constraint extraction performance (RQ1), we access the GPT-4 model through OpenAI's API. For result validation, given the absence of established benchmarks in this domain, two volunteer researchers independently reviewed the constraint extraction results from GPT-4, manually assessing each constraint's consistency with the original Python documentation. Any discrepancies were resolved through consensus discussion.


\subsection{Results}
\subsubsection{Accuracy of LLM in extracting constraints from API documentation.}
We display the result of RQ1 in Table~\ref{tab:rq1}. Our experiment shows that our tool correctly identified and extracted 66 out of 72 constraints contained in the Python documentation collected in our benchmark, achieving an accuracy of 91.7\%, which demonstrates that our tool can successfully extract most of the constraints accurately, with few errors or omissions.
Next, we look into the remaining failed cases and investigate the reasons for the inaccuracy. We found that out of the 6 incorrect cases, 4 involved missing constraints during the documentation processing. For example, one of the missing constraints is: ``(batch\_size = auto) $\rightarrow$ (batch\_size = min(200, n\_samples))''. Although this case involves a constraint between multiple parameters, the format is tricky because one of the parameters is within a function, which may have misled GPT-4 and caused it to miss this constraint during extraction. In the last 2 cases, the constraint information was identified but converted into incorrect logic expressions due to the complex logic or sentence structure.

We further explore \tool's abilities by conducting an ablation study. The results show that \tool without few-shot learning achieves an accuracy of 62.5\%. Most failures occur in the incorrect extraction of constraints. This indicates that including few-shot learning is important for \tool to generate accurate constraints.
Next, \tool without applying chain-of-thought techniques results in an accuracy of 79.2\%, with the number of missed constraints accounting for more than half of total non-equivalent cases. This suggests that GPT tends to miss more details in documentation when chain-of-thought is removed.
After including chain-of-thought and few-shot learning, \tool's performance shows a clear improvement.

\begin{table}[t]
	\vspace{2mm}
	\caption{Results of \tool on constraint extraction}
	\label{tab:rq1}
	\centering
  \small
	\begin{tabularx}{\linewidth}{l *{4}{>{\centering\arraybackslash}X}}
		\hline
		& \textbf{Equivalent} 
		& \multicolumn{2}{c}{\textbf{Non-Equivalent}} 
		& \textbf{Accuracy} \\ \hline
		& Correct extraction 
		& Incorrect extraction 
		& Missing constraints 
		& \\ \hline
		\textbf{\tool w/o few-shot learning} & 45 & 20 & 7 & 62.5\% \\ \hline
		\textbf{\tool w/o chain-of-thought}  & 57 & 7  & 8 & 79.2\% \\ \hline
		\textbf{\tool}                       & 66 & 2  & 4 & 91.7\% \\ \hline
	\end{tabularx}
\end{table}

\smallskip
\noindent\shadowbox{%
  \begin{minipage}{0.98\columnwidth}
    \textbf{Answer to RQ1:} \tool correctly extracted 66 constraints out of 72 in total, achieving an accuracy of 91.7\%. This demonstrates that \tool is effective in extracting constraints from Python documentation.
  \end{minipage}}



% ===== RQ2 =====
\subsubsection{\tool's effectiveness in detecting multi-parameter API documentation errors}
To measure \tool's effectiveness in detecting multi-parameter API documentation errors, we evaluated our tool on the inconsistency dataset. Table~\ref{tab:res} shows the results of \tool in detecting multi-parameter CDI on the inconsistency dataset.

The \vanillatool does not include fuzzy words or apply fuzzy constraint satisfaction theory, limiting its ability to handle implicit constraints. Despite these limitations, it achieved an impressive 69\% recall by detecting 87 inconsistencies. With fuzzy words and fuzzy constraints logic, \tool's performance significantly improved, successfully identifying 117 inconsistencies with a recall of 92.8\% and  an accuracy $=\frac{TP+TN}{TP+TN+FP+FN}=\frac{117+88}{216}=94.9\%$. This demonstrates that incorporating fuzzy words and fuzzy constraints can expand the range of detectable constraints. However, we identified two false positives because the constraints lie between the parameters and method calls, rather than among the parameters. One example is shown below:


\texttt{$($shape $=$ None$)$ $\wedge$ $($axes $\neq$ None$)$ $\rightarrow$ $($shape $=$ numpy.take$($x.shape,axes,axis=0$))$}


where, the value for ``shape'' is the function ``take()'' from numpy library. Modern software increasingly emphasizes code maintainability and reusability, leading to highly complex function calls, often involving nested or chained calls. To avoid the high risk of path explosion in symbolic execution, we alternate function calls with symbolic inputs during the preprocessing phase, which leads to misclassification of these two inconsistencies. 



The remaining 9 unresolved inconsistent constraints stem from external function dependencies. While incorporating external function code could resolve these constraints, this approach risks infinite recursive dependencies and path explosion. For us, best practices suggest that constraints should be handled within the documented function itself.




\begin{table}[]
	\vspace{2mm}
	\captionsetup{aboveskip=6pt, belowskip=-8pt}
	\begin{threeparttable}
		
  \caption{Results of LLM and \tool on detecting multi-parameter CDIs}
  \begin{tabularx}{0.91\linewidth}{p{0.75\linewidth}lrrrr}
  \hline
   Checker  & FN  & TP  & Recall\\ \hline
  LLM        & 119  & 7  & 5.6\% \\
  LLM+C      & 74  & 52  & 41.3\% \\
  \vanillatool   & 39  & 87  & 69.0\%\\
  \fuzzytool  & 9 & 117 & 92.8\% \\ \hline
  \end{tabularx}
  \begin{tablenotes}
  	\small
  	\item LLM: raw documentation and corresponding code; LLM+C: extracted \emph{doc-constraints} from documentation and corresponding code; \vanillatool: \tool without fuzzy words and fuzzy constraint logic; \fuzzytool: \tool with fuzzy words and fuzzy constraint logic.
  \end{tablenotes}
  \label{tab:res}
\end{threeparttable}
\vspace{-12pt}
\end{table}


A notable situation arose during one of our issue reporting, even though our issue had been confirmed, we encountered dissatisfaction from one of the developers. He believed we were an automated tool or bot based on AI because of our anonymous status, which diminished his enthusiasm for addressing the issue. With a mass of LLM-based program analysis or inconsistency checkers now available, while they offer insights sometimes, their results often cost more manual verification than traditional tools due to higher uncertainty.

\paragraph{Comparative Study}
Therefore, we conducted a comparative experiment between our tool and the approach of using only LLMs as a constraint checker on the same dataset. To align with our experiment settings, we chose GPT-4, one of the leading models, for comparison and evaluated its performance under two settings: 1) LLM: providing the raw documentation and code as input, directly prompting GPT to check for consistency, and 2) LLM+C: This is a two-phase processing. Extracting constraints using the LLM first, and then providing both these constraints and their corresponding code to the LLM for consistency check. In addition, we also require LLM to provide justifications for its answers.


As shown in the Table~\ref{tab:res}, when raw documentation and the corresponding code were provided as inputs to the LLM, LLM demonstrated significant limitations in detecting multi-parameter CDIs, finding only 7 inconsistencies with a recall of 5.6\%. When extracted constraints and code were given as inputs, LLM+C demonstrated heightened awareness of the task and had a higher probability of locating the constraint-related code segments. However, it still struggled to determine inconsistency. Out of 126 inconsistent constraints, 52 were identified correctly, yielding a recall of 41.3\%. After a thorough review of LLM's responses, we found that LLM+C gave correct results in many cases but provided unreasonable or even wrong explanations. These results demonstrate that the LLM still has limitations in detecting complicated multi-parameter CDIs, highlighting that our method’s design is the key factor in enhancing detection performance rather than any reliance on potential pretraining data leakage.



\smallskip
\noindent\shadowbox{%
  \begin{minipage}{0.98\columnwidth}
    \textbf{Answer to RQ2:}
		Large language model (LLM) exhibits limited capability in handling multi-parameter CDIs. Compared to LLM+C with a recall of 41.3\%, \fuzzytool successfully detected 117 out of 126 inconsistent constraints, achieving a recall of 92.8\%. Notably, \fuzzytool demonstrated a 23.8\% higher recall than \vanillatool, substantiating the effectiveness of the implementation of fuzzy words and fuzzy constraint logic.
  \end{minipage}}



% ===== RQ4 =====
\subsubsection{Practical effect of \tool}
Our tool's effectiveness in detecting unknown multi-parameter inconsistencies was validated by manual review and developer feedback. We reported 14 inconsistencies identified by \tool to the library maintenance team, receiving positive engagement and warm responses. Two of them even sparked further discussions about potential issues. This not only affirmed our reports' quality but also reflected the enthusiasm of the open-source community.

For example, an issue confirmed by the \texttt{scikit-learn} team originates from the independent function ``lars\_path'', as shown in Figure~\ref{fig:eg3}. Apparently, an inconsistency exists between the documentation and code regarding whether ``Gram'' is None when ``X'' is None. Therefore, we reported the issue~\cite{issue30099} and detailed the documentation sections with inconsistencies alongside its corresponding code snippet. The developer made a bit of archeology, admitted the documentation needed to be updated, and asked if we wanted make a PR to correct this error. Finally, the documentation description was fixed to ``If X is None, Gram must also be None''.

For most issue reports, we received quick feedback, and 11 inconsistencies were confirmed and improvements were made to documentation or code. Of the remaining three cases, two were reported at the initial phase of our experiments when our understanding of the project architecture was insufficient. The checks for these two constraints are done in other deeper files, but we were unable to verify them accurately at that time. The third inconsistency stemmed from ambiguity in the natural language, which resulted in a different interpretation diverged from the developers' original intent. Furthermore, these reported issues have contributed to four DS/ML repositories (\texttt{scikit-learn}, \texttt{keras}, \texttt{statsmodels}, \texttt{dask}), which emphasizes the generalization of our tool. 

At the time of writing, 10 out of 11 confirmed inconsistencies have been resolved: 7 through documentation fixes and 3 through updates to both documentation and code. This aligns with intuition: code errors are more likely to cause runtime failures and are thus easier to detect, whereas documentation errors and their potential efficiency impacts are often subtler and harder to identify.


\begin{figure*}[t]
	\vspace{2mm}
    \begin{subfigure}{\linewidth}
        \begin{tcolorbox}[colback=Emerald!10,colframe=cyan!40!black,title=\textbf{Constraint description of \texttt{X} and \texttt{Gram} in function \texttt{lars\_path}}]
            {\sffamily \textbf{> X } : None or ndarray of shape (n\_samples, n\_features)
            \\
            Input data. Note that \colorbox{blue!20}{\textbf{if X is None then the Gram matrix must be specified}}, i.e, cannot None or False.}
        \end{tcolorbox}
        \label{fig:eg3-doc}
    \end{subfigure}
    \begin{subfigure}{\linewidth}
\begin{tcolorbox}[colback=Salmon!20, colframe=Salmon!90!Black,title=\textbf{Corresponding code snippet in function \texttt{lars\_path}}]
\begin{lstlisting}[escapechar=@]
def lars_path(...):
    @\colorbox{blue!20}{if X is None and Gram is not None:}@
        raise ValueError("X cannot be None if Gram is not None. Use lars_path_gram to avoid passing X and y.")
\end{lstlisting}
\end{tcolorbox}
        \label{fig:eg3-code}
    \end{subfigure}
     \begin{subfigure}{\linewidth}
    	\begin{tcolorbox}[colback=Emerald!10,colframe=LimeGreen!60!black,title=\textbf{Fixed Constraint description of \texttt{X} and \texttt{Gram} in function \texttt{lars\_path}}]
    		{\sffamily \textbf{> X } : None or ndarray of shape (n\_samples, n\_features)
    			\\
    			Input data. Note that \colorbox{blue!20}{\textbf{if X is None, Gram must also be None.}}, If only the Gram matrix is available, use lars\_path\_gram instead.}
    	\end{tcolorbox}
    	\label{fig:eg3-fixed-doc}
    \end{subfigure}
    \caption{Example of the fixed documentation from \texttt{Scikit-learn}.}
    \label{fig:eg3}
    \vspace{-10pt}
\end{figure*}


\smallskip
\noindent\shadowbox{%
  \begin{minipage}{0.98\columnwidth}
    \textbf{Answer to RQ3:}  We reported 14 multi-parameter inconsistencies detected by \tool to library developers, who have already confirmed 11 inconsistencies by the time of submission (confirmation rate = 78.6\%)~\cite{issue28469,issue28470,issue28473,issue29440,issue29463,issue29464,issue9304,issue29509,issue11336,issue20141, issue30099}. These results demonstrate that \tool can effectively detect unknown API documentation errors. Some of them are even in unseen libraries which highlights its strong generalization capability.
  \end{minipage}}
\section{Discussions}

\subsection{Threats to Validity}\label{sec:valid-mpchecker}

\textbf{Internal}
There is no established ground truth for multi-parameter code-documentation inconsistencies. To
mitigate this, we manually reviewed and verified inconsistencies detected by \toolchecker. Given the
complexity of multi-parameter constraints, two of our authors spent an additional 10 minutes per
inconsistency to verify whether it was a true positive. Moreover, many of the confirmed true positives were further validated by the original library developers, enhancing the credibility of our manual labeling. Additionally, since the GPT-4 model used in our experiments had its last knowledge update in April 2023, and our first issue submissions occurred in February 2024, the risk of data leakage is not a significant concern for our approach.


\textbf{External} Our tool may not fully generalize to all Python libraries, particularly those outside the data science domain. Although our approach is designed to be broadly applicable, we concentrated on data science libraries due to several practical considerations: (1) they are among the most widely used in the Python ecosystem, (2) they commonly employ the two major docstring formats that \toolchecker supports, and (3) they provide well-structured documentation with rich multi-parameter constraints. To enhance the representativeness of our evaluation, we selected high-quality, widely adopted libraries with comprehensive API documentation and accessible source code. A further limitation arises from the unmature Python symbolic execution tools, which may not yet robustly handle all the latest language features. Addressing these challenges will require continued engineering efforts to broaden the applicability of our tool.



\subsection{Application Prospects}

Our framework's language-agnostic design extends beyond dynamic languages like Python, such as Java, where type information facilitates more comprehensive CDI detection.

Our work represents an effective integration of LLMs and traditional software analysis, with fuzzy constraint logic (FCL) acting as the glue that enables smooth synergy between the two.  For example, LLMs have
  been recently used to infer program specifications from code~\cite{Ma2025SAG}.
  In contrast to conventional verification techniques that yield binary outcomes, either
  \texttt{True} or \texttt{False}, FCL enables probabilistic evaluation of
  invariant validity within code contexts.
  This probabilistic framework will integrate better with LLMs, which may generate close yet
  incorrect program specifications. In particular, FCL can reduce non-logical
  errors caused by confusion between similar terms.

  
%\vspace{0.3cm}
\section{Related Work}\label{sec:related-mpchecker}

\paragraph{API Documentation Analysis}
Numerous empirical studies have revealed the challenges of maintaining high-quality API
documentation~\cite{aghajani2018large,arnaoudova2016linguistic, dagenais2010creating,
aghajani2020software, head2018not, liu2020generating, monperrus2012should, saied2015observational,
shi2011empirical, uddin2015api, zhong2020empirical, kang2021active}. These studies indicate that
documentation errors are prevalent, even in well-established and widely-used libraries.
Additionally, an empirical study from Aghajani et al.~\cite{aghajani2018large} shows that linguistic antipatterns in APIs
increase the likelihood of developers introducing errors and raising more questions compared to
using clean APIs. Saied et al.~\cite{saied2015observational} conducted an observational study focusing on API usage constraints
and their documentation. Zhong and Su~\cite{zhong2013detecting} proposed a method that combines natural language processing
(NLP) with code analysis to identify errors in API documentation, specifically targeting
grammatical mistakes (such as spelling errors) and incorrect code references (i.e., names that do
not exist in the source code). Lee et al.~\cite{lee2019automatic} developed a technique to extract change rules from code
revisions and apply them to detect outdated API names in Java documentation, with a particular
focus on names of Java classes, methods, and fields.

Another related field is code comments inconsistency~\cite{blasi2018replicomment, habib2018class, liu2014automatic, nie2019framework, panthaplackel2021deep, steidl2013quality, zhai2020cpc, wen2019large}. Existing research on code comment analysis predominantly follows two approaches. The traditional method employs program analysis and heuristic rules to detect inconsistencies between comments and the code. Technologies like CUP~\cite{liu2020automating}, CUP2~\cite{liu2021just}, and HebCup~\cite{lin2021automated} exemplify this approach, focusing on automatic just-in-time comment updates when corresponding code changes. The alternative approach leverages NLP techniques to retrieve and extract information from software artifacts.

\toolchecker focuses on a distinct problem, specifically on API documentation errors arising from multi-parameter constraints. These issues are more subtle and challenging to detect, particularly within data science libraries built on the dynamic language Python.


\paragraph{LLM-based Program Analysis}
A line of research~\cite{wadhwa2024codequality,jin2023programrepair, yang2024programrepair, xia2024programrepair, nam2024codeunderstand,zhang2024codeinconsistency, zhang2024codeinconsistency} focuses on using LLMs on program analysis. Wadhwa et al.~\cite{wadhwa2024codequality} focus on using LLMs to resolve code quality issues in multiple code languages. Several recent researches~\cite{jin2023programrepair, yang2024programrepair, xia2024programrepair} address applying LLMs on program repairing issues. Nam et al.~\cite{nam2024codeunderstand} apply the GPT model to explain code and provide usage details.
The existing approaches focus on different purposes compared to \toolchecker.
Zhang et al.~\cite{zhang2024codeinconsistencyfse, zhang2024codeinconsistencyase} use LLMs to extract constraints from code comments, and apply AST-based program analysis to identify inconsistencies.
Rong et al.~\cite{rong2024code} propose C4RLLaMA, a fine-tuned large language model based on the open-source Code Llama, to detect and correct code comment inconsistencies.


Overall, \toolchecker's approach is distinct in two aspects.
First, \toolchecker specifically focuses on detecting inconsistencies in multi-parameter constraints, which is a missing piece in state-of-the-art works.
Next, \toolchecker deals with code documentation, which involves longer and more complex text, and is more diverse than most code comments.




% \section{Conclusion}\label{sec:conclusion-mpchecker}

In this paper, we propose \toolchecker, a multi-parameter constraint checker for Python data science libraries. \toolchecker utilizes both LLMs and symbolic execution to detect inconsistencies between code and documentation. 
To mitigate the uncertainty introduced by LLM outputs, we apply fuzzy constraint logic to accommodate nearly-correct parameter constraints. The experimental results show that \toolchecker is effective in identifying multi-parameter API documentation errors. We further reported 14 detected inconsistencies, 11 of which were confirmed by the development team.
Our work intuitively explores the multi-parameter constraint inconsistencies between code and documentation, and may inspire more future studies in this field. 


\section{Data Availability}
The source code and dataset are available at \url{https://github.com/ParsifalXu/MPChecker}. The artifacts of this paper have been evaluated and are available online~\cite{zenodo}.



%!TEX root=../mythesis.tex
% Chapter Template

\chapter{Repo-level Code Localization} % Main chapter title
\chaptermark{DataLoc}  % replace the chapter name with its abbreviated form
\label{ch:chapter5}

\section{Introduction}\label{sec:intro-dataloc}
The rapid evolution of artificial intelligence has fundamentally reshaped every phase of the Software Development Life Cycle (SDLC), revolutionizing developer-codebase interactions. Through natural language, developers can now gain comprehensive insight into code repositories and perform sophisticated tasks such as code refactoring, feature implementation, and defect repair. At the heart of these capabilities lies the critical primitive of \textit{code localization}. Formally, code localization is the process of precisely identifying relevant source code snippets, ranging from specific methods to complex logic blocks, within a large-scale code repository that aligns with a natural language query. As the community pivots towards autonomous software engineering agents, the efficacy of code localization has emerged as the primary bottleneck. Accurately mapping high-level functional intent to intricate implementation logic is no longer merely a retrieval challenge, but an essential prerequisite for AI systems to reliably navigate and reason about real-world software architectures.

Contemporary state-of-the-art code localization approaches, which can be broadly categorized into three paradigms including embedding-based retrieval, pipeline-guided LLM workflows, and graph-augmented agentic exploration (Details in Section~\ref{sec:related}), have reported impressive results on issue-solving benchmarks such as SWE-bench~\cite{swebench2024}. However, our investigation reveals a significant but overlooked bias that we term the \textbf{Keyword Shortcut}. Recent mainstream benchmarks are predominantly curated from GitHub issues, which usually contain clear error traces or even verbatim code snippets. Such descriptions are inherently laden with \textbf{keywords} (e.g., precise class names or unique identifiers) that act as ``cheat sheets'', allowing models to locate code snippets via surface-level lexical matching rather than genuine logical reasoning. Our preliminary diagnostic study indicates that once these identifiers are stripped away, the performance of all three approaches suffers a catastrophic decline. This stark contrast uncovers a fundamental deficiency in current systems: a profound struggle with \textbf{Keyword-Agnostic Logical Code Localization (KA-LCL)}, where models must navigate codebases without the crutch of explicit naming hints. It is crucial to distinguish KA-LCL from general issue-solving code localization: while the latter is often reducible to a \textit{semantic matching} task between query keywords and code identifiers, KA-LCL represents a higher-order \textit{structural reasoning} challenge. To illustrate this, consider a seemingly straightforward structural logical query: \textit{``Find all functions where: (1) the function has more than 15 parameters, and (2) the function is not an \_\_init\_\_ method''} (Details in Section~\ref{sec:example}). Such an keyword-agnostic logical query poses a significant hurdle for SOTA approaches. While a human developer can easily identify these patterns by traversing the program's structural logic, SOTA solutions frequently fail because they cannot rely on semantic similarity to specific identifiers. Instead, these queries necessitate a deeper understanding of code structures, where existing approaches, deprived of naming cues, prove remarkably brittle.

To systematically evaluate the limits of existing tools, we first introduce \dataset, a diagnostic benchmark specifically curated for keyword-agnostic logical code localization. Unlike widely used benchmarks that take issue statements saturated with naming hints, \dataset targets scenarios where no key entities serve as anchors. By decomposing code structures, we synthesized a series of purely logical queries that focus on code patterns. Our diagnostic evaluation reveals a precipitous performance degradation in state-of-the-art methods, uncovering a critical ``reasoning gap'' that current AI-driven localization methods have yet to bridge.

The difficulty of KA-LCL arises from two intertwined technical bottlenecks that current paradigms are ill-equipped to handle. \ding{182} The absence of lexical anchors leads to an unmanageable search space, significantly exacerbating the \textit{lost-in-the-middle} phenomenon. Modern software systems are typically massive and complex, and model-based approaches rely heavily on specific identifiers as ``pruning signals'' to filter out irrelevant modules. Without these \textit{keyword shortcuts}, models are forced to ingest a vast volume of structural context to identify potential candidates. This data deluge overwhelms the limited context window of LLMs, where critical logical patterns become submerged within long input sequences, severely impairing the system's ability to robustly access and utilize relevant structural features. \ding{183} Existing approaches lack a deterministic reasoning mechanism to navigate the intricate hierarchical dependencies of a repository-level codebase. While pipeline-guided LLM workflows and graph-augmented agentic exploration attempt to address code relationships, they often operate as probabilistic recommendation systems that generate a ranked list of likely candidates, rather than performing rigorous structural deduction. Consequently, the lack of a formal reasoning framework prevents these systems from providing either a deterministic localization or a verifiable explanation, failing to bridge the gap from heuristic-based matching to genuine repository-level logical inference.

Conceptually, code localization can be viewed as a specialized form of code search~\cite{di2023code}. However, unlike general information retrieval, programming languages possess formally defined syntax and semantics that allow source code to be precisely parsed and analyzed. This formal nature endows code with an inherent reasonability that extends beyond surface-level text. From a high-level perspective, an effective repository-level localization engine requires a robust intermediate representation (IR) to bridge the semantic gap between natural language intent and implementation logic. Such an IR must effectively encode code entities, their intricate inter-relationships, and structural hierarchies, while remaining highly interpretable and actionable for LLM-based agents.

To overcome the aforementioned limitations, we propose \tooldataloc, a novel agentic framework that synergizes the rule-based reasoning of \textbf{Datalog} with the semantic power of LLMs to achieve precise, repository-level code localization. Our framework first employs static analysis to extract a comprehensive set of \textbf{program facts} from the source code, constructing a structured IR that captures both elemental properties and relational dependencies. Upon receiving a natural language query, the LLM agent interprets the underlying functional intent and synthesizes a corresponding Datalog query. As a powerful declarative logic programming language, Datalog is uniquely suited for traversing complex structural patterns that baffle traditional retrieval methods. These queries are then executed by \textbf{Soufflé}, a high-performance reasoning engine, which performs rigorous deduction against the pre-extracted facts to infer precise code locations. Crucially, by offloading structural reasoning to a deterministic engine, \tooldataloc not only significantly reduces token consumption but also empowers the agent to provide definitive negative responses when no matches exist. This avoids the common pitfall of probabilistic systems that hallucinate potential candidates, thereby achieving a paradigm shift from heuristic-based recommendation to verifiable, high-precision localization.

\textbf{Contributions.} Our work aims to integrate Datalog's rule-based inference engine with the advanced large language models. This framework embodies an exploration of the neuro-symbolic paradigm and hope to contribute to open science. In summary, we make the following contributions.

\begin{enumerate}
    \item We identify and formalize the \textit{Keyword Shortcut} bias in current code localization research. To address this, we introduce \dataset, a diagnostic benchmark specifically designed for Keyword-Agnostic Logical Code Localization (KA-LCL). It contains 25 high-quality purely logical queries with precise ground-truth locations, providing a rigorous testing ground for evaluating the structural reasoning capabilities of LLMs and AI agents.
    \item We proposed a novel agent-based framework for repo-level code localization that introduces program facts as an intermediate representation to capture both explicit and implicit code relationships. By synthesizing Datalog queries from natural language, \tooldataloc offloads intricate structural traversal to a high-performance deterministic reasoning engine, significantly enhancing reasoning capabilities and reducing token consumption.
    \item We implement our framework as an automated, end-to-end command-line tool. It features an iterative refinement mechanism where the LLM agent progressively generates and adjusts Datalog rules to navigate repositories. Our tool and benchmark are publicly available at: \url{https://anonymous.4open.science/r/DataLoc-EFF3}.
    \item We conduct an extensive evaluation of \tooldataloc on both \dataset and other issue-driven benchmarks. The experimental results demonstrate that \tooldataloc significantly outperforms state-of-the-art methods in KA-LCL tasks, achieving superior precision and the capacity for verifiable localization. Furthermore, \tooldataloc maintains competitive performance on standard issue-driven benchmarks, matching SOTA levels while offering higher reliability in handling negative queries through its deterministic logic.
\end{enumerate}




\section{Background}\label{sec:bg-dataloc}
This section introduces the background on program facts and Datalog that forms the symbolic reasoning component of our framework.

\subsection{Program Facts}
Program facts are a structured representation of information extracted from source code for the purpose of automated reasoning and analysis. They encode observable properties of a program, such as the existence of entities (e.g., functions, classes, variables), their attributes (e.g., names, locations, modifiers), and relations between them (e.g., containment, calls, inheritance), in a form suitable for systematic querying and inference.

Program facts are derived mechanically from source code through language-specific frontends, typically by parsing the code and traversing intermediate representations such as abstract syntax trees or control-flow graphs. Each extracted fact captures a single, well-defined aspect of the program, and together they provide a precise, machine-readable abstraction of the program’s structure and behavior. Importantly, program facts are \emph{descriptive}: they record what is present in the program, rather than how analyses should be performed.


\subsection{Datalog}

Datalog is a declarative logic programming language rooted in first-order logic and database theory. A Datalog specification consists of a set of rules that describe how new facts can be derived from existing ones.

\begin{definition}[Datalog Rule]
A Datalog rule has the form:
\begin{equation}
    R_0(t_1, \dots, t_k) \leftarrow R_1(u_1^{(1)}, \dots, u_{m_1}^{(1)}), \dots, R_n(u_1^{(n)}, \dots, u_{m_n}^{(n)})
\end{equation}
where each $R_i$ is a predicate symbol. The atom on the left-hand side is called the \emph{head}, and the atoms on the right-hand side form the \emph{body}.
Each argument $t_j$ or $u_j^{(i)}$ is either a constant or a variable.
\end{definition}

The rule is interpreted as follows: for any assignment of variables to constants that makes all body atoms simultaneously true, the corresponding instantiated head atom is also true. Variables thus serve as placeholders that allow a rule to match and relate multiple facts.

\begin{definition}[Datalog Program]
A Datalog program is a finite set of Datalog rules evaluated over a given set of ground facts. Its semantics is defined as the least fixpoint of rule application: rules are repeatedly applied to derive new ground facts until no further facts can be inferred.
\end{definition}

Datalog supports recursion and operates under a monotonic, set-based semantics, making it well suited for expressing transitive and structural properties such as reachability, dependency propagation, and hierarchical relations in program analysis.




\subsection{Program Facts in Datalog}

In program analysis, program facts are represented in Datalog as \emph{ground predicate instances}. Each predicate schema corresponds to a specific kind of program entity or relation, while each extracted program fact instantiates that schema with concrete constants derived from the source code.

For example, a predicate describing function definitions may be declared as:
\begin{minted}{prolog}
.decl function_definition(file_path: symbol, function_name: symbol, start_line: number, end_line: number, param_count: number, is_async: symbol, containing_class: symbol)
\end{minted}

An extracted function definition in the source code gives rise to a ground fact of this predicate, with all arguments bound to concrete values such as file paths, names, and line numbers.

Datalog rules operate over these ground program facts using variables to range over matching predicate instances. The body of a rule specifies patterns over existing program facts, while the head defines a new fact to be derived whenever the body is satisfied under some variable assignment. Through repeated rule application, Datalog derives higher-level program properties—such as reachability, dependency relations, or structural patterns—from the underlying set of extracted program facts.

This representation cleanly separates \emph{fact extraction}, which records concrete observations about the program, from \emph{logical inference}, which declaratively specifies how additional properties are derived.


\subsection{Motivating Example}\label{sec:example-dataloc}
% To demonstrate how the aforementioned definitions are applied to code localization, we consider a practical query designed to identify functions with more than 15 parameters that are not \_\_init\_\_ methods.

\begin{figure}
    \centering
\begin{tikzpicture}[node distance=0.5em, every node/.style={inner sep=0pt}]
    \node (issue) [anchor=north] {
        \begin{stepbox}[width=\linewidth, colframe=red!50!gray, boxed title style={colback=red!10, colframe=red!50!gray}]{Question}
            \begin{minted}[fontsize=\footnotesize]{text}
Find all functions where: (1) the function has more than 15 parameters, and 
(2) the function is not an __init__ method.
            \end{minted}
        \end{stepbox}
    };

    \node (code) [below=of issue] {
        \begin{stepbox}[width=\linewidth, colframe=cyan!50!gray, boxed title style={colback=cyan!10, colframe=cyan!50!gray}]{Datalog Query}
            \begin{minted}{prolog}
% EDB: Facts extracted from source code
.decl function_definition(file_path: symbol, function_name: symbol, start_line: number, end_line: number, param_count: number, is_async: symbol, containing_class: symbol)

% IDB: Derived analytical facts
.decl LargeFunctions(file_path: symbol, function_name: symbol, start_line: number, param_count: number, containing_class: symbol)

% Inference rule for localization
LargeFunctions(file_path, function_name, start_line, param_count, containing_class) :-
    function_definition(file_path, function_name, start_line, _, param_count, _, containing_class),
    param_count > 15,
    function_name != "__init__".

% Query
.output LargeFunctions
            \end{minted}
        \end{stepbox}
    };

    \node (result) [below=of code] {
        \begin{stepbox}[width=\linewidth, colframe=green!50!gray, boxed title style={colback=green!10, colframe=green!50!gray}]{Query Result}
            \begin{minted}[fontsize=\scriptsize]{text}
//  file_path                     function_name   start_line   param_count   containing_class
---------------------------------------------------------------------------------------------
astropy/convolution/convolve.py    convolve_fft        442	   19         module_level      
astropy/io/fits/column.py          _verify_keywords    952	   17         Column
            \end{minted}
        \end{stepbox}
    };

    \draw [-{Stealth[scale=1.2]}, line width=1pt, gray!60, shorten >=-12pt, shorten <=1pt] (issue.south) -- (code.north) 
        node[pos=1.8, right=6pt, circle, fill=black, text=white, inner sep=1pt, font=\tiny, ](n1){1};
        % node[right=2pt of num1, font=\footnotesize\sffamily, text=black] {Agent};

    
    \draw [-{Stealth[scale=1.2]}, line width=1pt, gray!60, shorten >=-12pt, shorten <=1pt] (code.south) -- (result.north) 
        node[pos=1.8, right=6pt, circle, fill=black, text=white, inner sep=1pt, font=\tiny](n2){2};
    \node[anchor=west, at={(n1.east)}, xshift=5pt, font=\scriptsize\sffamily, text=black] {Generated by LLM};
    \node[anchor=west, at={(n2.east)}, xshift=5pt, font=\scriptsize\sffamily, text=black] {Execute by Soufflé};
\end{tikzpicture}
    \caption{A motivating exampele of a logic query}
    \label{fig:example}
\end{figure}

To illustrate the query process in Datalog, we provide a motivating example. Consider a EA-LCL task that requires identifying functions in a codebase satisfying specific structural constraints: (1) the function has more than 15 parameters, and (2) the function is not an \_\_init\_\_ method.

Figure~\ref{fig:example} demonstrates a basic localization process. Given a natural language query, the Large Language Model translates the question's intention into a formal Datalog program. The generated program operates over two types of facts: \textit{Extensional Database (EDB)} facts extracted directly from source code, such as \texttt{function\_definition} containing metadata about each function's location, parameters, and context; and \textit{Intensional Database (IDB)} facts derived through logical inference, such as \texttt{LargeFunctions}. The Datalog query defines an inference rule (lines 8-11) that identifies target functions by matching against \texttt{function\_definition} facts. Irrelevant attributes are marked with underscore ``\_'' to avoid unnecessary computation, while the rule filters for functions with param\_count > 15 and excludes those named ``\_\_init\_\_''.

When executed by the Soufflé Datalog engine, the query precisely localizes two functions meeting the specified criteria: the \texttt{convolve\_fft} function with 19 parameters at line 442 in \texttt{astropy/convolution/convolve.py}, and the \texttt{\_verify\_keywords} function with 17 parameters at line 952 in \texttt{astropy/io/fits/column.py}. This example highlights the key advantages of our approach: the LLM bridges the semantic gap between natural language and formal logic, while Datalog ensures soundness and completeness of results through deductive reasoning over program facts, enabling precise localization without exhaustive manual repository traversal.
%%% Local Variables:
%%% mode: LaTeX
%%% TeX-master: "../main.tex"
%%% TeX-command-extra-options: "-shell-escape"
%%% End:

\section{Methodology}\label{sec:method}
In this section, we first formalize the repository-level code localization problem. We then analyze the intrinsic challenges that motivate our hybrid approach, \tooldataloc, which combines the strengths of large language models with the rigorous logical reasoning of Datalog.

Given a natural language query $q$ (e.g., a bug report or feature request) and a target codebase $\mathcal{C}$, the goal of \textbf{repository-level code localization} is to identify a list of relevant code locations $\mathcal{L}=\{l_1, l_2, \dots, l_k\}$. Effective localization bridges the gap between informal natural language and the rigid execution logic of software. We identify three primary challenges:

\begin{itemize}[leftmargin=*]
    \item \textbf{Challenge 1: Semantic and Lexical Disconnect.} There exists a non-trivial gap between the informal vocabulary of $q$ and the formal identifiers in $\mathcal{C}$. We categorize this disconnect into three progressive layers: (1) \textit{Keyword Absense}, where a query lacks any direct textual anchors present in the code; (2) \textit{Latent Semantic Mapping}, where high-level task descriptions (e.g., \texttt{login failure}) lack direct textual overlap with low-level implementation entities (e.g., \texttt{AuthManager} or \texttt{ValidateToken}); (3) \textit{Lexical Divergence}, where developers use synonyms or abbreviations (e.g., \texttt{fqn} for \texttt{fullyQualifiedName}) that elude exact match search.

    \item \textbf{Challenge 2: Non-local Structural Dependencies.} Relevant code is often not directly mentioned in queries but connected through dependencies or calls. In modern software, even a single feature may be implemented across multiple files. For example, identifying \textit{password validation} logic may require traversing complex call chains and data flows scattered across authentication modules and database layers. Existing pipeline-based tools rely on local directory traversal and fail to capture these deep, cross-file relational dependencies.
    
    \item \textbf{Challenge 3: Complex Logical Pattern Constraints.} Questions may contain logical pattern requirements that candidate code must satisfy. Certain localization tasks are defined by structural patterns rather than keywords. For example, \textit{``identifying the conditional statement that raises three distinct error types in different branches''} requires satisfying specific logical constraints defined by the language's syntax and semantics. Such patterns are difficult to locate by simply retrieving class or function information; it requires a more comprehensive understanding of the codebase and reasoning ability.
\end{itemize}

To address these challenges, we propose \tooldataloc, a synergy between LLMs and Datalog. Our intuition is that LLMs excel at resolving Challenge 1 by translating vague natural language intents into structured requirements, while Datalog provides the relational and reasoning power to solve Challenge 2 and 3 by performing exhaustive, sound traversal over codebase. 
In our framework, we represent the codebase $\mathcal{C}$ as a set of pre-extracted program facts $\mathcal{F}$, where $\mathcal{F}$ captures both the source entities and the structural relations (e.g., call graphs, inheritance). Each location $l_i \in \mathcal{L}$ corresponds to a specific level of granularity, such as a file, module, or function, that is essential for resolving the query $q$. Figure~\ref{fig:arch} illustrates the overview of the framework of \tooldataloc. Our framework operates in two stages. First, an offline program fact extraction stage analyzes the codebase to build a structured knowledge base. Second, an automated agent execution stage leverages these extracted facts to perform code localization by synthesizing high-quality Datalog queries.

\subsection{Program Facts Extraction}
In this step, we will discuss the details of extracting program facts from a given repository.
We extract program facts through a modular pipeline that separates source discovery, structural parsing, and fact emission. First, we enumerate source files under the repository root using configurable include/exclude patterns that respect version-control ignores, build artifacts, and generated code. Language frontend then analyzes its files using the most suitable intermediate representation. For Python, we parse source code into an abstract syntax tree and emit facts during traversal. The frontend produces Datalog facts annotated with precise source locations and stable identifiers, enabling traceability and incremental updates.


From these frontends, we emit facts describing program entities (files, modules, functions, classes, variables) and relations (containment, inheritance, import/use, call, and reference edges), together with optional control flow and data flow information when available. This representation directly addresses Challenge 2 (non-local structural dependencies) by making cross-file interactions explicit and queryable~\cite{wu2021diffbase}. Instead of relying on local directory traversal, localization can now operate over global dependency graphs, enabling the identification of code relevant to a feature even when it is scattered across multiple modules and layers.

Meanwhile, our facts enables expressive logical pattern matching, which helps solving the Challenge 3 (complex logical pattern constraints). Structural requirements, such as the presence of specific exception patterns, or multi-step call sequences, can be formulated as Datalog queries over extracted facts. This allows localization tasks defined by program structure rather than surface keywords to be handled systematically, without requiring the LLM to reason over raw source code.

\subsection{Agentic Workflow}
This section elaborates on our agent-based workflow for automated repository-level code localization, which builds upon the program facts constructed offline.

Our end-to-end agent is designed to accept natural language queries and return precise code locations. Unlike approaches that provide a fixed top-$n$ list of candidates, our system outputs a dynamic set of potential locations to maximize precision. To mitigate hallucinations and enhance abstention ability, we decouple reasoning from generation. A deterministic engine handles inference while the LLM functions as a coordinator for query analysis and result calibration. The agent is equipped with three basic tools: ``\texttt{exec\_dl}'' for executing Datalog programs, and ``\texttt{get\_file\_contents}'' along with ``\texttt{get\_sources}'' for retrieving source code and specific line ranges.

\subsubsection{Query Analysis and Information Resolution}
The workflow begins with performing a preliminary analysis to extract the core technical concepts and structural elements. By identifying program entities like specific file paths and module names, and core structure descriptions, the agent establishes an internal context. This preparatory step provides the necessary predicates and constraints for the subsequent synthesis of Datalog programs.

% \subsubsection{Syntax Correction and Intermediate-rule Diagnosis}
\subsubsection{Synthesize-Check-Refine Loop}
As shown in the red box in the Figure~\ref{fig:arch}, before execution, \tooldataloc follows a \textit{synthesize-check-refine} loop to mitigate the impact of hallucinations and improve the executability of LLM-generated Datalog programs. It mainly contains two critical, feedback-driven phases (Details in Section~\ref{sec:method:val} and \ref{sec:method:int}): (1) \textit{Syntax correction}. Each synthesized program will undergo a parser-gated validation to ensure syntactic well-formedness. Our workflow adopts a \emph{best-effort repair, then fallback} strategy: unambiguous cases are handled via conservative rule-based fixes, revalidated with \souffle’s parser, and all other cases return error feedback to the LLM.restructuring. (2) \textit{Intermediate-rule diagnosis}. We instrument the program to track row counts of intermediate relations, thereby identifying rules that produce empty results. Using controlled, mutation-based probing, we differentiate fragile-empty relations caused by over-constraints from stable-empty relations that reflect inherent dataset properties. These feedbacks help the model refine the generated program and ensure its quality before it reaches the inference engine. Validated programs are then executed through \texttt{exec\_dl} tool to generate a list of candidates. Excessive locations are treated as failures, triggering refinement with stricter constraints to reduce noise and improve precision.

\begin{figure}[t]
	\centering
	\includegraphics[width=\textwidth]{Figures/Chapter6/architecture.pdf}
	\caption{Overview of \tooldataloc framework}
	\label{fig:arch}
\end{figure}



% Programs undergo an initial syntactic parsing where simple errors are auto-corrected and ambiguous ones trigger an immediate return to the LLM. Survivors then proceed to semantic checks targeting Soufflé-specific constraints, with feedback provided for uncertain cases.


% our workflow integrates two critical, feedback-driven phases: parser-gated validation and mutation-based intermediate-rules diagnosis. 

% this process follows a ``synthesize-check-refine'' loop, as depicted in :

% \begin{enumerate}
%     \item \textit{Parser-Gated Validation.} Before execution, synthesized queries undergo a parser-gated validation workflow to ensure syntactic well-formedness. If failures occur, the system applies a \textit{best-effort repair} strategy, using rule-based fixes for unambiguous issues (e.g., reserved keyword renaming) or providing raw parser diagnostics back to the LLM for global restructuring.
%     \item \textit{Semantic Correctness Checking.} After passing the parser check, the workflow applies a set of semantic rules to rectify common logical misconceptions, such as inverted argument semantics in built-in predicates (e.g., contains). This ensures the query is not only syntactically correct but also logically sound before it reaches the execution engine.
%     \item \textit{Execution and Result Refinement.} Validated queries are executed via the \texttt{exec\_dl} tool. If the query returns an excessive number of results, the agent treats it as a failure and refines the logic with stricter constraints to minimize noise and improve precision.
% \end{enumerate}

\subsubsection{Context Retrieval and Final Verification}
As depicted in the yellow box in Figure~\ref{fig:arch}, once a set of potential code locations has been determined, the agent retrieves the relevant code snippets using \texttt{get\_sources} or \texttt{get\_file\_contents} and conducts a verification against the original query. These verified locations are then returned in a standardized format (e.g., \texttt{FILE\_PATH:CLASS.METHOD}). Although the agent may undergo multiple internal reasoning iterations, the user experience is streamlined into a single step: submitting a query and receiving a list of locations. This fully automated closed-loop design ensures both usability and seamless integration into production development environments.


% \begin{table}[htbp]
%     \centering
%     \caption{Available Tools for Agent}
%     \label{tab:available-tools}
%     % \vspace{2mm} 
%     \small
%     \begin{tabularx}{0.9\textwidth}{@{} l X @{}}
%         \toprule
%         \textbf{Tool Name} & \textbf{Description} \\
%         \midrule
%         \texttt{exec\_dl} & Execute Datalog programs (schema depends on detected language). \\
%         % \addlinespace
%         \texttt{get\_file\_contents} & Get complete source file contents using appropriate identifiers. \\
%         % \addlinespace
%         \texttt{get\_sources} & Get multiple specific line ranges from source files. \\
%         \bottomrule
%     \end{tabularx}
% \end{table}


\subsection{Program Repair for LLM Generated Datalog}
\label{sec:method:val}
In practice, LLM-generated Datalog programs (the dialect used by \souffle, in our case) often contain syntactic and semantic errors, especially when the model is not fine-tuned for Datalog programming.

Before executing an LLM-generated Datalog program, we enforce a \emph{parser-gated validation} workflow to ensure that only syntactically well-formed programs reach later stages of the pipeline. This design is motivated by the observation that some failures are caused by superficial syntactic issues that can be repaired deterministically, while more complex parse failures often require global restructuring that is better handled by the LLM. Our workflow therefore follows a \emph{best-effort repair, then fallback} strategy: we apply conservative rule-based fixes when the repair is unambiguous, re-check the program using \souffle's parser, and otherwise return error feedback to the LLM.

We invoke a lightweight parser helper (based on \souffle's parsing frontend) to validate the generated program. If parsing fails, we first attempt a small set of mechanical rewrite rules targeting high-frequency issues. For example, LLMs frequently introduce naming collisions by using reserved or special identifiers as variable names. E.g., using \texttt{count} as a variable name while it is a reserved keyword in \souffle. Such cases can be fixed locally via deterministic renaming.
%We similarly normalize other local parse hazards (e.g., missing terminators, malformed aggregate punctuation) when the repair is syntactically and semantically safe.

However, not all parser errors admit a reliable deterministic repair.
%Complex failures,s uch as malformed rule structure, severely mismatched parentheses, or directive misuse intertwined with rule bodies, often have multiple plausible fixes and may require the program to be re-synthesized rather than patched.
In these cases, we do not attempt speculative transformations. Instead, we return the raw parser diagnostics (optionally augmented with concise hints) to the LLM, allowing the model to revise the program directly.

Any program that does not pass the parser check is rejected and never proceeds to semantic validation or execution. Only after the program passes syntactic validation (either initially or after rule-based fixes) do we apply semantic rule checking and subsequent execution. Then we apply a set of semantic correctness checks that encode \souffle-specific usage rules and common domain conventions observed in LLM-generated programs. These checks target misconceptions that LLMs frequently exhibit when generating Datalog program.
%This staged design prevents cascading failures in later steps and ensures that semantic checking operates over a well-defined abstract syntax.


%Relying solely on iterative prompting or post-hoc error messages from the Datalog engine can lead to repeated failures and excessive tool calls. To address this issue, we introduce a repair component that automatically validates and repairs LLM-generated Datalog programs before sending it to the \souffle{} engine.

%We first employ \souffle's parser to perform strict syntactic validation on the generated program. Many syntactic issues can be easil
%Unlike treating the engine as a black box that only reports errors after a failed execution, we explicitly surface parse-time diagnostics and use them to drive targeted rewrites. Common syntactic issues include missing rule terminators, malformed aggregates, unmatched parentheses, and incorrect use of directives such as .decl, .input, and .output.
%For high-confidence cases, we apply deterministic fixes (e.g., normalizing aggregate syntax, inserting missing delimiters, or correcting directive placement). Programs that cannot be repaired unambiguously at this stage are passed through unchanged but annotated with structured diagnostics for downstream hint generation.

A representative example is the use of string containment constraints. \souffle{} provides a constraint function
\texttt{contains(sub:symbol,full:symbol)}, which is defined such that the second argument must contain the first (i.e., full includes sub). However, we observe that LLMs frequently invert this positional relationship by producing atoms like \texttt{contains(content,"keyword")} instead of the correct \texttt{contains("keyword",content)}. Since such inversion conforms to both syntax and type specifications, it leads to silent failure or empty results that are difficult for LLM to self-correct, even with multiple iterations of try and feedback.
% This error does not trigger any syntax or type error, produces a well-formed but logically inverted condition, and leads to silent failure or empty results that are difficult for LLM to realize, even with multiple iterations of try and feedback.

Prompt techniques such as few-shot learning cannot effectively eliminate this kind of error. Therefore, we construct a library of semantic rules that check the correct usage of built-in predicates with non-commutative argument semantics and other similar constraints. These repairs are applied only when the transformation is high-confidence and semantics-preserving. When uncertain, the checker records the issue for subsequent feedback to the LLM rather than applying a blind fix, thereby guiding the refinement in subsequent iterations.
We evaluate the contribution of this mechanism through an ablation study in~\cref{sec:ablation}, demonstrating its effectiveness in improving synthesis quality. 

% If uncertainty exists, the checker refrains from modifying the program and instead records the issue for later feedback to the LLM, guiding it to generate better results in subsequent iterations.



\subsection{Diagnosing Intermediate Rules via Conservative Mutation Analysis}
\label{sec:method:int}
\begin{figure}[t]
	\centering
	\includegraphics[width=\textwidth]{Figures/Chapter6/mutate.pdf}
	\caption{Intermediate rules mutation and feedback}
	\label{fig:mutate}
\end{figure}
We introduce an intermediate-rule diagnostic and mutation-based feedback mechanism to improve both the efficiency and effectiveness of LLM-based Datalog synthesis. As illustrated in~\cref{fig:mutate}, instead of evaluating a generated program solely by its final output, we instrument execution to collect row counts for intermediate relations and identify rules whose derived relations are empty. In practice, empty intermediate relations frequently indicate overly restrictive constraints, mismatched join keys, or incorrect predicate usage, and therefore serve as a useful signal for localizing potential errors in synthesized programs.

An empty relation is not inherently incorrect: depending on the user’s constraints and the underlying dataset, the semantically correct result may legitimately be the empty set. To avoid forcing spurious revisions or encouraging the model to hallucinate evidence, our approach explicitly distinguishes between fragile and stable emptiness through controlled diagnostic probing.

When detecting an intermediate relation that produces zero rows, we pick several applicable mutations from a small, fixed set of lightweight diagnostic mutations to the corresponding rule and re-execute the program, as shown in~\cref{fig:mutate}. These mutations are structure-preserving and intentionally conservative (e.g., relaxing exact string equality to substring or pattern matching, or weakening a single brittle filter), and they are used only for diagnosis, not as candidate replacements for the final query. We then observe whether any mutation yields non-empty results and how row counts change relative to the original execution.

Based on this behavior, we classify empty intermediates into two categories. A relation is considered fragile-empty if at least one diagnostic mutation produces a non-empty result, suggesting that the original rule may be over-constrained or mis-specified. In this case, we return targeted feedback identifying the affected relation, the specific mutations applied, and the observed changes in row counts (optionally including a small sample of newly surfaced tuples). Conversely, a relation is considered stable-empty if it remains empty under all tested mutations. In such cases, emptiness may reflect a genuine property of the dataset under the stated constraints rather than a synthesis error. To remain conservative, we do not pressure the model to introduce relaxations; instead, we report that the empty result appears robust and encourage the model to either preserve the current semantics or emit auxiliary diagnostic outputs rather than altering the core query.

Results produced by mutated programs are never used as final answers. Mutations serve only as execution-guided diagnostic probes to help the model decide whether and where revision is warranted. This design balances error localization with semantic caution, enabling targeted repair when evidence exists while avoiding misleading feedback in cases where empty results are likely correct. 


% \begin{algorithm}[t]
% \caption{Intermediate-Rule Diagnostic Feedback}
% \begin{algorithmic}[1]
% \Require Program $P$, dataset $D$
% \Ensure Feedback $F$

% \State $trace \gets \Call{InstrumentedExecute}{P, D}$
% \For{each empty relation $(r, rule) \in trace$}
%     \State $M \gets \Call{GenerateMutations}{rule}$ \Comment{Conservative probes}
%     \For{each mutation $m \in M$}
%         \State Test $m$ and record row count changes
%     \EndFor
%     \If{any mutation yields non-empty results}
%         \State $F \gets F \cup \{\text{fragile-empty}: \text{suggest revision}\}$
%     \Else
%         \State $F \gets F \cup \{\text{stable-empty}: \text{preserve semantics}\}$
%     \EndIf
% \EndFor
% \State \Return $F$
% \end{algorithmic}
% \end{algorithm}

% \begin{algorithm}
% 	\caption{\tool: Neurosymbolic Code Localization}
% 	\label{alg:main}
% 	\begin{algorithmic}[1]
% 		\Require Codebase $C$, Question $q$, Max iterations $maxIterations$
% 		\Ensure Code locations $L$
% 		\State $\mathcal{F} \leftarrow$ ExtractFacts($C$)
% 		\State $intent \leftarrow$ AnalyzeQuery($q$)
% 		\State $query \leftarrow$ GenerateDatalog($intent, \mathcal{F}$)
% 		\State $currentIterations \leftarrow 1$
% 		\State $candidates \leftarrow \emptyset$
% 		\While{$currentIterations \leq maxIterations$}
% 		\State $results \leftarrow$ ExecuteDatalog($query, \mathcal{F}$)
% 		\If{$results = \bot$} \Comment{Execution failed}
% 		\State $error \leftarrow$ GetExecutionError($query$)
% 		\State $query \leftarrow$ RefineDatalogQuery($query, error, intent, \mathcal{F}$) 
% 		\State $currentIterations \leftarrow currentIterations + 1$
% 		\Else
% 		\State $sources \leftarrow$ GetSources($results$)
% 		\State $candidates \leftarrow$ ValidateAndFilter($results, sources, intent$)
% 		\State $\mathcal{L} \leftarrow \mathcal{L} \cup candidates$
% 		\EndIf
% 		\EndWhile
% 		\Return $\mathcal{L}$
% 	\end{algorithmic}
% \end{algorithm}


%We introduce an intermediate-rule diagnostic and mutation-based feedback mechanism to improve both the efficiency and effectiveness of LLM-based Datalog synthesis. Instead of evaluating a generated program solely by its final output, we instrument execution to collect row counts for intermediate relations and identify rules whose derived relations are empty. In practice, empty intermediate relations frequently indicate overly restrictive constraints, mismatched join keys, or incorrect predicate usage, and therefore serve as a useful signal for localizing potential errors in synthesized programs.
%
%At the same time, an empty relation is not inherently incorrect: depending on the user’s constraints and the underlying dataset, the semantically correct result may legitimately be the empty set. To avoid forcing spurious revisions or encouraging the model to hallucinate evidence, our approach explicitly distinguishes between fragile and stable emptiness through controlled diagnostic probing.
%
%Concretely, for each intermediate relation that produces zero rows, we apply a small, fixed set of lightweight diagnostic mutations to the corresponding rule and re-execute the program. These mutations are structure-preserving and intentionally conservative (e.g., relaxing exact string equality to substring or pattern matching, or weakening a single potentially brittle filter), and they are used only for diagnosis, not as candidate replacements for the final query. We then observe whether any mutation yields non-empty results and how row counts change relative to the original execution.
%
%Based on this behavior, we classify empty intermediates into two categories. A relation is considered fragile-empty if at least one diagnostic mutation produces a non-empty result, suggesting that the original rule may be over-constrained or mis-specified. In this case, we return targeted feedback identifying the affected relation, the specific mutations applied, and the observed changes in row counts (optionally including a small sample of newly surfaced tuples). Conversely, a relation is considered stable-empty if it remains empty under all tested mutations. In such cases, emptiness may reflect a genuine property of the dataset under the stated constraints rather than a synthesis error. To remain conservative, we do not pressure the model to introduce relaxations; instead, we report that the empty result appears robust and encourage the model to either preserve the current semantics or emit auxiliary diagnostic outputs rather than altering the core query.
%
%Results produced by mutated programs are never used as final answers. Mutations serve only as execution-guided diagnostic probes to help the model decide whether and where revision is warranted. This design balances error localization with semantic caution, enabling targeted repair when evidence exists while avoiding misleading feedback in cases where empty results are likely correct. We evaluate the contribution of this mechanism through an ablation study (\cref{TODO:ablation}), demonstrating its effectiveness in accelerating convergence and improving synthesis quality without sacrificing correctness.





%For each occurrence of such predicates, the checker verifies whether:

%When a high-confidence inversion is detected


% \subsection{The \tooldataloc workflow}

%%% Local Variables:
%%% mode: LaTeX
%%% TeX-master: "../main.tex"
%%% TeX-command-extra-options: "-shell-escape"
%%% End:

\section{Evaluation}\label{sec:eval}

To evaluate the effectiveness and practicality of our approach at repository-level, we design the following research questions:
\begin{enumerate}
	\item \textbf{RQ1:} How effective is \tooldataloc in keyword-agnostic logical code localization?
	\item \textbf{RQ2:} How effective is \tooldataloc for issue-based code localization?
	\item \textbf{RQ3:} How efficient is \tooldataloc compared to baselines?
    \item \textbf{RQ4:} How does each component of \tooldataloc contribute to its performance?
\end{enumerate}


\subsection{Experiment Setup}

\subsubsection{Benchmarks.} 
We evaluate code localization performance on three Python-based benchmarks, covering both complex logical reasoning challenges and industrial issue-resolution tasks.

\textbf{SWE-bench Lite}~\cite{swebench2024}. A carefully curated and widely recognized subset from the full SWE-bench for more efficient and cost-effective evaluation of autonomous issue-solving capabilities. It consists of real-world GitHub issues with repository metadata and ground-truth patch locations. Following Suresh et al.~\cite{suresh2024cornstack}, we retained 274 of 300 original instances where patches modify existing functions or classes. We intentionally excluded instances introducing code corresponding to new functions or import statements to focus the evaluation on code localization within existing structures.

\textbf{\dataset} (Ours). To evaluate the capability of localization approaches in keyword-agnostic logical code localization, we constructed \dataset, a diagnostic benchmark comprising 25 high-quality logic-intensive queries. As illustrated in Table~\ref{tab:code_dimensions}, each query is formulated as a composite logical proposition by integrating code features across multiple dimensions. By combining structural granularity (e.g., classes, methods) with behavioral attributes (e.g., control flow, exception handling) and code metrics (e.g., inheritance depth and branch count), \dataset captures complex patterns that demand deep repository understanding.

For each query, we utilize the environment (repository and base commit version) from the first case of SWE-bench Lite as the foundation. For each query, we executed searches using \texttt{Cursor} and \texttt{GitHub Copilot} in agent mode with multiple latest advanced models like Claude-4.5-Opus and GPT-5.2, and manually validated all returned results to establish the ground-truth locations.

\dataset serves as a critical complement to issue-based benchmarks for localization task. In practice, issues are one of the most important channels for error feedback between users and maintainers. To facilitate debugging, those issue descriptions often provide sufficient information and clear keywords as cues to help maintainers better locate faults, such as accurate file paths, function identifiers, or even specific code snippets. Our analysis of SWE-bench Lite instances reveals that over 50\% of ground-truth locations are mentioned in the issue descriptions. Such \textit{keyword shortcut} enables models to succeed via simple lexical matching (e.g. grep) or embedding-based retrieval, without requiring genuine understanding and reasoning over the codebase. This undermines the validity of localization performance evaluations. Moreover, LLM-assisted development shifts the codebase interaction toward intent-based question answering, allowing developers to query repositories using natural language. However, for developers unfamiliar with a given repository, they typically cannot use precise identifiers and instead tend to express their search intent through high-level behavioral pattern descriptions or abstract logical structures. 


\textbf{\negset} (Ours). \negset is a variant of \dataset where queries are intentionally modified to ensure their ground-truth sets are empty. Current methods often adopt top-$n$ ranking to maximize recall, but ideal robust localization requires the ability to provide ascertained answers and avoid false positives. \negset evaluates the abstention capability when no valid location meets the query. Such ``refusal'' mechanism is a critical metric for ensuring the reliability of autonomous agents in production environments.


\begin{table}[htbp]
\centering
\caption{Taxonomy of Python Code Dimensions and Representative Elements}
\label{tab:code_dimensions}
\small
\renewcommand{\arraystretch}{1.2}
\begin{tabularx}{\textwidth}{lX}
\toprule
\textbf{Query Dimensions} & \textbf{Examples / Typical Elements} \\ \midrule
Code Structure & Functions, Methods, Classes, Modules, Decorators \\
Control Flow & Conditional (\texttt{if-elif-else}), Iteration (\texttt{for}, \texttt{while}), Context Management (\texttt{with}) \\
Condition Logic & Comparison (\texttt{==}, \texttt{>}), Identity (\texttt{is}), Membership (\texttt{in}), Type Checks (\texttt{isinstance}), Logical Operators (\texttt{and}, \texttt{or}, \texttt{not}), Early Exit (\texttt{return}, \texttt{break}, \texttt{continue}) \\
Data Structure & Built-in Collections (\texttt{list}, \texttt{dict}, \texttt{set}), Primitive Types (\texttt{int}, \texttt{str}) \\
Function Signatures & Default Values, Variadic Parameters (\texttt{*args}, \texttt{**kwargs}), Type Annotations \\
Exception Handling & Exception Propagation (\texttt{try-except-finally}), Exceptions (\texttt{TypeError}) \\
Code Metrics & Nesting Depth, Inheritance Depth, Assertion Count, Branch count \\ \bottomrule
\end{tabularx}
\end{table}

% To assess the effectiveness of our framework, we require datasets containing ground-truth annotations for code localization that can serve as reliable and standardized benchmarks. SWE-bench~\cite{jimenez2024swebench} constitutes a widely adopted benchmark for evaluating the capabilities of AI systems in performing end-to-end bug fixing across repository-level codebases. Each instance consists of a GitHub issue paired with its corresponding code patches. SWE-bench Lite~\cite{swebench2024} provides a carefully curated subset of 300 tasks from the full benchmark, designed to reduce evaluation costs while preserving the benchmark’s representativeness and overall quality. Following the approach of Suresh et al.~\cite{suresh2024cornstack}, we retained 274 of the 300 instances where patches modify existing functions or classes, excluding instances that introduce new functions or import statements. As code localization constitutes a critical yet implicit intermediate step in bug fixing process, we use the modified code locations from the patches as ground truth to evaluate localization performance.

% Similar to Swe-Bench, LocBench~\cite{chen-etal-2025-locagent}, proposed by Chen et al., is a dataset specifically designed for code localization, comprising 560 issues from Python repositories. Collected after October 2024 to mitigate data leakage and pre-training bias in recent LLMs, it encompasses a broad range of issue categories beyond bug fixing. After removing inaccessible repositories, we retain xxx of the original 560 examples. In summary, for Python, we adopt both SWE-bench Lite and LocBench.

% Our framework supports not only Python but also Java. For Java evaluation, we extract relevant instances from two multilingual datasets: SWE-bench Multilingual~\cite{yang2025swesmith}, which provides 300 tasks across 42 repositories and 9 programming languages including Java, and Multi-SWE-bench~\cite{zan2025multiswebench} from ByteDance, spanning 7 languages with 1,632 high-quality instances. After extracting Java-related entries and deduplicating across both sources, the combined dataset contains XXX instances. We manually annotate code locations as ground truth based on the \texttt{git diff} information, forming a dataset we refer to as SWE-bench Java.

% \todo{average file change, class change, function change, etc.}

\subsubsection{Baselines.} To assess \tooldataloc, we select four state-of-the-art baselines representing three distinct technical paradigms: embedding-based, pipeline-based, and agent-based approaches:
\begin{enumerate}[leftmargin=*]
    \item \textbf{SweRank}~\cite{reddy2025swerank} (\textit{Embedding-based}): It utilizes a retrieve-and-rerank architecture to identify issue locations. It employs SWERankEmbed (137M/7B parameters) to perform initial retrieval and SWERankLLM (7B/32B parameters) to rerank the results.
    \item \textbf{Agentless}~\cite{xia2024agentless} (\textit{Pipeline-based}): This approaches employs a hierarchical filtering strategy within a procedural workflow. It progressively prunes the search space from the file level down to specific classes or functions, utilizing an LLM to rank and select candidates at each stage.
    \item \textbf{LocAgent}~\cite{chen-etal-2025-locagent} (\textit{Agent-based}): It constructs a graph-based representation and sparse indexes of the project and enable an autonomous agent to perform iterative, tool-assisted retrieval.
    \item \textbf{CoSIL}~\cite{jiang2025cosil} (\textit{Agent-based}): This framework focuses on structural dependency traversal through call graphs to identity implicit locations via iterative exploration. It incorporates pruning to maintain context efficiency and restrict the search to high-relevance execution paths.
    % \item \textbf{Orca Loca~\cite{yu2025orcalocallmagentframework}:} Integrates priority-based scheduling, action decomposition with relevance scoring, and distance-aware context pruning. By optimizing the synergy between agentic reasoning and precise retrieval, it effectively navigates complex repositories to resolve the suboptimality of current search mechanisms.
\end{enumerate}

\subsection{Metrics}
We evaluate localization performance at three granularity: \textit{file, module, and function}. Let $Q$ denote the set of query instances, $G_q$ the set of ground-truth locations for query $q \in Q$, and $\mathcal{P}_q$ the set of predicted locations inferred by our agent workflow. $\mathbf{1}(\cdot)$ denotes the \textbf{indicator function}, which equals 1 if the logical condition holds and 0 otherwise. We adopt the following six metrics:

\begin{enumerate}[leftmargin=*, label=\textbf{M\arabic*.}]
    \item \textbf{Accuracy@k (ACC@k):} It measures the ability to achieve full coverage, where a success requires all ground-truth locations to be present within the top-$k$ predicted locations. When $k$ equals the length of the prediction set, this metric becomes the \textbf{Success Rate (SR)}:
    \begin{equation}
        Acc@k = \frac{1}{|Q|} \sum_{q \in Q} \mathbf{1}(G_q \subseteq \mathcal{P}_{q,k})
    \end{equation}

    \item \textbf{Recall (REC):} It represents the proportion of ground-truth locations successfully captured by the predicted set $\mathcal{P}_q$:
    \begin{equation}
        Rec@k = \frac{1}{|Q|} \sum_{q \in Q} \frac{|G_q \cap \mathcal{P}_{q,k}|}{|G_q|}
    \end{equation}

    \item \textbf{Precision (PRE):} This metric penalizes overprediction by calculating the fraction of predicted locations that are correct:
    \begin{equation}
        Pre = \frac{1}{|Q|} \sum_{q \in Q} \frac{|G_q \cap \mathcal{P}_q|}{|\mathcal{P}_q|}
    \end{equation}

    \item \textbf{Average Jaccard Similarity (AJS):} It quantifies the overlap between the predicted and ground-truth sets, which penalizes both missing targets and redundant predictions:
    \begin{equation}
        AJS = \frac{1}{|Q|} \sum_{q \in Q} \frac{|G_q \cap \mathcal{P}_q|}{|G_q \cup \mathcal{P}_q|}
    \end{equation}

    \item \textbf{Perfect Location Rate (PLR):} The most Stringent metric, measuring the ratio of instances where the predicted set $\mathcal{P}_q$ exactly matches the ground-truth set $G_q$. A PLR of 1.0 indicates perfect localization without any extraneous noise (i.e., $AJS = 1.0$):
    \begin{equation}
        PLR = \frac{1}{|Q|} \sum_{q \in Q} \mathbf{1}(\mathcal{P}_q = G_q)
    \end{equation}

    \item \textbf{Hit Rate (HR):} The most lenient metric, measuring the ratio of instances where the predicted set $\mathcal{P}_q$ provides at least one correct location:
    \begin{equation}
        HR = \frac{1}{|Q|} \sum_{q \in Q} \mathbf{1}(\mathcal{P}_{q} \cap G_q \neq \emptyset)
    \end{equation}
\end{enumerate}




\subsubsection{Implementation and environment.} All experiments were conducted on a server equipped with an Intel Xeon Silver 4216 CPU (2.10 GHz) and 62 GB RAM, running Ubuntu 22.04.5 LTS. Our framework was implemented using Python 3.12.11 and the Soufflé 2.4 Datalog engine. To evaluate \tooldataloc, we accessed \texttt{gpt-4o-20240513} via OpenAI’s API, \texttt{claude-3-5-sonnet-20241022} through AWS Bedrock services, \texttt{Deepseek-reasoner} via DeepSeek' API, and \texttt{Qwen3-Max} via Alibaba Cloud Service. For baseline comparisons, we instantiated runtime environments according to their respective official specifications and dependency requirements to ensure a fair evaluation.


\subsection{Results}

\subsubsection{Effectiveness for logic query}

As summarized in Table~\ref{tab:comparison_lq}, \tooldataloc achieves a decisive lead over all baselines across all metrics and granularities. At the file level, \tooldataloc reaches a Precision of 65\% and a Success Rate of 64\%, which is significantly higher than the baseline methods. Additionally, while all baselines fail completely to achieve perfect location (0\% PLR), \tooldataloc attains a PLR of 44\%, indicating the unique advantage of our framework's in capturing code structure and reasoning capabilities in assisting precise localization. To evaluate the generality of our framework, we applied it to Qwen3-Max. The framework achieved strong performance, even surpassing Claude-3.5 on some metrics, suggesting that it generalizes well across different models.

Even with the most lenient metric, Hit Rate (HR), which only requires at least one correct location, baseline performance drops sharply as the granularity shifts from file-level to module-level and function-level. Other metrics even approach zero. It indicates that, when deprived of explicit keywords and forced into deep tracing, they tend to resort to near-random guessing rather than structural reasoning. In contrast, \tooldataloc demonstrates strong stability, achieving a high HR of around 80\%. This robustness proves that \tooldataloc's success is not a byproduct of a coarse search space but is driven by rigorous, logic-based reasoning.

Baselines typically rely on top-$n$ recommendations to increase the probability of covering relevant locations, but this strategy is inherently a compromise rather than an optimal solution. An effective code localization tool should return results that precisely satisfy the query constraints, since the true number of relevant locations varies across tasks and is not predetermined. To evaluate this capability, we introduce two additional metrics: Average Jaccard Similarity (AJS) and Perfect Localization Rate (PLR). AJS penalizes both false positives and false negatives, while PLR represents the most stringent criterion, requiring the predicted set to exactly match the ground truth (i.e., achieving 100\% AJS). For instance, while LocAgent (Claude-3.5) achieves a 44\% hit rate at the file level, its AJS is only 1.57\%, indicating that true positives are diluted within an inflated candidate set containing substantial noise. By comparison, our approach consistently maintains high AJS scores, reflecting greater precision in returning constraint-satisfying results without extraneous recommendations. This precision is important for industrial deployment, as it reduces the validation overhead for developers or downstream automated agents, improving the efficiency of maintenance workflows.

Our investigation of SWE-bench Lite shows that most issues are highly localized, involving an average of only 1.15 code changes. This sparsity raises a key question: do existing tools truly pinpoint root causes, or do they merely rely on high-probability guessing within a narrow search space? To examine this, we use \negset to evaluate whether tools can recognize when no valid location exists. By modifying constraints, we deliberately created a mismatch between the issue description and the codebase, such that the original ground-truth locations are no longer valid. In this setting, the only correct output is a clear ``no match found''. Unfortunately, all SOTA baselines suffer from a compulsion to guess. They persistently return top-$n$ recommendations even when query prerequisites are not met. This over-eager behavior proves harmful in practice, as confident yet wrong targets mislead downstream agents, wasting computational resources, and risk introducing regression bugs. These findings suggest that the strong performance reported by existing baselines is partially inflated by their recommendation-centric design, which lacks true localization rationale. Notably, \tooldataloc demonstrates the necessary discernment to abstain when no valid location exists, returning a clear “no match found” response for over 70\% of the queries.
\begin{landscape}
\begin{table*}[t]
    \centering
    \scriptsize 
    \setlength{\tabcolsep}{1.2pt} 
    \caption{Evaluation results on LogicQuery}
    \label{tab:comparison_lq}
    
    \begin{tabularx}{\linewidth}{@{} l l *{18}{C} @{}} 
        \toprule
        \multirow{2}{*}{\textbf{Methods}} & \multirow{2}{*}{\textbf{LLM}} & \multicolumn{6}{c}{\textbf{File Level (\%)}} & \multicolumn{6}{c}{\textbf{Module Level (\%)}} & \multicolumn{6}{c}{\textbf{Function Level (\%)}} \\
        \cmidrule(lr){3-8} \cmidrule(lr){9-14} \cmidrule(lr){15-20}
        & & SR & REC & PRE & AJS & PLR & HR & SR & REC & PRE & AJS & PLR & HR & SR & REC & PRE & AJS & PLR & HR \\
        
        \midrule
        \multirow{2}{*}{SweRank} 
        & Small & 4.00 & 8.00 & 3.30 & 2.66 & 0 & 20.00 & 0 & 0 & 0 & 0 & 0 & 0 & 0 & 0 & 0 & 0 & 0 & 0 \\
        & Large & 0 & 8.13 & 4.67 & 3.47 & 0 & 20.00 & 0 & 6.04 & 2.92 & 2.18 & 0 & 16.67 & 0 & 4.76 & 2.38 & 1.87 & 0 & 14.29 \\

        \midrule
        \multirow{2}{*}{Agentless} 
        & GPT-4o & 0 & 0 & 0.80 & 0.50 & 0 & 4.00 & 0 & 0 & 0.35 & 0.28 & 0 & 4.17 & 0 & 0 & 0 & 0 & 0 & 0 \\
        & Claude-3.5 & 0 & 2.00 & 1.00 & 0.80 & 0 & 4.00 & 0 & 2.08 & 0.60 & 0.52 & 0 & 4.17 & 0 & 0 & 0 & 0 & 0 & 0 \\

        \midrule
        \multirow{2}{*}{LocAgent} 
        & GPT-4o & 12.00 & 2.00 & 1.37 & 1.12 & 0 & 20.00 & 4.17 & 0 & 0.85 & 0.62 & 0 & 12.50 & 0 & 0 & 0 & 0 & 0 & 0 \\
        & Claude-3.5 & 12.00 & 3.33 & 1.85 & 1.78 & 0 & 44.00 & 8.33 & 2.08 & 1.20 & 1.16 & 0 & 25.00 & 4.76 & 2.38 & 0.65 & 0.62 & 0 & 19.05 \\

        \midrule
        \multirow{2}{*}{CoSIL} 
        & GPT-4o & 0 & 1.00 & 1.00 & 0.57 & 0 & 4.00 & 0 & 0 & 0 & 0 & 0 & 0 & 0 & 0 & 0 & 0 & 0 & 0 \\
        & Claude-3.5 & 4.00 & 9.00 & 4.13 & 3.24 & 0 & 16.00 & 4.17 & 6.25 & 2.22 & 1.88 & 0 & 8.33 & 4.76 & 7.14 & 1.90 & 1.75 & 0 & 9.52 \\

        \midrule
        \multirow{2}{*}{\makecell{\tooldataloc}} 
        & Claude-3.5 & \textbf{64.00} & \textbf{61.00} & 62.45 & \textbf{57.05} & \textbf{44.00} & \textbf{80.00} & \textbf{62.50} & \textbf{60.42} & \textbf{60.85} & 55.12 & \textbf{41.67} & \textbf{79.17} & \textbf{61.90} & \textbf{62.70} & \textbf{63.63} & 57.09 & \textbf{42.86} & \textbf{80.95} \\
        & Qwen3-Max & \textbf{64.00} & 55.00 & \textbf{65.00} & 56.40 & \textbf{44.00} & 68.00 & 60.00 & 50.33 & 60.00 & \textbf{56.07} & 40.00 & 64.00 & 56.00 & 45.53 & 55.20 & \textbf{58.73} & 40.00 & 60.00 \\
        \bottomrule
    \end{tabularx}
\end{table*}
\end{landscape}

\smallskip
\noindent\shadowbox{%
  \begin{minipage}{0.98\columnwidth}
    \textbf{Answer to RQ1:}
		The results clearly show that \tooldataloc effectively handles the keyword-agnostic logical code localization challenge, whereas baselines perform poorly when keyword shorts are unavailable. This means these approaches still rely on shallow lexical matching rather than genuine logical reasoning. Furthermore, their performance on the \negset exposes fundamental weakness in their refusal ability.
  \end{minipage}}
%%% Local Variables:
%%% mode: latex
%%% TeX-master: "../main"
%%% End:

\subsubsection{Effectiveness for general issues}

To evaluate the practical utility of \tooldataloc in real-world software maintenance, we conducted a comparative analysis on the SWE-bench Lite. Existing approaches usually employ top-$n$ strategy, whereas \tooldataloc operates without a predefined $n$. Table~\ref{tab:comparison_isq} illustrates that \tooldataloc remains highly competitive.

A critical distinction lies in the recommendation density and target precision. While \tooldataloc provides instance-specific localization with an average of only 2 candidates per issue, SOTA baselines rely on much broader and often fixed-size candidate sets to improve their accuracy. Specifically, CoSIL and LocAgent typically default to a top-$5$ recommendation at the function level, while Agentless routinely recommends 5 locations as candidates regardless of the issue's actual complexity. SweRank adopts an even more aggressive strategy, utilizing top-100 rankings.

Consequently, even when Acc@$n$ metrics appear comparable, \tooldataloc achieves substantially higher precision. By providing a concise and accurate set of entry points, \tooldataloc minimizes the noise that developers or downstream agents must filter, reducing validation overhead. This precision-centric design also yields substantial gains in resource efficiency (details in Section~\ref{sec:cost}). 

% Compared to agent-based baselines that require extensive multi-turn interactions, \tooldataloc operates with remarkably low latency and token consumption by focusing only on logically necessary code segments. We also observed that this efficiency contrasts with the operational instability of tools like LocAgent, which frequently falls into infinite execution loops on complex or mutated instances. While we will provide a comprehensive breakdown of these overhead metrics in RQ3, these preliminary results highlight that \tooldataloc delivers a superior balance of competitive recall, surgical precision, and minimal resource expenditure.


\begin{table}[t]
\centering
\caption{Comparison of different methods and models across various localization granularities.}
\label{tab:comparison_isq}
\small 
\begin{tabularx}{\textwidth}{@{} ll CCCCC CC @{}}
\toprule
\multirow{2}{*}{\textbf{Method}} & \multirow{2}{*}{\textbf{Model}} & \multicolumn{3}{c}{\textbf{File (\%)}} & \multicolumn{2}{c}{\textbf{Module (\%)}} & \multicolumn{2}{c}{\textbf{Function (\%)}} \\ 
\cmidrule(lr){3-5} \cmidrule(lr){6-7} \cmidrule(lr){8-9}
& & Acc@1 & Acc@3 & Acc@5 & Acc@5 & Acc@10 & Acc@5 & Acc@10 \\ 
\midrule
\multirow{2}{*}{SweRank}   & Small           & 78.10 & 92.34 & 94.53 & 89.05 & 92.70 & 79.56 & 86.13 \\
                           & Large           & 83.21 & 94.89 & 95.99 & 90.88 & 93.43 & 81.39 & 88.69 \\ 
\midrule
% \addlinespace[0.5em] 
\multirow{2}{*}{Agentless} & gpt-4o          & 67.50 & 74.45 & 74.45 & 67.15 & 67.15 & 55.47 & 55.47 \\
                           & claude-3.5      & 72.63 & 79.20 & 79.56 & 68.98 & 68.98 & 58.76 & 58.76 \\ 
\midrule
% \addlinespace[0.5em]
\multirow{2}{*}{LocAgent}  & Qwen2.5-32B(ft) & 75.91 & 90.51 & 92.70 & 85.77 & 87.23 & 71.90 & 77.01 \\
                           & claude-3.5      & 77.74 & 91.97 & 94.16 & 86.50 & 87.59 & 73.36 & 77.37 \\ 
\midrule
% \addlinespace[0.5em]
\multirow{2}{*}{DataLoc}   & gpt-5.1         & 71.53 & 77.38 & 78.47 & 70.80 & 72.26 & 63.14 & 64.96 \\
                           & claude-3.5      & 72.26 & 80.66 & 81.02 & 75.55 & 75.55 & 68.98 & 68.98 \\ 
\bottomrule
\end{tabularx}
\end{table}

\smallskip
\noindent\shadowbox{%
  \begin{minipage}{0.98\columnwidth}
    \textbf{Answer to RQ2:}
		\tooldataloc also demonstrates competitive performance on issue-solving benchmarks. Notably, given that \tooldataloc produces only 2 candidate locations on average, it offers distinct advantages in recommendation efficiency and in excluding incorrect locations. This precise localization helps reduce the potential overhead of downstream tasks.
  \end{minipage}}
\subsubsection{Efficiency}\label{sec:cost-dataloc}
Figure~\ref{fig:cost_analysis} presents a comprehensive comparison using a lollipop chart, where the vertical axis represents average execution time (seconds) and bubble size reflects total token consumption (labeled in thousands). As illustrated, \tooldataloc (Claude3.5) establishes a new efficiency frontier for KA-LCL tasks, achieving the optimal balance between speed and token consumption with 37 seconds execution time and a 13.5k token consumption. Compared to other approaches, \tooldataloc exhibits a clear dual advantage in both temporal efficiency and token economy:

\textbf{Temporal efficiency.} \tooldataloc (Claude-3.5) completes localization in around half minute per task, outperforming all agentic baselines. Even the fastest CoSIL (GPT-4o) remain over 3$\times$ slower. The poor performance in Table~\ref{tab:comparison_lq} further indicate that, without keyword shortcuts, baseline methods fail to perform meaningful repo-level inference, despite exhibiting imtermediate reasoning steps.

\textbf{Token economy.} Approaches that rely more on agents incur substantially higher token consumption, with LocAgent and Agentless consuming over an order of magnitude more tokens per task than \tooldataloc. This token explosion reflects a trade-off where tokens are exchanged for intelligence, but this intelligence is currently tie to text. As a result, without keyword shortcuts to guide retrieval, these agents are trapped in multi-round conversation and iterative codebase exploration. In our evaluation, three queries causes LocAgent to enter infinite loops without producing results.

The efficiency of \tooldataloc stems from its hybrid architecture. In our design, the LLM-based agent is strategically confined to high-level tasks: query analysis, Datalog program synthesis, and final candidate verification. By shifting deep inference to a specialized engine, \tooldataloc greatly reduces the overhead of redundant multi-round exploration. In addition, errors carry small cost, requiring only the regeneration of a Datalog program. Beyond this low overhead, our parser-gated validation and intermediate-rule feedback actively assist LLM to synthesize high-quality programs.



\begin{figure}[t]
	\centering
	\includegraphics[width=\textwidth]{Figures/Chapter5/cost.png}
	\caption{Average execution time and token consumption of \dataset}
	\label{fig:cost_analysis}
\end{figure}

\smallskip
\noindent\shadowbox{%
  \begin{minipage}{0.98\columnwidth}
    \textbf{Answer to RQ3:}
		\tooldataloc defines the efficiency frontier in the KA-LCL challenge, maintaining an average execution time of 37s and a token consumption of 13.5k. By replacing expensive LLM-based exploration with Datalog-driven inference, \tooldataloc bypasses the prohibitive costs and execution loops of existing agentic baselines, demonstrating its potential industrial deployment.
  \end{minipage}}
\subsubsection{Ablation Study}\label{sec:ablation}
We design an ablation study to quantify how two mechanisms detailed in~\cref{sec:method:val} and \cref{sec:method:int} improve the quality of LLM-generated Datalog program.

%where ``quality'' encompasses (i) executability (passing parsing and evaluation), (ii) result usefulness (avoiding silent empty outputs caused by mistakes), and (iii) convergence efficiency (reaching a working query with fewer LLM iterations).

%For brevity, we refer to these two types of mechanism as \emph{validation and repair} (VAL) and \emph{intermediate-rule feedback} (INT), respectively.

\textbf{Configurations}
We evaluate three configurations that progressively enable these mechanisms:
\textbf{Base} directly executes the LLM-generated Datalog program without any validation or additional feedback, except the output or error messages from \souffle{} itself. \textbf{VAL} (validation and repair) enables parser-gated validation with deterministic syntactic repairs and high-confidence semantic checks. \textbf{Full} further enables intermediate-rule mutation and feedback.
%
All configurations share the same LLM, initial prompt, iteration budget, and underlying fact bases.
%Each natural-language query is evaluated independently.

\textbf{Metrics}
We compare  (i) execution success rate and non-empty result rate, (ii) mean LLM iterations to first successful execution and (iii) final answer correctness on \dataset.

% \usepackage{siunitx}
\begin{table}[t]
  \centering
  \small
  \setlength{\tabcolsep}{4.0pt}
  \renewcommand{\arraystretch}{1.15}
  \caption{Ablation results under two LLMs. \textbf{Base}: no mechanism;
  \textbf{VAL}: validation \& repair; \textbf{Full}: VAL+intermediate-rule feedback (INT).
  Rates are reported in \%. Iteration and time metrics report mean values (lower is better). Time is reported in seconds.}
  \label{tab:ablation}

  \resizebox{\columnwidth}{!}{%
  \begin{tabular}{ll
                  cc
                  cc c
                  cccccc}
    \toprule
    \multirow{2}{*}{\textbf{Model}} &
    \multirow{2}{*}{\textbf{Config}} &
    \multicolumn{2}{c}{\textbf{Rates} $\uparrow$} &
    \multicolumn{3}{c}{\textbf{Cost (Iter, Time)} $\downarrow$} &
    \multicolumn{6}{c}{\textbf{Correctness (function-level)} $\uparrow$} \\
    \cmidrule(lr){3-4}
    \cmidrule(lr){5-7}
    \cmidrule(lr){8-13}
    & & \textbf{ExecSucc} & \textbf{$\ne\varnothing$} &
    \textbf{First} & \textbf{Iter} & \textbf{Time} &
    \textbf{SR} & \textbf{PRE} & \textbf{REC} & \textbf{AJS} & \textbf{PLR} & \textbf{HR} \\
    \midrule

    \multirow{3}{*}{Qwen3-Max} &
    Base   & 73.59 & 49.41 & 1.48 & 12.92 & 427 & 20.00 & 18.20 & 15.87 & 26.73 & 16.00 & 20.00 \\
    & VAL  & 84.12 & 75.68 & 1.28 & 10.16 & 104 & 40.00 & 38.93 & 31.87 & 43.67 & 32.00 & 44.00 \\
    & Full & 84.12 & 78.52 & 1.24 & 10.92 & 136 & 56.00 & 55.20 & 45.53 & 58.73 & 40.00 & 60.00 \\
    \midrule

    \multirow{3}{*}{Claude-3.5} &
    Base   & 89.42 & 67.95 & 1.16 & 4.20 & 38 & 44.00 & 38.10 & 33.87 & 43.70 & 24.00 & 52.00 \\
    & VAL  & 94.67 & 75.86 & 1.04 & 4.28 & 40 & 44.00 & 38.90 & 31.87 & 48.36 & 32.00 & 52.00 \\
    & Full & 94.67 & 80.69 & 1.08 & 4.08 & 37 & 61.90 & 62.70 & 63.63 & 57.09 & 42.86 & 80.95 \\
    \bottomrule
  \end{tabular}}

  \vspace{1mm}
  \footnotesize\raggedright
  \textbf{Cost metrics.}
  \textbf{$\ne\varnothing$} denotes the fraction of queries that produce non-empty answers.
  \textbf{First} is the mean number of LLM iterations to the first successfully executing program.
  \textbf{Iter} is the average total number of LLM iterations consumed per query.
  \textbf{Time} reports the mean end-to-end wall-clock time per query, including failed attempts and tool feedback.
\end{table}

\ref{tab:ablation} summarizes the ablation results under two different LLMs, comparing the baseline system with progressively enabled mechanisms.
The results show that validation and repair (VAL) substantially improves the quality of LLM-generated Datalog for both models, with a markedly stronger effect on Qwen3 Max than on Claude 3.5 Sonnet. This difference is expected given the models' baseline capabilities: without VAL, Qwen3 Max exhibits a significantly lower execution success rate due to its weaker ability to consistently produce syntactically correct \souffle{} Datalog, whereas Claude already achieves a relatively high baseline level of syntactic validity.

For Qwen3 Max, enabling VAL leads to a dramatic improvement across nearly all metrics. The non-empty result rate increases from 49\% to 75\%, indicating that a large fraction of previously failing or unproductive queries were recoverable once parser-gated validation and conservative repairs were applied. At the same time, the average end-to-end execution time drops sharply from 427 seconds to 104 seconds, reflecting a reduction in wasted iterations caused by unrecoverable parser errors and repeated failed executions. Improvements are also reflected in downstream task quality: function-level correctness metrics show substantial gains, with precision increasing from 18\% to 50\%, demonstrating that VAL does not merely enable execution but also materially improves the semantic adequacy of the resulting queries.
In contrast, the effect of VAL on Claude is more moderate: execution success rate increases by +5 percentage points, and the non-empty result rate increases by +8 percentage points. 

Enabling intermediate-rule feedback (INT) in the Full configuration produces a different pattern. The primary role of INT is to guide the LLM toward identifying which specific rule is semantically invalid in the sense of producing no results, and how that rule can be locally revised. As a result, INT may slightly increase iteration count or execution time in some cases due to additional diagnostic executions; however, this overhead consistently translates into higher non-empty rates and improved final answer correctness.

\smallskip
\noindent\shadowbox{%
  \begin{minipage}{0.98\columnwidth}
    \textbf{Answer to RQ4:} Validation and repair (VAL) substantially improves the robustness of
    LLM-generated Datalog, with particularly large gains for weaker models by increasing non-empty
    result rates and reducing execution cost. Intermediate-rule feedback (INT) complements VAL by
    guiding targeted revisions of logically unproductive rules, occasionally incurring additional
    diagnostic cost but improving final answer correctness across models.
  \end{minipage}}
%%% Local Variables:
%%% mode: latex
%%% TeX-master: "../main"
%%% End:


%%% Local Variables:
%%% mode: latex
%%% TeX-master: "../main"
%%% End:

\section{Threats to Validity}
\textbf{Internal Validity.} The primary internal threats concern baseline implementation and data leakage. We use the official implementations of all baselines with default configurations. For baselines that rely on large language models, we employ the same underlying models (GPT-4o and Claude-3.5-Sonnet) to avoid model-related bias. The KA-LCL queries in \dataset and \negset are constructed based on the decomposed code structure, resulting in novel query instances. Candidate ground truth are generated by state-of-the-art models (GPT-5.2, Claude-4.5-Opus, and Gemini-3-Pro) under advanced AIDE's agent mode and determined through independent manual verification by two authors. These measures eliminate the risk of data leakage.

\textbf{External Validity.} A main threat to external validity is that our current evaluation focused only on Python. Although the proposed framework is language-agnostic in principle, extending it to additional languages or incorporating more analysis results into program facts remains future work, and addressing this challenge needs further engineering efforts to broaden the applicability. 
\section{Related Work}\label{sec:related}
\noindent \textbf{LLM for Issue Resolution and Question Answering.} 
Large language models are increasingly integrated into software engineering, motivating a wide range of benchmarks for codebase question answering and issue resolution. Existing benchmarks fall into two main categories. Code QA benchmarks~\cite{strich2024improving, li2024infibench, li2024procqa} evaluate models on either code snippets (e.g. CodeQueries~\cite{sahu2024codequeries}, CodeQA~\cite{liu2021codeqa}, CoSQA~\cite{liu2021codeqa}) or repository-level contexts derived from GitHub issues (e.g. CodeRepoQA~\cite{hu2024coderepoqa}, CoReQA~\cite{chen2025coreqa}, SWE-QA~\cite{peng2025swe}). End-to-end issue resolution benchmarks, such as SWE-Bench~\cite{jimenez2023swe} and its extensions, assess full issue-solving capabilities~\cite{jimenez2023swe,chen-etal-2025-locagent,zhuo2024bigcodebench,deng2025nocode,niu2023crosscodebench,chen2021evaluating,ouyang2024benchmarking,gao2023benchmarking,jiang2024collubenchbenchmarkpredictinglanguage,jain2024r2e,mundler2024swt,xie2024osworld, yang2025swesmith}, primarily focus on bug-fix tasks and limited programming languages. However, excessive keywords in these benchmarks provide models with too many shortcuts for code localization by superficial lexical matching. To mitigate this bias, we propose \dataset, which removes semantic keywords and only keeps logic structures, and \negset, an empty-ground-truth variant designed to evaluate abstention ability.


% Large language models are increasingly integrated into software engineering, where developers rely on them for issue resolution and question answering for complex projects. 
% On one hand, many code question answering benchmarks have been developed to evaluate QA systems. Snippet-level benchmarks like CodeQueries~\cite{sahu2024codequeries}, CodeQA~\cite{liu2021codeqa}, and CoSQA~\cite{liu2021codeqa} focus on isolated code snippets or single functions. Repository-level benchmarks such as CodeRepoQA~\cite{hu2024coderepoqa}, CoReQA~\cite{chen2025coreqa} and SWE-QA~\cite{peng2025swe} collect data from Github issues, while Spyder-CodeQA~\cite{strich2024improving} provides only QA pairs from one project. InfiBench~\cite{li2024infibench} and ProCQA~\cite{li2024procqa} focus on general programming tasks. One the other hand, numerous benchmarks have emerged from efforts to assess end-to-end issue-solving ability~\cite{jimenez2023swe,chen-etal-2025-locagent,zhuo2024bigcodebench,deng2025nocode,niu2023crosscodebench,chen2021evaluating,ouyang2024benchmarking,gao2023benchmarking,jiang2024collubenchbenchmarkpredictinglanguage,jain2024r2e,mundler2024swt,xie2024osworld}. SWE-Bench~\cite{jimenez2023swe} is among the most widely adopted, comprising 2,294 issue–pull request pairs from 12 open-source Python projects, primarily targeting bug-fix tasks. Its lightweight variant, SWE-Bench Lite, condenses the benchmark into 300 curated issues, making it a practical and standardized evaluation set for subsequent studies.  To overcome the limitation of single programming language evaluation, SWE-Bench Multilingual~\cite{yang2025swesmith} introduces 300 tasks from 42 additional repositories across nine programming languages. Expanding beyond bug fixing, LocBench~\cite{chen-etal-2025-locagent} extends coverage to 560 issues spanning a broader range of task types. 


% Unlike prior studies that evaluate LLMs on a single programming language (e.g., Python) or a single benchmark, we evaluate \tool across multiple programming languages (Java and Python) and multiple benchmarks, including LocBench, SWE-Bench, and SWE-Bench Multilingual.

% reference:https://arxiv.org/html/2503.22424v1

\noindent \textbf{Code Localization.}  
Code localization refers to identifying relevant code locations (e.g., files, modules, or functions) to resolve developers' queries. 
Recent advancements in this area have taken two complementary directions. 
Meanwhile, LLM-based retrieval techniques have been proposed to improve code localization performance by leveraging semantic understanding~\cite{xia2024agentless,chen-etal-2025-locagent,reddy2025swerank,wang2024openhands,yang2024swe,jiang2025cosil,zhang2024autocoderover,tao2024magis,xie2025swe,ma2025sorft}.
Recent research has proposed numerous code localization approaches that can be broadly categorized
into three classes: (1) \textit{Embedding-based approaches} (e.g.,
SWERankEmbed~\cite{reddy2025swerank}, CodeSage~\cite{zhang2024code}) encode code entities and natural language descriptions as
embeddings, ranking them based on semantic similarity. While they achieve high recall by retrieving
a broad range of relevant candidate locations, they can only identify code snippets that ``look
similar'' without understanding the logical relationships between them.
Furthermore, they suffer from hallucination and noise interference, such as methods with identical
names but entirely different functionalities.
(2) \textit{Pipeline-based LLM approaches} (e.g., Agentless~\cite{xia2024agentless}) follow a
structured, multi-stage workflow from files to functions.
However, such hierarchical localization design overly depends on initial file-level
localization and fails to capture cross-level dependencies, implicitly assuming that developers
adhere to good naming conventions.
(3) \textit{Agent-based LLM approches} (e.g., LocAgent~\cite{chen-etal-2025-locagent},
CoSIL~\cite{jiang2025cosil}, Orca Loca~\cite{yu2025orcalocallmagentframework}, GraphLocator~\cite{liu2025graphlocator}) offer greater flexibility by
allowing LLMs to autonomously traverse the repository graph.
Nevertheless, they only consider surface-level relevance and exhibit rapid performance degradation
without explicit contextual guidance.

% AIDE, such as \texttt{Cursor}~\cite{cursor_ai_editor}, \texttt{Gemini-CLI}~\cite{gemini_cli}, and \texttt{Claude-code}~\cite{claude_code}, have emerged to assist developers by suggesting or navigating to relevant code. 


% CodeQA
% CodeQL
% code search
% BM25 retrieval


%----------------------------------------------------------------------------------------
%	THESIS CONTENT - APPENDICES
%----------------------------------------------------------------------------------------

\addtocontents{toc}{\vspace{0.8em}} % Add a gap in the Contents, for aesthetics

\appendix % Cue to tell LaTeX that the following 'chapters' are Appendices

% Include the appendices of the thesis as separate files from the Appendices folder
% Uncomment the lines as you write the Appendices

\input{./Appendices/AppendixA}
%\input{./Appendices/AppendixB}

\cleardoublepage    % start a new page

%----------------------------------------------------------------------------------------
%	AUTHOR'S PUBLICATIONS
%----------------------------------------------------------------------------------------

\authorpublications{

\section*{Awards}

\begin{itemize}
	\item \textbf{Best Paper Awards}, ``A Great System,'' \emph{Nature}.
\end{itemize}

\section*{Patents}
\begin{itemize}
	\item \textbf{A Great System}, ``A Great System,'' \emph{Nature}.
\end{itemize}

\section*{Journal Articles}

\begin{itemize}
    \item \textbf{My name} and My colleague, ``A Great System,'' \emph{Nature}.
\end{itemize}

\section*{Conference Proceedings}

\begin{itemize}
  \item \textbf{My name}, My colleague 1, My colleague 3 and My colleague 3, ``Greater System,'' in \emph{Conference of Vision, 2018}.
\end{itemize}

}

\backmatter

%----------------------------------------------------------------------------------------
%	BIBLIOGRAPHY
%----------------------------------------------------------------------------------------

\label{Bibliography}
\setstretch{1}
\bibliographystyle{unsrtnat}
\bibliography{References/first_paper,References/second_paper,References/third_paper}

\end{document}
